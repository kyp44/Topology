\setcounter{subsection}{17-1}
\subsection{Closed Sets and Limit Points}

\exercise{1}{
  Let $\cC$ be a collection of subsets of the set $X$.
  Suppose that $\es$ and $X$ are in $C$, and that finite unions and arbitrary intersections of elements of $\cC$ are in $\cC$.
  Show that the collection
  \gath{
    \cT = \braces{X - C \where C \in \cC}
  }
  is a topology on $X$.
}
\sol{
  \dwhitman

  \qproof{
    First, clearly $\es$ and $X$ are in $\cT$ since $\es = X - X$ and $X = X - \es$ and both $X$ and $\es$ are in $\cC$.
    This shows the first defining property of a topology.

    Now consider an arbitrary subcollection $\cA$ of $\cT$.
    Then, for each $A \in \cA$, we have that $A = X - B$ for some $B \in \cC$ since also $A \in \cT$.
    So let $\cB = \braces{B \in \cC \where X - B \in \cA}$.
    Then we have that
    \gath{
      \bigcup \cA = \bigcup_{A \in \cA} A = \bigcup_{B \in \cB} (X - B) = X - \bigcap_{B \in \cB} B = X - \bigcap \cB
    }
    by DeMorgan's law.
    By the definition of $\cC$ we have that $\bigcap \cB \in \cC$ since it is an arbitrary intersection of elements of $\cC$.
    It then follows that $\bigcup \cA = X - \bigcap \cB$ is in $\cT$ by definition.
    This shows the second defining property of a topology.

    Lastly, suppose that $\cA$ is a nonempty finite subcollection of $\cT$, which of course can be expressed as $\cA = \braces{A_k \where k \in \intsfin{n}}$ for some positive integer $n$.
    Then, again we have that that $A_k = X - B_k$ for some $B_k \in \cC$ for all $k \in \intsfin{n}$ since $A_k \in \cT$.
    Then we have
    \gath{
      \bigcap \cA = \bigcap_{k=1}^n A_k = \bigcap_{k=1}^n (X - B_k) = X - \bigcup_{k=1}^n B_k
    }
    by DeMorgan's law.
    Then clearly $\bigcup_{k=1}^n B_k$ is in $\cC$ by definition since it is a finite union of elements of $\cC$.
    It then follows that $\bigcap \cA = X - \bigcup_{k=1}^n B_k$ is in $\cT$ by definition.
    Since $\cA$ was an arbitrary finite subcollection, this shows the third defining property of a topology.
    Hence $\cT$ is a topology by definition.
  }
}

\exercise{2}{
  Show that if $A$ is closed in $Y$ and $Y$ is closed in $X$, then $A$ is closed in $X$.
}
\sol{
  \dwhitman

  \qproof{
    Since $A$ is closed in $Y$, it follows from Theorem~17.2 that $A = B \cap Y$ where $B$ is some closed set in $X$.
    Hence by definition $X - B$ is open in $X$.
    Also, since $Y$ is closed in $X$, we have that $X - Y$ is open in $X$ by definition.
    We then have
    \gath{
      X - A = X - (B \cap Y) = (X - B) \cup (X - Y)
    }
    by DeMorgan's law.
    Since both $X - B$ and $X - Y$ are open in $X$, clearly their union must also be open since we are in a topological space.
    Hence $X - A$ is open in $X$ so that $A$ is closed in $X$ by definition.
  }
}

\exercise{3}{
  Show that if $A$ is closed in $X$ and $B$ is closed in $Y$, then $A \times B$ is closed in $X \times Y$.
}
\sol{
  \dwhitman
  \begin{lem}\label{lem:closedlim:setid}
    If $X$, $Y$, $A$, and $B$ are sets then $X \times Y - A \times B = (X-A) \times Y \cup X \times (Y - B)$.
  \end{lem}
  \qproof{
    We show this via logical equivalences:
    \ali{
      (x,y) \in X \times Y - A \times B &\bic (x,y) \in X \times Y \land (x,y) \notin A \times B \\
      &\bic (x \in X \land y \in Y) \land \lnot(x \in A \land y \in B) \\
      &\bic (x \in X \land y \in Y) \land (x \notin A \lor y \notin B) \\
      &\bic (x \in X \land y \in Y \land x \notin A) \lor (x \in X \land y \in Y \land y \notin B) \\
      &\bic (x \in X - A \land y \in Y) \lor (x \in X \land y \in Y - B) \\
      &\bic (x,y) \in (X-A) \times Y \lor (x,y) \in X \times (Y-B) \\
      &\bic (x,y) \in (X-A) \times Y \cup X \times (Y-B)
    }
    as desired.
  }

  \mainprob
  \qproof{
    Since $A$ is closed we have that $X-A$ is open in $X$.
    Since also $Y$ itself is open in $Y$, we have that $(X-A) \times Y$ is a basis element in the product topology by definition, and is therefore obviously open.
    An analogous argument shows that $X \times (Y-B)$ is also open in the product topology since $B$ is closed in $Y$.
    Hence their union is also open in the product topology, but by Lemma~\ref{lem:closedlim:setid} we have
    \gath{
      (X-A) \times Y \cup X \times (Y-B) = X \times Y - A \times B
    }
    so that $X \times Y - A \times B$ is also open in the product topology.
    It then follows by definition that $A \times B$ is closed as desired.
  }
}

\exercise{4}{
  Show that if $U$ is open in $X$ and $A$ is closed in $X$, then $U-A$ is open in $X$, and $A-U$ is closed in $X$.
}
\sol{
  \dwhitman

  \begin{lem}\label{lem:closedlim:setmid}
    If $A$, $B$, and $C$ are sets then $A - (B - C) = (A-B) \cup (A \cap C)$.
  \end{lem}
  \qproof{
    We show this by a sequence of logical equivalences:
    \ali{
      x \in A - (B - C) &\bic x \in A \land x \notin B - C \\
      &\bic x \in A \land \lnot(x \in B \land x \notin C) \\
      &\bic x \in A \land (x \notin B \lor x \in C) \\
      &\bic (x \in A \land x \notin B) \lor (x \in A \land x \in C) \\
      &\bic x \in A - B \lor x \in A \cap C \\
      &\bic x \in (A-B) \cup (A \cap C)
    }
    as desired.
  }

  \begin{cor}\label{cor:closedlim:cc}
    If $A \ss X$ and $B = X - A$, then $A = X - B$.
  \end{cor}
  \qproof{
    By Lemma~\ref{lem:closedlim:setmid}, we have that
    \gath{
      X - B = X - (X - A) = (X - X) \cup (X \cap A) = \es \cup (X \cap A) = X \cap A = A
    }
    since $A \ss X$.
  }

  \mainprob
  \qproof{
    First, since $A$ is closed in $X$, we have that $B = X - A$ is open in $X$, and it follows from Corollary~\ref{cor:closedlim:cc} that $A = X - B$.
    Then we have that
    \gath{
      U - A = U - (X - B) = (U - X) \cup (U \cap B)
    }
    by Lemma~\ref{lem:closedlim:setmid}.
    Since $U \ss X$, it follows that $U - X = \es$, and hence
    \gath{
      U - A = \es \cup (U \cap B) = U \cap B \,.
    }
    Then, since both $U$ and $B$ are open, their intersection is as well and therefore $U - A$ is open.

    Next, we have by Lemma~\ref{lem:closedlim:setmid}
    \gath{
      X - (A - U) = (X - A) \cup (X \cap U) = B \cup (X \cap U) = B \cup U.
    }
    since $U \ss X$ so that $X \cap U = U$.
    Since both $B$ and $U$ are open, clearly their union is as well and hence $X - (A - U)$ is open.
    This of course means that $A - U$ is closed by definition.
  }
}

\exercise{5}{
  Let $X$ be an ordered set in the order topology.
  Show that $\closure{(a,b)} \ss [a,b]$.
  Under what conditions does equality hold?
}
\sol{
  \dwhitman

  \qproof{
    First, the closed interval $[a,b]$ is closed (hence why it is called such!) because clearly its compliment is
    \gath{
      X - [a,b] = (-\infty, a) \cup (b, \infty)
    }
    and we know that open rays are always open so that their union is as well.
    Clearly also $[a,b]$ contains $(a,b)$.
    Hence $[a,b]$ is a closed set containing $(a,b)$.
    Since $\closure{(a,b)}$ is defined as the intersection of closed sets that contain $(a,b)$ clearly we have that $\closure{(a,b)} \ss [a,b]$ as desired.
  }

  The conditions required for equality are such that $[a,b]$ is also a subset of $\closure{(a,b)}$ and, in particular both $a$ and $b$ must be in $\closure{(a,b)}$.
  Since clearly $a,b \notin (a,b)$, it has to be that they are both limit points of $(a,b)$.
  This is equivalent to the condition that $a$ has no immediate successor and $b$ no immediate predecessor.
  We show only the first of these since the second is analogous.
  \qproof{
    $(\imp)$ We show the contrapositive of this.
    So suppose that $a$ \emph{does} have an immediate successor $c$.
    Then the open ray $(-\infty, c)$ is an open set that contains $a$ but does not intersect $(a,b)$.
    This is easy to see, because if they did intersect, there would be an $x \in (a,b)$ where also $x \in (-\infty, c)$.
    From these it follows that $a < x < c$, which contradicts the fact that $c$ is the immediate successor of $a$.
    Hence by definition $a$ is not a limit point of $(a,b)$.

    $(\pmi)$ Suppose that $a$ is not a limit point of $(a,b)$.
    Then there is an open set $U$ containing $a$ that does not intersect $(a,b)$.
    From this it follows that there is a basis element $B$ containing $a$ such that $B \ss U$, and thus $B$ also cannot intersect $(a,b)$ (as, if it did, then so would $U$).
    Suppose that $B$ is the open interval $(c,d)$ so that $c < a < d$.
    It also must be that $d < b$ for otherwise, for any element of $x$ of $(a,b)$, we would have $c < a < x < b \leq d$ so that $x \in (c,d) = B$ and $B$ and $(a,b)$ would not be disjoint.
    We claim that $d$ is the immediate successor of $a$.
    If this is not the case then there would be an $x$ such that $c < a < x < d$ and hence $x \in (c,d) = B$.
    Also $a < x < d < b$ so that also $x \in (a,b)$.
    Therefore $B$ and $(a,b)$ would not be disjoint.
    Similar arguments can be made if $B$ are other types of basis element in the order topology.
    (Actually $B$ cannot be of the form $\opcl{e,f}$ for largest element $f$ of $X$ since then any element of $(a,b)$ would also be in $B$ and they would not be disjoint.)
  }

  It is also worth noting that the Hausdorff axiom (and therefore also the T1 axiom since it is implied by the Hausdorff axiom) is not sufficient for general equivalence of $[a,b]$ and $\closure{(a,b)}$.
  For example the order topology on $\ints$ results in the discrete topology so that every subset is both open and closed.
  Thus for any pair $x_1, x_2$ in $\ints$, the sets $\braces{x_1}$ and $\braces{x_2}$ are neighborhoods of $x_1$ and $x_2$, respectively, that are disjoint.
  This shows that this topology is a Hausdorff space.
  However, the fact that $a$ has an immediate successor in $\pints$ is sufficient to show that $[a,b] \neq \closure{(a,b)}$ per what was just shown above.
}

\exercise{6}{
  Let $A$, $B$, and $A_\a$ denote subsets of a space $X$.
  Prove the following:
  \eparts{
  \item If $A \ss B$, then $\clA \ss \clB$.
  \item $\closure{A \cup B} = \clA \cup \clB$.
  \item $\closure{\bigcup A_a} \sps \bigcup \clA_\a$; give an example where equality fails.
  }
}
\sol{
  \dwhitman

  (a)
  \qproof{
    Suppose that $A \ss B$ and consider any $x \in \clA$.
    Consider any neighborhood $U$ of $x$ so that $U$ intersects $A$ by Theorem~17.5a.
    Hence there is a point $y \in U \cap A$ so that $y \in U$ and $y \in A$.
    But then clearly $y \in B$ also since $A \ss B$.
    Therefore $y \in U \cap B$ so that $U$ intersects $B$.
    Since $U$ was an arbitrary neighborhood of $x$, this shows that $x \in \clB$, again by Theorem~17.5a.
    This of course shows that $\clA \ss \clB$ as desired since $x$ was arbitrary.
  }

  (b)
  \qproof{
    $(\ss)$
    We show this by contrapositive.
    So suppose that $x \notin \clA \cup \clB$.
    Then clearly $x \notin \clA$ and $x \notin \clB$.
    Thus, by Theorem~17.5a, there is an open set $U_A$ such that $U_A$ does not intersect $A$, and likewise an open $U_B$ that does not intersect $B$.
    Let $U = U_A \cap U_B$, which is clearly open since $U_A$ and $U_B$ are.
    We also note that $U$ contains $x$ since both $U_A$ and $U_B$ do.
    Then it must be that $U$ does not intersect $A$ since, if it did, then $U_A$ would also intersect $A$ since $U \ss U_A$.
    Similarly, $U$ cannot intersect $B$.
    Thus, for all $y \in U$, $y \notin A$ and $y \notin B$.
    This is logically equivalent to saying that there is no $y \in U$ where $y \in A$ or $y \in B$, therefore there is no $y \in U$ where $y \in A \cup B$.
    Hence $U$ and $A \cup B$ do not intersect.
    Since $U$ is open and contains $x$, this shows that $x \notin \closure{A \cup B}$, again by Theorem~17.5a.
    Therefore, by contrapositive, $x \in \closure{A \cup B}$ implies that $x \in \clA \cup \clB$ so that $\closure{A \cup B} \ss \clA \cup \clB$.

    $(\sps)$
    Consider any $x \in \clA \cup \clB$ and any neighborhood $U$ of $x$.
    If $x \in \clA$ then $U$ intersects $A$ by Theorem~17.5a.
    Hence there is a $y \in U \cap A$ so that $y \in U$ and $y \in A$.
    Then clearly $y \in A \cup B$ so that $y$ is also in $U \cap (A \cup B)$.
    Hence $U$ intersects $A \cup B$.
    An analogous argument shows that this is also true if $x \in \clB$ instead.
    Since $U$ was an arbitrary neighborhood, this shows that $x \in \closure{A \cup B}$ by Theorem~17.5a.
    Hence $\clA \cup \clB \ss \closure{A \cup B}$ since $x$ was arbitrary.
  }

  (c)
  \qproof{
    Consider any $x \in \bigcup \clA_\a$ so that there is a particular $\b$ where $x \in \clA_\b$.
    Suppose that $U$ is any open set containing $x$ so that $U$ intersects $A_\b$ by Theorem~17.5a since $x \in \clA_\b$.
    Then clearly $U$ also intersects $\bigcup A_\a$ since $A_\b \ss \bigcup A_\a$.
    Since $U$ was an arbitrary open set containing $x$, this shows that $x \in \closure{\bigcup A_\a}$ by Theorem~17.5a.
    This shows that $\bigcup \clA_\a \ss \closure{\bigcup A_\a}$ since $x$ was arbitrary, which is of course the desired result.
  }

  As an example where equality fails, consider the standard topology on $\reals$ and the sets $A_n = \opcl{1/n, 2}$ for $n \in \pints$.
  It is then trivial to show that $\bigcup A_n = \opcl{0, 2}$ so that clearly $0$ is a limit point of $\bigcup A_n$, and hence $0 \in \closure{\bigcup A_n}$.
  However, for any $n \in \pints$, the open interval $(-1, 1/n)$ is clearly an open set containing $0$ that is disjoint from $\opcl{1/n, 2} = A_n$.
  This shows that $0 \notin \clA_n$ for every $n \in \pints$ by Theorem~17.5a, from which it follows that $0 \notin \bigcup \clA_n$.
  Hence $\closure{\bigcup A_n}$ is not a subset of $\bigcup \clA_n$ and thus $\closure{\bigcup A_n} \neq \bigcup \clA_n$.
}

\exercise{7}{
  Criticize the following ``proof'' that $\closure{\bigcup A_\a} \ss \bigcup \clA_\a$: if $\braces{A_\a}$ is a collection of sets in $X$ and if $x \in \closure{\bigcup A_\a}$, then every neighborhood $U$ of $x$ intersects $\bigcup A_\a$.
  Thus $U$ must intersect some $A_\a$, so that $x$ must belong to the closure of some $A_\a$.
  Therefore, $x \in \bigcup \clA_\a$.
}
\sol{
  \dwhitman

  The problem with this ``proof'' is that, just because every neighborhood $U$ intersects \emph{some} $A_\a$, it does not mean that \emph{every} $U$ intersects a single $A_\a$, which is what is required for $x$ to be in $\clA_\a$.
  This is illustrated in the counterexample above at the end of Exercise~17.6c.
  There, every neighborhood of $0$ clearly intersects \emph{some} set $A_n = \opcl{1/n, 2}$, but, for any given $n \in \pints$, not every neighborhood of $0$ intersects $A_n$, for example the neighborhood $(-1, 1/n)$ does not.
}

\exercise{8}{
  Let $A$, $B$, and $A_\a$ denote subsets of a space $X$.
  Determine whether the following equations hold; if an equality fails, determine whether one of the inclusions $\sps$ or $\ss$ holds.
  \eparts{
  \item $\closure{A \cap B} = \clA \cap \clB$.
  \item $\closure{\bigcap A_\a} = \bigcap \clA_\a$.
  \item $\closure{A - B} = \clA - \clB$.
  }
}
\sol{
  \dwhitman

  (a) We claim that $\closure{A \cap B} \ss \clA \cap \clB$ but equality is not always true.
  \qproof{
    Consider any $x \in \closure{A \cap B}$ and any open set $U$ containing $x$.
    Then by, Theorem~17.5a, $U$ intersects $A \cap B$, from which it immediately follows that $U$ intersects both $A$ and $B$.
    However, since $U$ was an arbitrary neighborhood of $x$, it follows from Theorem~17.5a again that $x$ is in both $\clA$ and $\clB$.
    Hence $x \in \clA \cap \clB$, which shows that $\closure{A \cap B} \ss \clA \cap \clB$ since $x$ was arbitrary.

    Now consider the standard topology on $\reals$ with $A = \clop{-1,0}$ and $B = \opcl{0,1}$.
    As these are clearly disjoint, we have that $A \cap B = \es$ so that $\closure{A \cap B} = \es$ also.
    However, since we also clearly have that $\clA = [-1,0]$ and $\clB = [0,1]$, it follows that $\clA \cap \clB = \braces{0}$.
    Thus clearly $\closure{A \cap B} = \es \neq \braces{0} = \clA - \clB$ as desired.
  }

  (b) We again claim that $\closure{\bigcap A_\a} \ss \bigcap \clA_\a$ but that equality is not generally true.
  \qproof{
    Consider any $x \in \closure{\bigcap A_\a}$ and any open set $U$ of $x$.
    Then, by Theorem~17.5a, $U$ intersects $\bigcap A_\a$ so that, for any particular $A_\b$, $U$ intersects $A_\b$.
    This shows that $x \in \clA_\b$ by Theorem~17.5a so that $x \in \clA_\a$ for every $\a$ since $\b$ was arbitrary.
    Hence $x \in \bigcap \clA_\a$, which shows that $\closure{\bigcap A_\a} \ss \bigcap \clA_\a$ since $x$ was arbitrary.
    
    As in part (a), equality fails if we have $A_1 = \clop{-1,0}$ and $A_2 = \opcl{0,1}$ in the standard topology on $\reals$.
    By the same argument as in part (a) it follows that $\closure{\bigcap_{n=1}^2 A_n} = \es \neq \braces{0} = \bigcap_{n=1}^2 \clA_n$.
  }

  (c) Here we claim that $\closure{A-B} \sps \clA - \clB$ but that the converse does not always hold.
  \qproof{
    Consider any $x \in \clA - \clB$ and any open set $U$ containing $x$.
    Then $x \in \clA$ so that every open set containing $x$ intersects $A$ by Theorem~17.5a.
    Also $x \notin \clB$ so that there is an open set $V$ containing $x$ that does not intersect $B$, also by Theorem~17.5a.
    Let $W = U \cap V$ so that $W$ contains $x$ since both $x \in U$ and $x \in V$.
    Now, since $W$ is also an open set containing $x$, $W$ intersects $A$ so that there is a $y \in W$ where also $y \in A$.
    It also cannot be that $y \in B$ since we have $y \in W \ss V$ so that then $V$ would intersect $B$.
    Therefore $y \in A - B$.
    Also we have $y \in W \ss U$ so that also $y \in U$.
    Hence $U$ intersects $A - B$, which shows that $x \in \closure{A-B}$ by Theorem~17.5a since $U$ was an arbitrary neighborhood of $x$.
    Therefore $\closure{A-B} \sps \clA - \clB$ as desired since $x$ was arbitrary.

    As a counterexample to equality, consider the standard topology on $\reals$ with $A = [0,2]$ and $B = \opcl{1,3}$.
    Then clearly $\clA = A = [0,2]$ and $\clB = [1,3]$, from which it is easily shown that $\clA - \clB = \clop{0,1}$.
    But we also have $A - B = [0,1]$ so that obviously $\closure{A-B} = [0,1]$ as well.
    Therefore $\closure{A-B} = [0,1] \neq \clop{0,1} = \clA - \clB$ as desired.
  }
}

\exercise{9}{
  Let $A \ss X$ and $B \ss Y$.
  Show that in the space $X \times Y$,
  \gath{
    \closure{A \times B} = \clA \times \clB \,.
  }
}
\sol{
  \dwhitman

  \qproof{
    $(\ss)$ Consider $(x,y) \in \closure{A \times B}$.
    Also suppose that $U$ and $V$ are any open sets in $X$ and $Y$, respectively, that contain $x$ and $y$, respectively.
    Then $U \times V$ is a basis element of the product topology on $X \times Y$, by definition, that contains $(x,y)$.
    It then follows from Theorem~17.5b that $U \times V$ intersects $A \times B$ and hence there is a point $(w,z) \in U \times V$ where also $(w,z) \in A \times B$.
    Then $w \in U$ and $w \in A$ so that $U$ intersects $A$, and hence $x \in \clA$ by Theorem~17.5a since $U$ was an arbitrary neighborhood of $x$.
    An analogous argument shows that $y \in \clB$.
    Therefore $(x,y) \in \clA \times \clB$ so that $\closure{A \times B} \ss \clA \times \clB$ since $x$ was arbitrary.

    $(\sps)$ Now suppose that $(x,y)$ is any point in $\clA \times \clB$ so that $x \in \clA$ and $y \in \clB$.
    Suppose also that $U \times V$ is any basis element of $X \times Y$ that contains $(x,y)$ so that by definition $U$ and $V$ are open in $X$ and $Y$, respectively.
    Since $x \in \clA$ and $U$ is an open set where $x \in U$, it follows from Theorem~17.5a that $U$ intersects $A$.
    Thus there is $w \in U$ where $w \in A$ as well.
    An analogous argument shows that $V$ intersects $B$ so that there is a $z \in V$ where also $z \in B$.
    We therefore have that $(w,z) \in U \times V$ and $(w,z) \in A \times B$ so that $U \times V$ intersects $A \times B$.
    Since $U \times V$ was an arbitrary basis element containing $(x,y)$, it follows from Theorem~17.5b that $(x,y) \in \closure{A \times B}$.
    This shows that $\clA \times \clB \ss \closure{A \times B}$ since the point $(x,y)$ was arbitrary.
  }
}

\exercise{10}{
  Show that every order topology is Hausdorff.
}
\sol{
  \dwhitman

  \qproof{
    Suppose that $X$ is an ordered set with the order topology.
    Consider a pair of distinct points $x_1$ and $x_2$ in $X$.
    Since $X$ is an order, $x_1$ and $x_2$ must be comparable since they are distinct, so we can assume that $x_1 < x_2$ without loss of generality.

    Case: $x_2$ is the immediate successor of $x_1$.
    Then, if $X$ has a smallest element $a$ then clearly the set $U_1 = \clop{a, x_2}$ is a neighborhood (because it is a basis element) of $x_1$.
    If $X$ has no smallest element then there is an $a < x_1$ so that $U_1 = (a, x_2)$ is a neighborhood of $x_1$.
    Similarly $U_2 = \opcl{x_1, b}$ or $U_2 = (x_1, b)$ is a neighborhood of $x_2$, where $b$ is either the largest element of $X$ or $x_2 < b$, respectively.
    Either way, for any $y \in U_1$ we have that $y < x_2$ so that $y \leq x_1$ since $x_2$ is the immediate successor of $x_1$.
    Hence it is not true that $y > x_1$ so that $y \notin U_2$.
    This shows that $U_1$ and $U_2$ are disjoint.

    Case: $x_2$ is not the immediate successor of $x_1$.
    Then there is an $x \in X$ where $x_1 < x < x_2$.
    So let $U_1 = \clop{a,x}$ (or $U_1 = (a, x)$) for the smallest element $a$ of $X$ (or some $a < x_1$).
    Similarly let $U_2 = \opcl{x,b}$ (or $U_2 = (x,b)$) for the largest element $b$ of $X$ (or some $x_2 < b$).
    Either way $U_1$ and $U_2$ are neighborhoods of $x_1$ and $x_2$, respectively.
    If $y \in U_1$ then $y < x$ so that clearly it is not true that $y > x$ so that $x \notin U_2$.
    Hence again $U_1$ and $U_2$ are disjoint.

    Thus in either case we have shown that $X$ is a Hausdorff space as desired since $x_1$ and $x_2$ were an arbitrary pair.
  }
}

\exercise{11}{
  Show that the product of two Hausdorff spaces is Hausdorff.
}
\sol{
  \dwhitman

  \qproof{
    Suppose that $X$ and $Y$ are Hausdorff spaces and consider two distinct points $(x_1,y_1)$ and $(x_2,y_2)$ in $X \times Y$.
    Since these points are distinct, it has to be that $x_1 \neq x_2$ or $y_1 \neq y_2$.
    In the first case $x_1$ and $x_2$ are distinct points of $X$ so that there are disjoint neighborhoods $U_1$ and $U_2$ of $x_1$ and $x_2$, respectively.
    This of course follows from the fact that $X$ is a Hausdorff space.
    Then we have that $U_1 \times Y$ and $U_2 \times Y$ are both basis elements, and therefore open sets, in the product space $X \times Y$ since $Y$ itself is obviously an open set of $Y$.
    Clearly also $(x_1,y_1) \in U_1 \times Y$ and $(x_2,y_2) \in U_2 \times Y$ so that $U_1 \times Y$ is a neighborhood of $(x_1,y_1)$ and $U_2 \times Y$ is a neighborhood of $(x_2,y_2)$.

    Then, for any $(x,y) \in U_1 \times Y$ we have that $x \in U_1$ so that $x \notin U_2$ since they are disjoint.
    Then it has to be that $(x,y) \notin U_2 \times Y$.
    This suffices to show that $U_1 \times Y$ and $U_2 \times Y$ are disjoint since $(x,y)$ was arbitrary.
    Thus $X \times Y$ is a Hausdorff space since the points $(x_1,y_1)$ and $(x_2,y_2)$ were arbitrary.
    An analogous argument in the case in which $y_1 \neq y_2$ shows the same result.
  }
}

\exercise{12}{
  Show that a subspace of a Hausdorff space is Hausdorff.
}
\sol{
  \dwhitman

  \qproof{
    Suppose that $X$ is a Hausdorff space and that $Y$ is a subset of $X$.
    Consider any two distinct points $y_1$ and $y_2$ in $Y$ so that of course also $y_1,y_2 \in X$.
    Then there are neighborhoods $U_1$ and $U_2$ of $y_1$ and $y_2$, respectively, that are disjoint since $X$ is Hausdorff.
    Since $U_1$ is open in $X$, we have that $V_1 = U_1 \cap Y$ is open in $Y$ by the definition of a subspace topology.
    Clearly also $V_1$ contains $y_1$ since $y_1 \in U_1$ and $y_1 \in Y$.
    Similarly $V_2 = U_2 \cap Y$ is an open set of $Y$ that contains $y_2$.
    Then, for any $x \in V_1$ clearly $x \in U_1$ so that $x \notin U_2$ since $U_1$ and $U_2$ are disjoint.
    Then $x \notin U_2 \cap Y = V_2$.
    Since $x$ was arbitrary, this shows that $V_1$ and $V_2$ are disjoint, which then shows that $Y$ is a Hausdorff space as desired.
  }
}
