\setcounter{subsection}{6-1}
\subsection{Finite Sets}

\exercise{1}{
  \eparts{
  \item Make a list of all the injective maps
    \gath{
      f : \braces{1,2,3} \to \braces{1,2,3,4} \,.
    }
    Show that none is bijective.
    (This constitutes a \emph{direct} proof that a set $A$ of cardinality three does not have cardinality four.)
  \item How many injective maps
    \gath{
      f : \intsfin{8} \to \intsfin{10}
    }
    are there?
    (You can see why one would not wish to try to prove \emph{directly} that there is no bijective correspondence between these sets.)
  }
}
\sol{
  \dwhitman

  \begin{lem}\label{lem:finset:inj}
    The number of injective mappings (i.e. the cardinality of the set of injective functions) from $\intsfin{m}$ to $\intsfin{n}$, where $m \leq n$, is equal to the number of $m$-permutations of $n$, which is
    \gath{
      \frac{n!}{(n-m)!} \,.
    }
  \end{lem}
  \qproof{
    We fix $n$ and show this for all $m \leq n$ by induction.
    First, for $m=1$, the domain of the mappings is simply $\braces{1}$ so that we need only choose a single element to which to map $1$.
    Since there are $n$ elements to choose from (since the range is $\intsfin{n}$) there are clearly
    \gath{
      n = \frac{n!}{(n-1)!} = \frac{n!}{(n-m)!}
    }
    mappings, all of which are trivially injective.

    Now suppose that $m < n$ and that there are $n! / (n-m)!$ injective mappings from $\intsfin{m}$ to $\intsfin{n}$.
    Consider any such mapping $(f_1, \ldots, f_m)$.
    Since this mapping is injective, each $f_i$ is unique so that it uses $m$ of the $n$ available numbers in $\intsfin{n}$.
    Thus there are $n-m$ numbers to choose from to which to set $f_{m+1}$ so that the mapping $(f_1, \ldots, f_{m+1})$ is still injective.
    Hence for each injective mapping $(f_1, \ldots, f_m)$ there are $n-m$ injective mappings from $\intsfin{m+1}$ to $\intsfin{n}$.
    Since there are $n!/(n-m)!$ such mappings by the induction hypothesis, the total number of mappings from $\intsfin{m+1}$ to $\intsfin{n}$ will be
    \gath{
      \frac{n!}{(n-m)!}(n-m) = \frac{n!}{(n-m-1)!} = \frac{n!}{[n-(m+1)]!} \,,
    }
    which completes the induction.
  }

  \mainprob
  
  (a) Here we have $n=4$ and $m=3$ in Lemma~\ref{lem:finset:inj} so that we expect $4!/(4-3)! = 4!/1! = 4! = 24$ injective mappings.
  Since the domain of each $f$ is a section of the positive integers, these maps can be written simply as 3-tuples.
  They are enumerated below:
  \begin{multicols}{4}
    \begin{enumerate}[itemsep=0cm]
    \item $(1, 2, 3)$
    \item $(1, 2, 4)$
    \item $(1, 3, 2)$
    \item $(1, 3, 4)$
    \item $(1, 4, 2)$
    \item $(1, 4, 3)$
    \item $(2, 1, 3)$
    \item $(2, 1, 4)$
    \item $(2, 3, 1)$
    \item $(2, 3, 4)$
    \item $(2, 4, 1)$
    \item $(2, 4, 3)$
    \item $(3, 1, 2)$
    \item $(3, 1, 4)$
    \item $(3, 2, 1)$
    \item $(3, 2, 4)$
    \item $(3, 4, 1)$
    \item $(3, 4, 2)$
    \item $(4, 1, 2)$
    \item $(4, 1, 3)$
    \item $(4, 2, 1)$
    \item $(4, 2, 3)$
    \item $(4, 3, 1)$
    \item $(4, 3, 2)$
    \end{enumerate}
  \end{multicols}
  Note that they are all injective since no number is used more than once in each tuple.
  Also none are surjective since it is easily verified that there is always an element of $\braces{1,2,3,4}$ that is not in each tuple.
  Thus none are a bijection since they are not surjective.

  (b) Here we have $n = 10$ and $m = 8$ in Lemma~\ref{lem:finset:inj} so that there are $10! / (10-8)! = 10! / 2! = 1814400$ injective mappings.
  That is nearly two million!
  Certainly a direct proof would be unfeasible by hand, but could be done by computer fairly easily.
}

\exercise{2}{
  Show that if $B$ is not finite and $B \ss A$, then $A$ is not finite.
}
\sol{
  \dwhitman

  \qproof{
    Suppose that $B$ is not finite and $B \ss A$ but that $A$ \emph{is} finite.
    Since $B \ss A$, either $B = A$ or $B$ is a proper subset of $A$.
    In the former case we clearly have a contradiction since $B$ would be finite since $A$ is and $B = A$.
    In the latter case we have that there is a bijection from $A$ to $\intsfin{n}$ for some $n \in \pints$ by definition since $A$ is finite.
    Then, since $B$ is a proper subset of $A$, it follows from Theorem~6.2 that there is a bijection from $B$ to $\intsfin{m}$ for some $m < n$.
    However, then clearly $B$ is finite by definition, which is also a contradiction since we know $B$ is not finite.
    Hence in either case there is a contradiction so that $A$ must not be finite.
  }
}

\exercise{3}{
  Let $X$ be the two-element set $\braces{0,1}$.
  Find a bijective correspondence between $X^\w$ and a proper subset of itself.
}
\sol{
  \dwhitman

  \qproof{
    Let $Y = \braces{\vx \in X^\w \where x_1 = 0}$, which is clearly a proper subset of $X^\w$ since, for example, $(1,1,\ldots)$ is in $X^\w$ but not in $Y$.
    We construct a bijective function $f$ from $X^\w$ to $Y$.
    So consider any $\vx \in X^\w$ and define
    \gath{
      y_i = \begin{cases}
        0 & i = 1 \\
        x_{i-1} & i \neq 1
      \end{cases}
    }
    for $i \in \pints$, noting that when $i \neq 1$ we have $i > 1$ so that $i-1 \geq 1$ so that $y_i = x_{i-1}$ is defined.
    Now define $f(\vx) = \vy = (y_1, y_2, \ldots)$ so that clearly $f$ is a function from $X^\w$ to $Y$, since $y_1 = 0$ for any input $\vx$.

    To show that $f$ is injective, consider $\vx$ and $\vx'$ in $X^\w$ where $\vx \neq \vx'$, and let $\vy = f(\vx)$ and $\vy' = f(\vx')$.
    Now, since $\vx \neq \vx'$, there is an $i \in \pints$ where $x_i \neq x_i'$.
    Since $i > 0$ (since $i \in \pints$) it follows that $i+1 > 1$ so that $i+1 \neq 1$.
    We then have by the definition of $f$ that $y_{i+1} = x_{(i+1)-1} = x_i \neq x_i' = x_{(i+1)-1}' = y_{i+1}'$ so that clearly $f(\vx) = \vy \neq \vy' = f(\vx')$.
    Since $\vx$ and $\vx'$ were arbitrary, this shows that $f$ is indeed injective.

    Now consider any $\vy \in Y$ so that $y_1 = 0$.
    Define $x_i = y_{i+1}$ for any $i \in \pints$ and let $\vx = (x_1, x_2, \ldots)$.
    Then $\vx \in X^\w$ since clearly each $x_i = y_{i+1} \in X$.
    Now let $\vy' = f(\vx)$ and consider any $i \in \pints$.
    If $i = 1$ then clearly $y_i' = y_1' = 0 = y_1 = y_i$ ($y_1' = 0$ since the range of $f$ is $Y$).
    If $i \neq 1$ then $y_i' = x_{i-1}' = y_{(i-1)+1} = y_i$.
    Hence $y_i' = y_i$ in both cases so that $f(\vx) = \vy' = \vy$ since $i$ was arbitrary.
    This shows that $f$ is surjective since $\vy$ was arbitrary.

    Therefore $f$ is bijective as desired.
  }
}

\exercise{4}{
  Let $A$ be a nonempty finite simply ordered set.
  \eparts{
  \item Show that $A$ has a largest element.
    [Hint: Proceed by induction on the cardinality of $A$.]
  \item Show that $A$ has the order type of a section of positive integers.
  }
}
\sol{
  \dwhitman

  (a)
  \qproof{
    We show by induction that, for all $n \in \pints$, any simply ordered set with cardinality $n$ has a largest element.
    This of course shows the result since, by definition, $A \neq \es$ has cardinality $n$ for some $n \in \pints$ when $A$ is finite.

    First, suppose that $A$ is simply ordered and has cardinality 1 so that clearly $A = \braces{a}$ for some element $a$.
    It is also clear that $a$ is trivially the largest element of $A$ since it is the only element.

    Now suppose that any simply ordered set with cardinality $n$ has a largest element.
    Suppose that $A$ is simply ordered by $\prec$ and has cardinality $n+1$.
    Then there is a bijection $f$ from $A$ to $\intsfin{n+1}$, noting that obviously $\inv{f}$ is also a bijection.
    Clearly $A$ is nonempty (since the cardinality of $A$ is $n+1 > n > 0$) so that there is an $a \in A$.
    Let $A' = A-\braces{a}$ so that $A'$ has cardinality $n$ by Lema~6.1.
    Note also that clearly $A'$ is simply ordered by $\prec$ as well (technically we must restrict $\prec$ to elements of $A'$ so that it is really orderd by $\prec \cap (A' \times A')$).
    It then follows that $A'$ has a largest element $b$ by the induction hypothesis.
    Since $a$ and $b$ must be comparable in $\prec$ by the definition of a simple order we have the following:

    Case: $a = b$.
    This is not possible since $b \in A'$ but clearly $a \notin A - \braces{a} = A'$.
    
    Case: $a \prec b$.
    We claim that $b$ is the largest element of $A$.
    To see this, consider any $x \in A$ so that either $x = a$ or $x \in A'$.
    In the former case clearly $x = a \prece b$, and in the latter $x \prece b$ since $b$ is the largest element of $A'$.
    This shows that $b$ is the largest element of $A$ since $x$ was arbitrary.

    Case: $b \prec a$.
    We claim that $a$ is the largest element of $A$.
    So consider any $x \in A$ so that $x = a$ or $x \in A'$.
    In the first case obviously $x \prece x = a$, and in the second $x \prece b \prece a$ since $b$ is the largest element of $A'$.
    This shows that $a$ is the largest element of $A$ since $x$ was arbitrary.

    Thus in all cases we have shown that $A$ has a largest element, which completes the induction.
  }

  (b)
  \qproof{
    We again show this by induction on the (finite) cardinality of the set.
    First, if $A$ is a simply ordered set with cardinality 1 then clearly $A = \braces{a}$ for some $a$, which is clearly trivially the same order type as the section $\braces{1}$.

    Now suppose that all simply ordered sets of cardinality $n$ have the order type of a section of positive integers.
    Consider then a set $A$ simply ordered by $\prec$ that has cardinality $n+1$.
    Clearly $A \neq \es$ so that there is an $a \in A$.
    It then follows from Lemma~6.1 that $A' = A - \braces{a}$ has cardinality $n$ so that there is a bijection $f$ from $A'$ to $\intsfin{n}$.
    Also, since $A'$ is also simply ordered by $\prec$ (with the appropriate restriction), it follows from the induction hypothesis that $A'$ has the order type of a section of integers.
    Thus there is an order-preserving bijection $g$ from $A'$ to $\intsfin{m}$ for some $m \in \pints$.
    Since the cardinality of a finite set is unique (by Corollary~6.5), it has to be that $m = n$ (since by definition $A'$ has cardinality $m$ and $n$ since there are bijections from $A'$ to both $\intsfin{m}$ and $\intsfin{n}$) so that $g$ is an order-preserving bijection from $A'$ to $\intsfin{n}$.

    Now, let $B = \braces{x \in A \where x \prec a}$, noting that it could very well be that $B = \es$.
    If $B \neq \es$ then, since clearly $a \notin B$ but $a \in A$, it follows that $B$ is a proper subset of $A$ so that $B$ is nonempty and finite by Corollary~6.6.
    Since it is also clearly simply ordered by $\prec$ (with the appropriate restriction), it follows from part (a) that $B$ has a largest element $b$.
    So set $k = g(b)+1$.
    On the other hand, if $B = \es$ then simply set $k=1$.

    Now define
    \gath{
      h(x) = \begin{cases}
        g(x) & x \prec a \\
        k & x = a \\
        g(x)+1 & a \prec x
      \end{cases}
    }
    for any $x \in A$.
    We first claim that $h$ is a function from $A$ to $\intsfin{n+1}$.
    Consider any $x \in A$.
    If $x \prec a$ then clearly $h(x) = g(x) \in \intsfin{n}$ since the range of $g$ is $\intsfin{n}$.
    Then clearly $h(x) \in \intsfin{n+1}$ since $\intsfin{n} \ss \intsfin{n+1}$.
    If $x = a$ then either $h(x) = k = 1$ or $h(x) = k = g(b)+1$ for a $b \in B$.
    Clearly $h(x) = 1 \in \intsfin{n+1}$ in the former case and, in the latter case, $1 \leq g(b) \leq n$ since the range of $g$ is $\intsfin{n}$, and hence $1 \leq 2 \leq h(x)=g(b)+1 \leq n+1$ so that clearly $h(x) \in \intsfin{n+1}$.
    In the final case in which $a \prec x$ we again have that $1 \leq h(x) = g(x)+1 \leq n+1$ so that $h(x) \in \intsfin{n+1}$.
    This shows that $h$ is indeed a function from $A$ to $\intsfin{n+1}$ since $x$ was arbitrary.

    Next we show that $h$ preserves order.
    So consider $x$ and $x'$ in $A$ where $x \prec x'$.

    Case: $x \prec a$.
    Then $h(x) = g(x)$ and clearly $x \in B$ by definition so that $B \neq \es$, from which it follows that $k = g(b)+1$ where $b$ is the largest element of $B$.
    Hence $x \prece b$.
    Now, if also $x' \prec a$ then $h(x') = g(x')$ so that $h(x) = g(x) < g(x') = h(x')$ since $g$ preserves order, noting that both $x,x' \in A'$ so they are in the domain of $g$, since they are not equal to $a$.
    If $x' = a$ then $h(x') = k = g(b)+1$.
    Also since $x \prece b$ we have that $g(x) \leq g(b)$ since $g$ preserves order.
    Hence we have $h(x) = g(x) \leq g(b) < g(b)+1 = k = h(x')$.
    Lastly, if $a \prec x'$ then $h(x') = g(x')+1$ so that $h(x) = g(x) < g(x') < g(x')+1 = h(x')$ since $g$ preserves order and $x \prec x'$ (noting again that both $x$ and $x'$ are in the domain of $g$).

    Case: $x = a$.
    Then of course $h(x) = k$.
    Also $a = x \prec x'$ so that $h(x') = g(x')+1$.
    In the case when $B = \es$ we have $k=1$ so that $h(x) = k = 1 < g(x')+1 = h(x')$ since obviously $0 < g(x')$.
    In the other case in which $B \neq \es$ then $h(x) = k = g(b)+1$ for the largest element $b$ of $B$.
    Since $b \in B$, we have $b \prec a = x$ by definition so that $b \prec x \prec x'$.
    Hence $g(b) < g(x')$ since $g$ preserves order.
    Then of course $h(x) = k = g(b)+1 < g(x')+1 = h(x')$.

    Case: $a \prec x$.
    Then we have $h(x) = g(x)+1$.
    We also have that $a \prec x \prec x'$ so that also $h(x') = g(x')+1$.
    Noting that we clearly have $x,x' \in A'$, we have that $g(x) < g(x')$ since $g$ preserves order.
    Hence $h(x) = g(x)+1 < g(x')+1 = h(x')$.

    Therefore in all cases and sub-cases we have shown that $h(x) < h(x')$, which shows that $h$ preserves order since $x$ and $x'$ were arbitrary.
    Note that this also shows that $h$ is injective since, for any $x,x' \in A$ where $x \neq x'$, we can assume without loss of generality that $x \prec x'$ (since it must be that $x \prec x'$ or $x' \prec x$) so that $h(x) < h(x')$, and hence $h(x) \neq h(x')$.

    Lastly, we show that $h$ is surjective.
    So consider any $k \in \intsfin{n+1}$.
    If $k \leq n$ then $k \in \intsfin{n}$, which is range of $g$ so that there is an $x \in A'$ where $g(x) = k$ since $g$ is surjective (since it is bijective).
    We then have that $x \neq a$ since $x \in A'$.
    Thus $x \prec a$ or $a \prec x$.
    In the former case we have $h(x) = g(x) = k$, and in the latter case $h(x) = g(x)+1$.
  }
}
