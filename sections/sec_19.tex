\setcounter{subsection}{19-1}
\subsection{The Product Topology}

\exercise{1}{
  Prove Theorem~19.2
}
\sol{
  \dwhitman

  Let $\cC$ be the collection of sets that are alleged to be a basis for the box or product topologies in Theorem~19.2.
  \qproof{
    We show that $\cC$ is a basis of the box or product topology using Lemma~13.2.
    First, it is easy to see that $\cC$ is a collection of open sets.
    Consider any $B \in \cC$ so that $B = \prod  B_\a$ where each $B_\a \in \cB_\a$ (for a finitely many $\a \in J$ and $B_\a = X_\a$ for the rest in the product topology).
    Since each $B_\a$ is a basis element of $X_\a$ (or $X_\a$ itself), they are open so that $B$ is a basis element of the box or product topology by definition and therefore open.
    Note that the basis for the product topology is given directly by Theorem~19.1.
    
    Now suppose that $U$ is an any open set of the box topology and consider any $x \in U$.
    Then it follows that there is a basis element $\prod_{\a \in J} U_\a$  of the box or product topology containing $x$ where $\prod_{\a \in J} U_\a \ss U$.
    Thus each $U_\a$ is an open set of $X_\a$ (or $U_\a = X_\a$ for all but finitely many $\a \in J$ for the product topology).
    Also $x \in \prod_{\a \in J} U_\a$ so that $x = (x_\a)_{\a \in J}$ where each $x_\a \in U_\a$.
    It then follows that there is basis element $B_\a \in \cB_\a$ of $X_\a$ containing $x_\a$ where $B_\a \ss U_\a$ (for $U_\a = X_\a$ we simply set $B_\a = X_\a$ as well).
    
    Then clearly $x \in \prod_{\a \in J} B_\a$ and $\prod_{\a \in J} B_\a \in \cC$.
    Consider also any $y \in \prod_{\a \in J} B_\a$ so that $y = (y_\a)_{\a \in J}$ where each $y_\a \in B_\a$.
    Then also each $y_\a \in U_\a$ since $B_\a \ss U_\a$.
    This suffices to show that $y \in \prod_{\a \in J} U_\a \ss U$.
    Since $y$ was arbitrary this shows that $\prod_{\a \in J} B_\a \ss U$.
    Therefore $\cC$ is a basis of the box topology by Lemma~13.2.
  }
}

\exercise{2}{
  Prove Theorem~19.3.
}
\sol{
  \dwhitman

  \qproof{
    The basis of the box or product topologies on $\prod A_\a$ is the collection of sets $\prod V_\a$, where each $V_\a$ is open in $A_\a$ and, in the case of the product topology, $V_\a = A_\a$ for all but finitely many $\a \in J$ (by Theorem~19.1).
    Denote this basis collection by $\cC$.
    By Lemma~16.1, the collection
    \gath{
      \cB_A = \braces{B \cap \prod A_\a \where B \in \cB}
    }
    is a basis of the subspace topology on $\prod A_\a$, where $\cB$ is the basis of $\prod X_\a$.
    To prove that $\prod A_\a$ is a subspace of $\prod X_\a$, it therefore suffices to show that $\cC = \cB_A$.

    $(\ss)$ First consider any element $B \in \cC$ so that $B = \prod V_\a$ for open sets $V_\a$ in $A_\a$ (and $V_\a = A_\a$ for all but finite many $\a \in J$ for the product topology).
    For each $\a \in J$, we then have that $V_\a = U_\a \cap A_\a$ for some open set $U_\a$ in $X_\a$ since $A_\a$ is a subspace of $X_\a$.
    Note that this is true even for those $\a$ where $V_\a = A_\a$ in the product topology since then $V_\a = A_\a = X_\a \cap A_\a$.
    In fact, for these $\a$ we need to choose $U_\a = X_\a$ as will become apparent.
    We then have the following:
    \ali{
      x \in B &\bic x \in \prod V_\a \\
      &\bic \forall \a \in J (x_\a \in V_\a) \\
      &\bic \forall \a \in J (x_\a \in U_\a \cap A_\a) \\
      &\bic \forall \a \in J (x_\a \in U_\a \land x_\a \in A_\a) \\
      &\bic \forall \a \in J (x_\a \in U_\a) \land \forall \a \in J (x_\a \in A_\a) \\
      &\bic x \in \prod U_\a \land x \in \prod A_\a \\
      &\bic x \in \parens{\prod U_\a} \cap \parens{\prod A_\a} \,,
    }
    Since $U_\a = X_\a$ for all but a finitely many $\a \in J$ for the product topology, we have that $\prod U_\a$ is a basis element of $\prod X_\a$, i.e. $\prod U_\a \in \cB$.
    This shows that $B \in \cB_A$ so that $\cC \ss \cB_A$ since $B$ was arbitrary.

    $(\sps)$ Now suppose that $B \in \cB_A$ so that $B = B_X \cap \prod A_\a$ for some basis element $B_X \in \cB$ of $\prod X_\a$.
    We then have that $B_X = \prod U_\a$ where each $U_\a$ is an open set of $X_\a$ (and $U_\a = X_\a$ for all but finitely many $\a \in J$ for the product topology).
    Then let $V_\a = U_\a \cap A_\a$ for each $\a \in J$, noting that $V_\a = X_\a \cap A_\a = A_\a$ when $U_\a = X_\a$.
    Following the above chain of logical equivalences in reverse order then shows that $B = \prod V_\a$ so that $B \in \cC$ since clearly each $V_\a$ is open in the subspace topology $A_\a$.
    Hence $\cC \sps \cB_A$ since $B$ was arbitrary.
  }
}

\exercise{3}{
  Prove Theorem~19.4.
}
\sol{
  \dwhitman

  \qproof{
    Suppose that $x$ and $y$ are distinct points of $\prod X_\a$.
    Then $x = (x_a)$ and $y = (y_\a)$ where each $x_\a, y_\a \in X_\a$, and there must be a $\b$ where $x_\b \neq y_\b$ since $x \neq y$.
    Thus $x_\b$ and $y_\b$ are distinct points of $X_\b$, so that there are neighborhoods $W_x$ and $W_y$ of $x_\b$ and $y_\b$, respectively, that are disjoint since $X_\b$ is a Hausdorff space.
    So define the sets
    \ali{
      U_\a &= \begin{cases}
        W_x & \a = \b \\
        X_\a & \a \neq \b
      \end{cases}
      &
      V_\a &= \begin{cases}
        W_y & \a = \b \\
        X_\a & \a \neq \b
      \end{cases}
    }
    so that clearly $x \in \prod U_\a$ and $y \in \prod V_\a$.
    Then since each $U_\a$ and $V_\a$ are open, we have that $\prod U_\a$ and $\prod V_\a$ are both basis elements of $\prod X_\a$ and therefore open.
    Note that this is true for both the box and product topologies since, in the case of the latter, $U_\a$ and $V_\a$ are not all of $X_\a$ for only one $\a$, namely $\a = \b$.
    Thus $\prod U_\a$ is a neighborhood of $x$ and $\prod V_\a$ is a neighborhood of $y$ in $\prod X_\a$.

    We also assert that $\prod U_\a$ and $\prod V_\a$ are disjoint, which of course completes the proof that $\prod X_\a$ is Hausdorff.
    To see this, suppose to the contrary that there is a $z$ in both $\prod U_\a$ and $\prod V_\a$.
    Then $z = (z_\a)$ and in particular we would have that $z_\b \in U_\b = W_x$ and $z_\b \in V_\b = W_y$.
    But then $z_\b \in W_x \cap W_y$, which contradicts the fact that $W_x$ and $W_y$ are disjoint!
    So it must be that in fact $\prod U_\a$ and $\prod V_\a$ are disjoint.
  }
}

\def\doms{\cpfin{X}{n}}
\def\rans{(\cpfin{X}{n-1}) \times X_n}
\exercise{4}{
  Show that $\rans$ is homeomorphic to $\doms$.
}
\sol{
  \dwhitman

  \qproof{
    First we note that since we are dealing with finite products, the box and product topologies are the same; we shall find it most convenient to use the box topology definition.
    Also, as there are no intervals involved here, we use the traditional tuple notation using parentheses.
    So define $f: \doms  \to \rans$ by
    \gath{
      f(x_1, \ldots, x_{n-1}) = ((x_1, \ldots, x_{n-1}), x_n) \,.
    }
    It is obvious that this is a bijection, and it is trivial to prove.
    Also obvious and trivial to prove based on the definition of $f$ is that $f(\cpfin{A}{n}) = (\cpfin{A}{n-1}) \times A_n$ when each $A_k \ss X_k$.
    
    First we show that $f$ is continuous by showing that the inverse image of every basis element in $\rans$ is open in $\doms$.
    So consider any basis element $C$ of $\rans$ and let $U = \inv{f}(C)$ so that of course $f(U) = C$ and $U \ss \doms$.
    We then have that $C = V' \times V_n$ where $V'$ is open in $\cpfin{X}{n-1}$ and $V_n$ is open in $X_n$ by the definition of the box/product topology.
    Now consider any $x \in U$ so that $x = (x_k)_{k=1}^n$ and we have that $f(x) = ((x_1, \ldots, x_{n-1}), x_n) \in f(U) = C$.
    Hence $x' = (x_1, \ldots, x_{n-1}) \in V'$ and $x_n \in V_n$.
    Since $V'$ is open in $\cpfin{X}{n-1}$ there is a basis element $C'$ containing $x'$ that is a subset of $V'$.
    By the definition of the box topology, we then have that $C' = \cpfin{V}{n-1}$ where each $V_k$ is open in $X_k$.

    We then have that $B = \cpfin{V}{n}$ is a basis element of $\doms$ and also clearly $B$ contains $x$ since $(x_1, \ldots, x_{n-1}) = x' \in C' = \cpfin{V}{n-1}$ and $x_n \in V_n$.
    Now suppose that $y = (y_k)_{k=1}^n \in B$ so that each $y_k \in V_k$.
    Then we have that $y' = (y_1, \ldots, y_{n-1}) \in C'$ so that also $y' \in V'$ since $C' \ss V'$.
    Since also of course $y_n \in V_n$, we have that $(y', y_n) \in V' \times V_n = C$.
    Also clearly $f(y) = (y', y_n) \in C = f(U)$ so that $y \in U$.
    Since $y$ was arbitrary this shows that $B \ss U$, which suffices to show that $U$ is open since $x$ was arbitrary.
    This completes the proof that $f$ is continuous.

    Next we show that $\inv{f}$ is continuous, which is a little simpler.
    Let $B$ be any basis element of $\doms$ so that $B = \cpfin{U}{n}$ where each $U_k$ is open in $X_k$ by the definition of the box topology.
    Then we have that $f(B) = (\cpfin{U}{n-1}) \times U_n$.
    By the definition of the box topology, we then have that $U' = \cpfin{U}{n-1}$ is a basis element of $\cpfin{X}{n-1}$ and is therefore open.
    Since $U_n$ is also open, we have that $f(B) = U' \times U_n$ is a basis element of $\rans$ by the definition of the box/product topology, and is therefore open.
    Since $f(B) = \inv{(\inv{f})}(B)$ is the inverse image of $B$ under $\inv{f}$, this shows that $\inv{f}$ is also continuous.

    We have shown that both $f$ and $\inv{f}$ are continuous, which proves that $f$ is a homeomorphism by definition.
  }
}

\exercise{5}{
  One of the implications stated in Theorem~19.6 holds for the box topology. Which one?
}
\sol{
  \dwhitman

  Example 19.2 gives a function $f$ that is not continuous in the box topology even though all of its constituent functions $f_\a$ are continuous.
  Hence the only implication that can be generally true in the box topology is that $f$ being continuous implies that each $f_\a$ is continuous.
  A proof of this is straightforward.
  \qproof{
    As in Theorem~19.6 suppose that $f: A \to \prod_{\a \in J} X_\a$ be given by
    \gath{
      f(a) = (f_\a(a))_{\a \in J} \,,
    }
    where $f_\a: A \to X_\a$ for each $\a \in J$.
    Here $\prod X_\a$ has the box topology.
    Suppose that $f$ is continuous and consider any $\b \in J$.
    We show that $f_\b$ is continuous, which of course shows the desired result.

    So let $V$ be any open set of $X_\b$ and define
    \gath{
      B_\a = \begin{cases}
        V & \a = \b \\
        X_\a & \a \neq \b \,.
      \end{cases}
    }
    Then, since each $B_\a$ is clearly open in $X_\a$, we have that $B = \prod B_\a$ is a basis element of the box topology by definition and is therefore open.
    Hence $U = \inv{f}(B)$ is open in $A$ since $f$ is continuous.
    We claim that $U = \inv{f_\b}(V)$, which shows that $f_\b$ is continuous since $U$ is open in $A$ and $V$ was an arbitrary open set of $X_\b$.

    $(\ss)$ If $x \in U = \inv{f}(B)$ then of course $f(x) \in B$ so that each $f_\a(x) \in B_\a$ since $f(x) = (f_\a(x))_{\a \in J}$ and $B = \prod B_\a$.
    In particular $f_\b(x) \in B_\b = V$ so that $x \in \inv{f_\b}(V)$.
    Hence $U \ss \inv{f_\b}(V)$ since $x$ was arbitrary.

    $(\sps)$ If $x \in \inv{f_\b}(V)$ then $f_\b(x) \in V = B_\b$.
    Since of course every other $f_\a(x) \in X_\a = B_\a$ we have that $f(x) \in \prod B_\a = B$.
    Hence $x \in \inv{f}(B) = U$ so that $\inv{f_\b}(V) \ss U$ since $x$ was arbitrary.
  }
}
