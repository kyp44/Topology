\setcounter{subsection}{10-1}
\subsection{Well-Ordered Sets}

\exercise{1}{
  Show that every well-ordered set has the least upper bound property.
}
\sol{
  \dwhitman

  \qproof{
    Suppose that $A$ is a set with well-ordering $<$, and that $B$ is some nonempty subset of $A$ with upper bound $a \in A$.
    Let $C$ then be the set of upper bounds of $B$, which is not empty since clearly $a \in C$.
    Then $C$ is a nonempty subset of $A$ and so has a smallest element $c$ since $A$ is well-ordered.
    Clearly then $c$ is the least upper bound of $B$ by definition.
    This shows that $A$ has the least upper bound property since $B$ was arbitrary.
  }
}

\exercise{2}{
  \eparts{
  \item Show that in a well-ordered set, every element except the largest (if one exists) has an immediate successor.
  \item Find a set in which every element has an immediate successor that is not well-ordered.
  }
}
\sol{
  \dwhitman

  (a)
  \qproof{
    Suppose that $A$ is well-ordered by $<$ and consider any $a \in A$ where $a$ is not the largest element.
    It then follows that there is some $x \in A$ where $a < x$ since otherwise $a$ would be the largest element of $A$.
    Let $X = \braces{y \in A \where a < y}$ so that clearly $X \ss A$ and $x \in X$.
    Thus $X$ is a nonempty subset of $A$ and so has a smallest element $b$ since $<$ well-orders $A$.
    We claim that $b$ is the immediate successor of $a$.
    To see this suppose that there is a $z \in A$ such that $a < z < b$, noting that clearly $a < b$ since $b \in X$.
    Then we would have that $z \in X$ but $z < b$ so that it is not true that $b \leq z$, which contradicts the definition of $b$ as the smallest element of $X$.
    So it must be that no such $z$ exists, which shows that $b$ is indeed the immediate successor of $a$.
  }

  (b) The most natural example of such a set is $\ints$.
  We show that this has the desired properties.
  \qproof{
    First, clearly $\ints$ is not well-ordered since, for example, the set of negative integers is a nonempty subset of $\ints$ but has no smallest element.
    Also, for any $n \in \ints$, clearly $n+1$ is the immediate successor of $n$, which was shown back in Corollary~\ref{cor:intreal:intbet}.
  }
}

\def\ot{\braces{1,2}}
\exercise{3}{
  Both $\ot \times \pints$ and $\pints \times \ot$ are well-ordered in the dictionary order.
  Do they have the same order type?
}
\sol{
  \dwhitman

  We claim that they do \emph{not} have the same order type, which we show presently.
  \qproof{
    First, clearly $(1,1)$ is the smallest element of both ordered sets.
    For brevity let $A = \ot \times \pints$, $B = \pints \times \ot$, and $<_A$ and $<_B$ be the corresponding dictionary orderings, with $<$ being the normal ordering of $\pints$.

    So assume that they \emph{do} have the same order type so that there is an order-preserving bijection $f : A \to B$.
    Consider $(2,1) \in A$, which is clearly not the smallest element since $(2,1) \neq (1,1)$.
    Let $(n,b) = f(2,1) \in B$, which cannot be the smallest element of $B$ since $f$ preserves order, so that $(n,b) \neq (1,1)$.
    Clearly $b \in \ot$ so that $b=1$ or $b=2$.
    In the former cases we must have that $n > 1$ so that $n-1 \in \pints$.
    So set $y = (n-1,2)$.
    In the latter case set $y = (n, 1)$.
    It is easy to see, and trivial to formally show, that $y$ is the immediate predecessor of $(n,b)$ in either case.

    Now let $x = \inv{f}(y)$, noting that $\inv{f}$ is an order-preserving bijection from $B$ to $A$ since $f$ is an order-preserving bijection.
    It then follows that $x <_A (2, 1)$ since $f(x) = y <_B (n,b) = f(2,1)$.
    If $x = (m,a)$ then it has to be that $m < 2$ so that $m = 1$ (because $m \in \ot$) since there is no $a \in \pints$ where $a < 1$.
    Thus $x = (1, a)$ for some $a \in \pints$.
    We then have that $a+1 \in \pints$ so that clearly $x = (1, a) <_A (1, a+1) <_A (2,1)$.
    From this we have, $y = f(1,a) <_B f(1,a+1) <_B f(2,1) = (n,b)$, which contradicts the fact that $y$ is the immediate predecessor of $(n,b)$.
    So it has to be that they do not have the same order type.
  }

  It is worth noting that, in the theory of ordinal numbers, $A = \ot \times \pints$ has order type $\w + \w = \w \cdot 2$ whereas $B = \pints \times \ot$ has simply order type $\w$.
  This also shows that $A$ and $B$  have different order types since distinct ordinal numbers always have different order types.
}

\exercise{4}{
  \eparts{
  \item Let $\nints$ denote the set of negative integers in the usual order.
    Show that a simple ordered set $A$ fails to be well-ordered if and only if it contains a subset having the same order type as $\nints$.
  \item Show that if $A$ is simply ordered and every countable subset of $A$ is well-ordered, then $A$ is well-ordered.
  }
}
\sol{
  \dwhitman

  (a)
  \qproof{
    Let $A$ be a set with simple order $\prec$.
    
    $(\imp)$ Suppose that $\prec$ is not a well-ordering of $A$.
    Then there exists a nonempty subset $B$ of $A$ such that $B$ has no smallest element.
    For any $b \in B$ define the set $X_b = \braces{x \in B \where x \prec b}$.
    Clearly $X_b \ss B$ and $X_b \neq \es$ for any $b \in B$ since otherwise $b$ would be the smallest element of $B$.
    Now let $c$ be a choice function on the collection of nonempty subsets of $B$, which of course exists by the axiom of choice.
    Since $B$ is nonempty there is a $b_0 \in B$.
    It then follows from the principle of recursive definition that there is a function $f : \pints \to B$ such that
    \ali{
      f(1) &= b_0 \,, \\
      f(n) &= c(X_{f(n-1)}) \condgap \text{for $n > 1$.}
    }
    It then is easy to show that $f(n+1) \prec f(n)$ for all $n \in \pints$, i.e. that the sequence defined by $f$ is decreasing.
    If we then simply define $g : \nints \to \pints$ by $g(n) = -n$ for $n \in \nints$, it is clear that $f \circ g$ is an order-preserving bijection from $\nints$ to some subset $C$ of $B$.
    Clearly also $C \ss A$ since $B \ss A$.
    Hence the subset $C$ has the same order type as $\nints$.

    $(\pmi)$ Now suppose that $A$ has a subset $B$ with the same order type as $\nints$.
    Clearly then $B$ is nonempty and has no smallest element since $\nints$ does not.
    The existence of this $B$  shows that $A$ fails to be well-ordered.
  }

  (b)
  \qproof{
    Suppose that $A$ is a set that is simply ordered by $\prec$ such that every countable subset is well-ordered by $\prec$.
    Consider any nonempty subset $B \ss A$.
    Suppose for a moment that the restricted $\prec$ does not well-order $B$.
    Then it follows from part (a) that $B$ has a subset $C$ with the same order type as $\nints$.
    However, clearly $C \ss A$ (since $B \ss A$) and $C$ is countable (since $\nints$ is countable) and thus it should be well-ordered.
    As this is impossible since $C$ has the same order-type as $\nints$ (which is clearly not well-ordered), it has to be that the restricted $\prec$ does in fact well-order $B$.
    Hence $B$ has a $\prec$-smallest element, which shows that $A$ is well-ordered since $B$ was arbitrary.
  }
}

\exercise{5}{
  Show the well-ordering theorem implies the choice axiom.
}
\sol{
  \dwhitman

  \qproof{
    Suppose that $\col{A}$ is a collection of nonempty sets.
    Then, by the well-ordering theorem there is a well-ordering $<$ of $\bigcup \col{A}$.
    We construct a choice function $c : \col{A} \to \bigcup \col{A}$.
    Consider any set $A \in \col{A}$.
    Since clearly $A$ is then a nonempty subset of $\bigcup \col{A}$, it follows that it has a unique smallest element $a$ according to $<$ since $\bigcup \col{A}$ is well-ordered by $<$.
    So simply set $c(A) = a$ so that clearly then $c(A) = a \in A$.
    This shows that $c$ is in fact a choice function on $\col{A}$.
  }
}

\exercise{6}{
  Let $S_\W$ be the minimal uncountable well-ordered set.
  \eparts{
  \item Show that $S_\W$ has no largest element.
  \item Show that for every $\a \in S_\W$, the subset $\braces{x \where \a < x}$ is uncountable.
  \item Let $X_0$ be the subset of $S_\W$ consisting of all elements $x$ such that $x$ has no immediate predecessor.
    Show that $X_0$ is uncountable.
  }
}
\sol{
  \dwhitman

  \begin{lem}\label{lem:wellord:uncount}
    If $A$ is an uncountable set and $B \ss A$ is countable then $A - B$ is uncountable.
  \end{lem}
  \qproof{
    If we let $C = A - B$, then clearly $A = C \cup B$.
    If $C$ were countable then $A = C \cup B$ would be countable by Theorem~7.5 since $B$ is also countable.
    Since we know that $A$ is uncountable it therefore must be that $C = A - B$ is uncountable as well.
  }

  \mainprob

  It is assumed in the following that $<$ is the well-order on $S_\W$.
  
  (a)
  \qproof{
    Suppose to the contrary that $S_\W$ does have a largest element $\a$.
    Then, for any $x \in S_\W$, we have that $x \leq \a$.
    Hence either $x \in \braces{y \in S_\W \where y < \a} = S_\a$ or $x = \a$.
    Therefore $S_\W = S_\a \cup \braces{\a}$ since clearly both $S_\a$ and $\braces{\a}$ are both subsets of $S_\W$.
    Now since $S_\a$ is a section of $S_\W$, it is countable.
    Since $\braces{\a}$ is also clearly countable, it follows from Theorem~7.5 that their union $S_\a \cup \braces{\a} = S_\W$ is countable.
    But this contradicts the fact that $S_\W$ is uncountable!
    Hence it has to be that $S_\W$ has no largest element as desired.
  }

  (b)
  \qproof{
    Consider any $\a \in S_\W$.
    Let $T_\a = \braces{x \in S_\W \where \a < x}$ so that we must show that $T_\a$ is uncountable.
    Let $\bar{S}_\a = S_\a \cup \braces{\a}$ so that clearly we have that $\bar{S}_\a = \braces{x \in S_\W \where x \leq \a}$.
    It is then easy to show that $T_\a = S_\W - \bar{S}_\a$.
    Now, since $S_\a$ is a section of $S_\W$, it is countable so that clearly $\bar{S}_\a = S_\a \cup \braces{\a}$ is also countable by Theorem~7.5.
    Then, since $S_\W$ itself is uncountable, it follows that $T_\a = S_\W - \bar{S}_\a$ is also uncountable by Lemma~\ref{lem:wellord:uncount}.
  }

  (c)
  \qproof{
    First we show that $X_0$ is not bounded above.
    Assume the contrary so that $\a \in S_\W$ is an upper bound of $X_0$.
    It then follows that the set $T_\a = \braces{x \in S_\W \where \a < x}$ is such that every element of $T_\a$ has an immediate predecessor since otherwise there would be a $\b \in T_\a$ where $\b \in X_0$ so that $\a$ would not be an upper bound of $X_0$ since then $\a < \b$.

    Now, we know from part (a) that $S_\W$ has no largest element so that it follows from Exercise~10.2 that every element of $S_\W$ has an immediate successor.
    Since $T_\a \ss S_\W$ it follows that each element $x \in T_\a$ has an immediate successor $y$.
    Moreover we then have that $\a < x < y$ so that $y \in T_\a$ also.
    Hence every element of $T_\a$ has an immediate successor in $T_\a$.
    
    Now, we know that $T_\a$ is uncountable by part (b) so that it has a smallest element $\b$ since it is then a nonempty subset of the well-ordered $S_\W$.
    We derive a contradiction by showing that $T_\a$ has the same order type as $\pints$ and is thus countable.
    We do this by defining an increasing bijection $f: \pints \to T_\a$.
    First, set $f(1) = \b$ and then set $f(n)$ to the immediate successor of $f(n-1)$ for $n > 1$, which was shown to exist above.
    Then the function $f$ uniquely exists by the principle of recursive definition.
    Clearly we have that $f(n+1) > f(n)$ for all $n \in \pints$ since $f(n+1)$ is the immediate successor of $f(n)$.
    Hence $f$ is increasing and therefore also injective.

    To show that $f$ is surjective suppose the contrary so that the set $T_\a - f(\pints)$ is nonempty.
    Since clearly this is a subset of the well-ordered $S_\W$, it has a smallest element $y$.
    Now, we know that $f(1) = \b$ so that $y \neq \b$, and in fact $\b < y$ since $\b$ is the smallest element of $T_\a$.
    Since $y \in T_\a$ we know that it has an immediate predecessor $x$ and that $\a < \b \leq x$ so that $x \in T_\a$.
    However, it cannot be that $x \in T_\a - f(\pints)$ since $x < y$ and $y$ is the smallest element of $T_\a - f(\pints)$.
    Thus $x \in f(\pints)$ so that there is an $n \in \pints$ where $f(n) = x$.
    But then $f(n+1) = y$ since $y$ is the immediate successor of $x$.
    As this contradicts the fact that $y \notin f(\pints)$, it must be that $f$ is in fact surjective!

    Therefore we have shown that $f$ is a bijection from $\pints$ to $T_\a$ so that $T_\a$ is countable.
    But we know from part (b) that $T_\a$ is uncountable.
    As mentioned above, this is a contradiction so that it must be that indeed $X_0$ is not bounded above.
    From this it immediately follows from the contrapositive of Theorem~10.3 that $X_0$ must be uncountable.
  }
}
