\setcounter{subsection}{10-1}
\subsection{Well-Ordered Sets}

\exercise{1}{
  Show that every well-ordered set has the least upper bound property.
}
\sol{
  \qproof{
    Suppose that $A$ is a set with well-ordering $<$, and that $B$ is some nonempty subset of $A$ with upper bound $a \in A$.
    Let $C$ then be the set of upper bounds of $B$, which is not empty since clearly $a \in C$.
    Then $C$ is a nonempty subset of $A$ and so has a smallest element $c$ since $A$ is well-ordered.
    Clearly then $c$ is the least upper bound of $B$ by definition.
    This shows that $A$ has the least upper bound property since $B$ was arbitrary.
  }
}

\exercise{2}{
  \eparts{
  \item Show that in a well-ordered set, every element except the largest (if one exists) has an immediate successor.
  \item Find a set in which every element has an immediate successor that is not well-ordered.
  }
}
\sol{
  (a)
  \qproof{
    Suppose that $A$ is well-ordered by $<$ and consider any $a \in A$ where $a$ is not the largest element.
    It then follows that there is some $x \in A$ where $a < x$ since otherwise $a$ would be the largest element of $A$.
    Let $X = \braces{y \in A \where a < y}$ so that clearly $X \ss A$ and $x \in X$.
    Thus $X$ is a nonempty subset of $A$ and so has a smallest element $b$ since $<$ well-orders $A$.
    We claim that $b$ is the immediate successor of $a$.
    To see this suppose that there is a $z \in A$ such that $a < z < b$, noting that clearly $a < b$ since $b \in X$.
    Then we would have that $z \in X$ but $z < b$ so that it is not true that $b \leq z$, which contradicts the definition of $b$ as the smallest element of $X$.
    So it must be that no such $z$ exists, which shows that $b$ is indeed the immediate successor of $a$.
  }

  (b) The most natural example of such a set is $\ints$.
  We show that this has the desired properties.
  \qproof{
    First, clearly $\ints$ is not well-ordered since, for example, the set of negative integers is a nonempty subset of $\ints$ but has no smallest element.
    Also, for any $n \in \ints$, clearly $n+1$ is the immediate successor of $n$, which was shown back in Corollary~\ref{cor:intreal:intbet}.
  }
}

\def\ot{\braces{1,2}}
\exercise{3}{
  Both $\ot \times \pints$ and $\pints \times \ot$ are well-ordered in the dictionary order.
  Do they have the same order type?
}
\sol{
  We claim that they do \emph{not} have the same order type, which we show presently.
  \qproof{
    First, clearly $(1,1)$ is the smallest element of both ordered sets.
    For brevity let $A = \ot \times \pints$, $B = \pints \times \ot$, and $<_A$ and $<_B$ be the corresponding dictionary orderings, with $<$ being the normal ordering of $\pints$.

    So assume that they \emph{do} have the same order type so that there is an order-preserving bijection $f : A \to B$.
    Consider $(2,1) \in A$, which is clearly not the smallest element since $(2,1) \neq (1,1)$.
    Let $(n,b) = f(2,1) \in B$, which cannot be the smallest element of $B$ since $f$ preserves order, so that $(n,b) \neq (1,1)$.
    Clearly $b \in \ot$ so that $b=1$ or $b=2$.
    In the former cases we must have that $n > 1$ so that $n-1 \in \pints$.
    So set $y = (n-1,2)$.
    In the latter case set $y = (n, 1)$.
    It is easy to see, and trivial to formally show, that $y$ is the immediate predecessor of $(n,b)$ in either case.

    Now let $x = \inv{f}(y)$, noting that $\inv{f}$ is an order-preserving bijection from $B$ to $A$ since $f$ is an order-preserving bijection.
    It then follows that $x <_A (2, 1)$ since $f(x) = y <_B (n,b) = f(2,1)$.
    If $x = (m,a)$ then it has to be that $m < 2$ so that $m = 1$ (because $m \in \ot$) since there is no $a \in \pints$ where $a < 1$.
    Thus $x = (1, a)$ for some $a \in \pints$.
    We then have that $a+1 \in \pints$ so that clearly $x = (1, a) <_A (1, a+1) <_A (2,1)$.
    From this we have, $y = f(1,a) <_B f(1,a+1) <_B f(2,1) = (n,b)$, which contradicts the fact that $y$ is the immediate predecessor of $(n,b)$.
    So it has to be that they do not have the same order type.
  }

  It is worth noting that, in the theory of ordinal numbers, $A = \ot \times \pints$ has order type $\w + \w = \w \cdot 2$ whereas $B = \pints \times \ot$ has simply order type $\w$.
  This also shows that $A$ and $B$  have different order types since distinct ordinal numbers always have different order types.
}

\exercise{4}{
  \eparts{
  \item Let $\nints$ denote the set of negative integers in the usual order.
    Show that a simple ordered set $A$ fails to be well-ordered if and only if it contains a subset having the same order type as $\nints$.
  \item Show that if $A$ is simply ordered and every countable subset of $A$ is well-ordered, then $A$ is well-ordered.
  }
}
\sol{
  (a)
  \qproof{
    Let $A$ be a set with simple order $\prec$.
    
    $(\imp)$ Suppose that $\prec$ is not a well-ordering of $A$.
    Then there exists a nonempty subset $B$ of $A$ such that $B$ has no smallest element.
    For any $b \in B$ define the set $X_b = \braces{x \in B \where x \prec b}$.
    Clearly $X_b \ss B$ and $X_b \neq \es$ for any $b \in B$ since otherwise $b$ would be the smallest element of $B$.
    Now let $c$ be a choice function on the collection of nonempty subsets of $B$, which of course exists by the axiom of choice.
    Since $B$ is nonempty there is a $b_0 \in B$.
    It then follows from the principle of recursive definition that there is a function $f : \pints \to B$ such that
    \ali{
      f(1) &= b_0 \,, \\
      f(n) &= c(X_{f(n-1)}) \condgap \text{for $n > 1$.}
    }
    It then is easy to show that $f(n+1) \prec f(n)$ for all $n \in \pints$, i.e. that the sequence defined by $f$ is decreasing.
    If we then simply define $g : \nints \to \pints$ by $g(n) = -n$ for $n \in \nints$, it is clear that $f \circ g$ is an order-preserving bijection from $\nints$ to some subset $C$ of $B$.
    Clearly also $C \ss A$ since $B \ss A$.
    Hence the subset $C$ has the same order type as $\nints$.

    $(\pmi)$ Now suppose that $A$ has a subset $B$ with the same order type as $\nints$.
    Clearly then $B$ is nonempty and has no smallest element since $\nints$ does not.
    The existence of this $B$  shows that $A$ fails to be well-ordered.
  }

  (b)
  \qproof{
    Suppose that $A$ is a set that is simply ordered by $\prec$ such that every countable subset is well-ordered by $\prec$.
    Consider any nonempty subset $B \ss A$.
    Suppose for a moment that the restricted $\prec$ does not well-order $B$.
    Then it follows from part~(a) that $B$ has a subset $C$ with the same order type as $\nints$.
    However, clearly $C \ss A$ (since $B \ss A$) and $C$ is countable (since $\nints$ is countable) and thus it should be well-ordered.
    As this is impossible since $C$ has the same order-type as $\nints$ (which is clearly not well-ordered), it has to be that the restricted $\prec$ does in fact well-order $B$.
    Hence $B$ has a $\prec$-smallest element, which shows that $A$ is well-ordered since $B$ was arbitrary.
  }
}

\exercise{5}{
  Show the well-ordering theorem implies the choice axiom.
}
\sol{
  \qproof{
    Suppose that $\col{A}$ is a collection of nonempty sets.
    Then, by the well-ordering theorem there is a well-ordering $<$ of $\bigcup \col{A}$.
    We construct a choice function $c : \col{A} \to \bigcup \col{A}$.
    Consider any set $A \in \col{A}$.
    Since clearly $A$ is then a nonempty subset of $\bigcup \col{A}$, it follows that it has a unique smallest element $a$ according to $<$ since $\bigcup \col{A}$ is well-ordered by $<$.
    So simply set $c(A) = a$ so that clearly then $c(A) = a \in A$.
    This shows that $c$ is in fact a choice function on $\col{A}$.
  }
}

\exercise{6}{
  Let $S_\W$ be the minimal uncountable well-ordered set.
  \eparts{
  \item Show that $S_\W$ has no largest element.
  \item Show that for every $\a \in S_\W$, the subset $\braces{x \where \a < x}$ is uncountable.
  \item Let $X_0$ be the subset of $S_\W$ consisting of all elements $x$ such that $x$ has no immediate predecessor.
    Show that $X_0$ is uncountable.
  }
}
\sol{
  \begin{lem}\label{lem:wellord:uncount}
    If $A$ is an uncountable set and $B \ss A$ is countable then $A - B$ is uncountable.
  \end{lem}
  \qproof{
    If we let $C = A - B$, then clearly $A = C \cup B$.
    If $C$ were countable then $A = C \cup B$ would be countable by Theorem~7.5 since $B$ is also countable.
    Since we know that $A$ is uncountable it therefore must be that $C = A - B$ is uncountable as well.
  }

  \mainprob

  It is assumed in the following that $<$ is the well-order on $S_\W$.
  
  (a)
  \qproof{
    Suppose to the contrary that $S_\W$ does have a largest element $\a$.
    Then, for any $x \in S_\W$, we have that $x \leq \a$.
    Hence either $x \in \braces{y \in S_\W \where y < \a} = S_\a$ or $x = \a$.
    Therefore $S_\W = S_\a \cup \braces{\a}$ since clearly both $S_\a$ and $\braces{\a}$ are both subsets of $S_\W$.
    Now since $S_\a$ is a section of $S_\W$, it is countable.
    Since $\braces{\a}$ is also clearly countable, it follows from Theorem~7.5 that their union $S_\a \cup \braces{\a} = S_\W$ is countable.
    But this contradicts the fact that $S_\W$ is uncountable!
    Hence it has to be that $S_\W$ has no largest element as desired.
  }

  (b)
  \qproof{
    Consider any $\a \in S_\W$.
    Let $T_\a = \braces{x \in S_\W \where \a < x}$ so that we must show that $T_\a$ is uncountable.
    Let $\bar{S}_\a = S_\a \cup \braces{\a}$ so that clearly we have that $\bar{S}_\a = \braces{x \in S_\W \where x \leq \a}$.
    It is then easy to show that $T_\a = S_\W - \bar{S}_\a$.
    Now, since $S_\a$ is a section of $S_\W$, it is countable so that clearly $\bar{S}_\a = S_\a \cup \braces{\a}$ is also countable by Theorem~7.5.
    Then, since $S_\W$ itself is uncountable, it follows that $T_\a = S_\W - \bar{S}_\a$ is also uncountable by Lemma~\ref{lem:wellord:uncount}.
  }

  (c)
  \qproof{
    First we show that $X_0$ is not bounded above.
    Assume the contrary so that $\a \in S_\W$ is an upper bound of $X_0$.
    It then follows that the set $T_\a = \braces{x \in S_\W \where \a < x}$ is such that every element of $T_\a$ has an immediate predecessor since otherwise there would be a $\b \in T_\a$ where $\b \in X_0$ so that $\a$ would not be an upper bound of $X_0$ since then $\a < \b$.

    Now, we know from part~(a) that $S_\W$ has no largest element so that it follows from Exercise~10.2 that every element of $S_\W$ has an immediate successor.
    Since $T_\a \ss S_\W$ it follows that each element $x \in T_\a$ has an immediate successor $y$.
    Moreover we then have that $\a < x < y$ so that $y \in T_\a$ also.
    Hence every element of $T_\a$ has an immediate successor in $T_\a$.
    
    Now, we know that $T_\a$ is uncountable by part~(b) so that it has a smallest element $\b$ since it is then a nonempty subset of the well-ordered $S_\W$.
    We derive a contradiction by showing that $T_\a$ has the same order type as $\pints$ and is thus countable.
    We do this by defining an increasing bijection $f: \pints \to T_\a$.
    First, set $f(1) = \b$ and then set $f(n)$ to the immediate successor of $f(n-1)$ for $n > 1$, which was shown to exist above.
    Then the function $f$ uniquely exists by the principle of recursive definition.
    Clearly we have that $f(n+1) > f(n)$ for all $n \in \pints$ since $f(n+1)$ is the immediate successor of $f(n)$.
    Hence $f$ is increasing and therefore also injective.

    To show that $f$ is surjective suppose the contrary so that the set $T_\a - f(\pints)$ is nonempty.
    Since clearly this is a subset of the well-ordered $S_\W$, it has a smallest element $y$.
    Now, we know that $f(1) = \b$ so that $y \neq \b$, and in fact $\b < y$ since $\b$ is the smallest element of $T_\a$.
    Since $y \in T_\a$ we know that it has an immediate predecessor $x$ and that $\a < \b \leq x$ so that $x \in T_\a$.
    However, it cannot be that $x \in T_\a - f(\pints)$ since $x < y$ and $y$ is the smallest element of $T_\a - f(\pints)$.
    Thus $x \in f(\pints)$ so that there is an $n \in \pints$ where $f(n) = x$.
    But then $f(n+1) = y$ since $y$ is the immediate successor of $x$.
    As this contradicts the fact that $y \notin f(\pints)$, it must be that $f$ is in fact surjective!

    Therefore we have shown that $f$ is a bijection from $\pints$ to $T_\a$ so that $T_\a$ is countable.
    But we know from part~(b) that $T_\a$ is uncountable.
    As mentioned above, this is a contradiction so that it must be that indeed $X_0$ is not bounded above.
    From this it immediately follows from the contrapositive of Theorem~10.3 that $X_0$ must be uncountable.
  }

  It is interesting to note that $S_\W$ corresponds to the ordinal number $\w_1$, which is the first uncountable ordinal, and the set $X_0$ of part~(c) corresponds to the set of limit ordinals in $\w_1$.
  All of the curious properties deduced here for $S_\W$ apply to $\w_1$ too, assuming we allow the choice axiom.
}

\exercise{7}{
  Let $J$ be a well-ordered set.
  A subset $J_0$ of $J$ is said to be \boldit{inductive} if for every $\a \in J$,
  \gath{
    \parens{S_\a \ss J_0} \imp \a \in J_0 \,.
  }
  \emph{Theorem (The principle of transfinite induction).}
  If $J$ is a well-ordered set and $J_0$ is an inductive subset of $J$, then $J_0 = J$.
}
\sol{
  \qproof{
    Suppose that $J_0$ is an inductive subset of the well-ordered set $J$.
    Also suppose that $J_0 \neq J$.
    Since $J_0 \ss J$, it follows that there must be an $x \in J$ such that $x \notin J_0$.
    Thus the set $J - J_0$ is nonempty.
    Since clearly this is also a subset of $J$, it must have a smallest element $\a$ since $J$ is well-ordered.
    Consider any $y \in S_\a$ so that $y < \a$.
    Then it cannot be that $y \in J - J_0$ since otherwise $\a$ would not be the smallest element of $J - J_0$.
    Since clearly $y \in J$ (since $S_\a \ss J$) it has to be that $y \in J_0$.
    Since $y$ was arbitrary this shows that $S_\a \ss J_0$.
    It then follows that $\a \in J_0$ since $J_0$ is inductive.
    However, this contradicts the fact that $\a \in J - J_0$ so that our initial supposition that $J_0 \neq J$ must be incorrect.
    Hence $J_0 = J$ as desired.
  }
}

\exercise{8}{
  \eparts{
  \item Let $A_1$ and $A_2$ be disjoint sets, well-ordered by $<_1$ and $<_2$, respectively.
    Define an order relation on $A_1 \cup A_2$ by letting $a < b$ either if $a,b \in A_1$ and $a <_1 b$, or if $a,b \in A_2$ and $a <_2 b$, or if $a \in A_1$ and $b \in A_2$.
    Show that this is a well-ordering.
  \item Generalize (a) to an arbitrary family of disjoint well-ordered sets, indexed by a well-ordered set.
  }
}
\sol{
  (a)
  \qproof{
    It is easy but tedious to show that $<$ is actually an order on $A_1 \cup A_2$, so we shall skip that proof and jump straight to the proof that it is a well-ordering.

    So consider any nonempty subset $A$ of $A_1 \cup A_2$.

    Case: $A_1 \cap A \neq \es$.
    Then clearly $A_1 \cap A$ is a nonempty subset of $A_1$ so that it has a smallest element $a$ according to $<_1$ since it is a well-ordering.
    We then claim that $a$ is the smallest element of $A$ according to $<$.
    So consider any $x \in A$ so that clearly also $x \in A_1 \cup A_2$.
    Hence $x \in A_1$ or $x \in A_2$.
    If $x \in A_1$ then obviously $x \in A_1 \cap A$ so that $a \leq_1 x$ since $a$ is the smallest element of $A_1 \cap A$.
    Then also $a \leq x$ by definition since $a$ and $x$ are both in $A_1$.
    On the other hand, if $x \in A_2$ then we again have that $a < x$ since $a \in A_1$ and $x \in A_2$.
    Therefore $a \leq x$ no matter what so that $a$ is the smallest element of $A$ since $x$ was arbitrary.

    Case: $A_1 \cap A = \es$.
    Then it has to be that $A_2 \cap A \neq \es$ since $A$ is nonempty and $A = A_1 \cup A_2$.
    Thus $A_2 \cap A$ is a nonempty subset of $A_2$ so that it has a smallest element $a$ by $<_2$ since it is a well-ordering.
    We claim that $a$ is the smallest element of $A$.
    So consider any $x \in A$.
    It has to be that $x \in A_2$ since $A_1 \cap A$ is empty and $A = A_1 \cup A_2$.
    Therefore $x \in A_2 \cap A$ so that $a \leq_2 x$ since $a$ is the smallest element of $A_2 \cap A$.
    Then, by definition, $a \leq x$ since both $a$ and $x$ are elements of $A_2$.
    This shows that $a$ is the smallest element of $A$ since $x$ was arbitrary.

    In either case we have shown that $A$ has a smallest element so that $<$ is a well-ordering of $A_1 \cup A_2$ since $A$ was arbitrary.
  }

  Note that well-ordering a union of two well-ordered sets like this is analogous to the addition of two ordinal numbers.
  In particular if $A_1$ has order type $\a_1$ and $A_1$ has order type $\a_2$ where $\a_1$ and $\a_2$ are an ordinal numbers, then $A_1 \cup A_2$ with the above well-ordering has order type $\a_1 + \a_2$.

  (b) Suppose that $J$ is well-ordered by $<_J$ and $\braces{A_\a}_{\a \in J}$ is a collection of well-ordered sets where $A_\a$ is well-ordered by $<_\a$ for each $\a \in J$.
  Now define an order $<$ on $A = \bigcup_{\a \in J} A_\a$ as follows.
  For any $x$ and $y$ in $A$ there are clearly $\a$ and $\b$ in $J$  where $x \in A_\a$ and $y \in A_\b$, noting that $\a$ and $\b$ are unique since the collection is mutually disjoint.
  So set $x < y$ if and only if either $\a = \b$ and $x <_\a y$, or else $\a <_J \b$, noting that these are clearly mutually exclusive.
  We then claim that $<$ is a well-ordering of $A$.
  \qproof{
    Let $B$ be any nonempty subset of $A$ and $I$ be the set of $\a \in J$ such that there is an $x \in B$ where $x \in A_\a$.
    Now, since $B$ is nonempty, there is a $z \in B$.
    Since $B \ss A = \bigcup_{\a \in J} A_\a$, there is an $\g \in J$ where $z \in A_\g$.
    Then clearly $\g \in I$ so that $I$ is a nonempty subset of $J$.
    Then $I$ has a smallest element $\a$ since it is well-ordered by $<_J$.
    By the definition of $I$ there is a $w \in B$ where $w \in A_\a$.
    Then clearly $w \in A_\a \cap B$ so that it is a nonempty subset of $A_\a$.
    It then follows that $A_\a \cap B$ has a smallest element $a$ according to $<_\a$ since it is a well-ordering on $A_\a$.
    We claim that $a$ is the smallest element of $B$ by $<$.

    So consider any $x \in B$ so that there is a $\b \in J$ where $x \in A_\b$ since $B \ss A$.

    Case: $\b = \a$.
    Then both $a$ and $x$ are in $A_\a \cap B = A_\b \cap B$ so that $a \leq_\a b$ since $a$ is the smallest element of $A_\a \cap B$.
    It then follows from the definition of $<$ that $a \leq x$.

    Case: $\b \neq \a$.
    Clearly then $\b \in I$ so that $\a \leq_J \b$ since $\a$ is the smallest element of $J$.
    Since we know that $\b \neq \a$ it must be that $\a <_J \b$.
    From this it follows that $a < x$ by definition.

    Hence in either case it is true that $a \leq x$, which shows that $a$ is the smallest element of $B$.
    Since $B$ was an arbitrary nonempty subset of $A$, this shows that $A$ is well-ordered by $<$.
  }
}

\exercise{9}{
  Consider the subset $A$ of $(\pints)^\w$ consisting of all infinite sequences of positive integers $\vx = \seqinf{x}$ that end in an infinite string of 1's.
  Give $A$ the following order: $\vx < \vy$ if $x_n < y_n$ and $x_i = y_i$ for $i > n$.
  We call this the ``antidictionary order'' on $A$.
  \eparts{
  \item Show that for every $n$, there is a section of $A$ that has the same order type as $(\pints)^n$ in the dictionary order.
  \item Show that $A$ is well-ordered.
  }
}
\sol{
  (a)
  \qproof{
    Consider any positive integer $n$.
    Define a sequence $\va = \seqinf{a}$ in $A$ by
    \gath{
      a_i = \begin{cases}
        2 & i = n + 1 \\
        1 & i \neq n + 1 \,.
      \end{cases}
    }
    We claim that the section $S_\va$ has the same order type as $(\pints)^n$.
    To show this we construct an order-preserving mapping $f: S_\va \to (\pints)^n$.
    So consider any sequence $\vx = \seqinf{x}$ in $S_\va$ so that $\vx < \va$.
    Now define a finite sequence where $y_i = x_{n-i+1}$ for any $i \in \intsfin{n}$, and set $f(\vx) = \vy = \seqfin{y}{n}$.
    Clearly $f(\vx) \in (\pints)^n$ since $\vx \in A \ss (\pints)^\w$.

    Here we must digress for a moment and show that, for all  $\vx = \seqinf{x} \in A$, $\vx \in S_\va$ if and only if $x_i = 1$ for all $i > n$.

    $(\imp)$ We show the contrapositive.
    So suppose that there is an $i > n$ where $x_i \neq 1$.
    Moreover let $i$ be the greatest such index, which must exist since $\vx$ must end in an infinite string of 1's.
    Clearly then the fact that $x_i \in \pints$ and $x_i \neq 1$ means that $x_i > 1$.
    Now, if $i > n+1$ then we have that $x_i > 1 = a_i$ and $x_j = 1 = a_j$ for all $j > i$ so that clearly $\vx > \va$.
    If $i = n + 1$ and $i > 2$ clearly $x_i > 2 = a_{n+1} = a_i$ and $x_j = 1 = a_j$ for all $j > i$ so that again $\vx > \va$.
    Lastly suppose that $i = n+1$ but that $x_i = 2 = a_{n+1} = a_i$.
    If $x_j = 1 = a_j$ for all $j < i = n+1$ then clearly $\vx = \va$.
    On the other hand if there is a $1 \leq j < n+1 = i$ where $x_j \neq 1$ then let $j$ be the greatest such index.
    Then we clearly have $x_j > 1 = a_j$ while $x_k = a_k$ for all $k > j$ so that $\vx > \va$.
    Therefore in every one of these exhaustive cases we have that $\vx \geq \va$ so that $\va \notin S_\va$.

    $(\pmi)$ Now suppose that $x_i = 1$ for every $i > n$.
    Then we have that $x_{n+1} = 1 < 2 = a_{n+1}$ while $x_j = 1 = a_j$ for all $j > n+1 > n$ so that $\vx < \va$ and hence $\vx \in S_\va$.

    Now, returning to the main proof, we first show that $f$ as defined above preserves order.
    To this end let $\prec$ denote the dictionary order on $(\pints)^n$.
    Now consider any $\vx = \seqinf{x}$ and $\vx' = \seqinf{x'}$ in $S_\va$ where $\vx < \vx'$.
    Also let $\vy = f(\vx)$ and $\vy' = f(\vx')$.
    Then there is an $m \in \pints$ where $x_m < x_m'$ and $x_i = x_i'$ for all $i > m$.
    We also have by what was shown above that $x_i = 1 x_i'$ for all $i > n$ since $\vx,\vx' \in S_\va$.
    So it has to be that $m \leq n$.
    It then follows from the fact that $1 \leq m \leq n$ that $1 \leq n-m+1 \leq n$ as well.
    Thus we have
    \gath{
      y_{n-m+1} = x_{n-(n-m+1)+1} = x_m < x_m' = x_{n-(n-m+1)+1}' = y_{n-m+1}' \,.
    }
    For any $1 \leq j < n-m+1$ we have that $n-j+1 > m$ so that
    \gath{
      y_j = x_{n-j+1} = x_{n-j+1}' = y_j' \,.
    }
    Thus by definition we have that $f(\vx) = \vy \prec \vy' = f(\vx')$, which shows that $f$ preserves order since $\vx$ and $\vx'$ were arbitrary.
    Note that this also clearly shows that $f$ is injective.

    To show that $f$ is also surjective, consider any $\vy = \seqfin{y}{n} \in (\pints)^n$.
    Now define a sequence
    \gath{
      x_i = \begin{cases}
        y_{n-i+1} & 1 \leq i \leq n \\
        1 & i > n
      \end{cases}
    }
    so that clearly $\vx = \seqinf{x} \in S_\va$ by what was shown above.
    Now let $\vy' = \seqfin{y'}{n} = f(\vx)$.
    Consider any $1 \leq i \leq n$ and let $j = n-i+1$ so that also $n-j+1 = i$, noting also that $1 \leq j \leq n$.
    Then we have
    \gath{
      y_i = y_{n-j+1} = x_j = x_{n-i+1} = y_i'
    }
    by the definition of $f$.
    Since $i$ was arbitrary this shows that $f(\vx) = \vy' = \vy$, which shows that $f$ is surjective since $\vy$ was arbitrary.

    The existence of $f$ therefore shows that $S_\va$ and $(\pints)^n$ have the same order type.
  }

  (b)
  \qproof{
    Consider any nonempty subset $B$ of $A$.
    Clearly the sequence $(1, 1, \ldots)$ is the smallest element of $A$ and hence if it is in $B$ then it is also the smallest element of $B$.
    So suppose that $(1, 1, \ldots) \notin B$ so that, for every $\vx \in B$ there is a unique greatest $n_\vx \in \pints$ where $x_{n_\vx} > 1$ but $x_i = 1$ for all $i > n_\vx$.
    So let $I = \braces{n_\vx \where \vx \in B}$, noting that $B \neq \es$ implies that $I \neq \es$ as well.
    Thus $I$ is a nonempty subset of $\pints$ and hence has a smallest element $n$.
    If we then let $B_n$ be the set of sequences $\vx \in B$ where $x_n > 1$ but $x_i = 1$ for all $i > n$, then the fact that $n \in I$ clearly implies that $B_n \neq \es$.
    Also, if we define the sequence
    \gath{
      a_i = \begin{cases}
        2 & i = n+1 \\
        1 & i \neq n+1
      \end{cases}
    }
    as in part~(a) then it follows from what was shown there that $B_n \ss S_\va$.
    Moreover it was shown that $S_\va$ has the same order type as the dictionary order of $(\pints)^n$, which we know to be a well-ordering.
    Hence $S_\va$ must also be a well-ordering so that $B_n$ has a smallest element $\vb = \seqinf{b}$ since it is a nonempty subset of $S_\va$.
    We claim that $\vb$ is in fact the smallest element of all of $B$.

    So consider any $\vx \in B$ so that $n_\vx \in I$.
    It then follows that $n \leq n_\vx$ since it is the smallest element of $I$.
    If $n = n_\vx$ then we have that $\vx \in B_n$ so that $\vb \leq \vx$ since it is the smallest element of $B_n$.
    If $n < n_\vx$ then we have that $b_{n_\vx} = 1 < x_{n_\vx}$ but $b_i = 1 = x_i$ for every $i > n_\vx > n$.
    This shows that $\vb < \vx$.
    Thus in all cases $\vb \leq \vx$, which shows that $\vb$ is the smallest element of $B$ since $\vx$ was arbitrary.
    Since $B$ was arbitrary, this shows that $A$ is well-ordered as desired.
  }

  Note that, in the theory of ordinal numbers, the set $(\pints)^n$ (and therefore the corresponding section of $A$) has order type $\w^n$.
  It would seem then that the set $A$ has order type $\w^\w$.
}

\exercise{10}{
  \emph{Theorem.} Let $J$ and $C$ be well-ordered sets; assume that there is no surjective function mapping a section of $J$ onto $C$.
  Then there exists a unique function $h: J \to C$ satisfying the equation
  \ali{
    (*)& & h(x) &= \text{smallest[$C - h(S_x)$]} \hspace{4cm}
  }
  for each $x \in J$, where $S_x$ is the section of $J$ by $x$.

  \emph{Proof.}
  \eparts{
  \item If $h$ and $k$ map sections of $J$, or all of $J$, into $C$ and satisfy $(*)$ for all $x$ in their respective domains, show that $h(x) = k(x)$ for all $x$ in both domains.
  \item If there exists a function $h: S_\a \to C$ satisfying $(*)$, show that there exists a function $k : S_\a \cup \braces{\a} \to C$ satisfying $(*)$.
  \item If $K \ss J$ and for all $\a \in K$ there exists a function $h_\a : S_\a \to C$ satisfying $(*)$, show that there exists a function
    \gath{
      k : \bigcup_{\a \in K} S_\a \to C
    }
    satisfying $(*)$.
  \item Show by transfinite induction that for every $\b \in J$, there exists a function $h_\b : S_\b \to C$ satisfying $(*)$.
    [Hint: If $\b$ has an immediate predecessor $\a$, then $S_\b = S_\a \cup \braces{\a}$.
      If not, $S_\b$ is the union of all $S_\a$ with $\a < \b$.]
  \item Prove the theorem.
  }
}
\sol{
  The following lemma is proof by transfinite induction, which is more straightforward than having to frame everything in terms of inductive sets.
  Henceforth we use this whenever transfinite induction is required.
  \begin{lem}\label{lem:wellord:tind}
    (Proof by transfinite induction)
    Suppose that $J$ is a well-ordered set and $P(x)$ is a proposition with parameter $x$.
    Suppose also that if $P(x)$ is true for all $x \in S_\a$ (where $S_\a$ is a section of $J$), then $P(\a)$ is also true.
    Then $P(\b)$ is true for every $\b \in J$.
  \end{lem}
  \qproof{
    Let $J_0 = \braces{x \in J \where P(x)}$.
    We show that $J_0$ is inductive.
    So consider any $\a \in J$ and suppose that $S_\a \ss J_0$.
    Then, for any $x \in S_\a$ we have that $x \in J_0$ so that $P(x)$.
    It then follows that $P(\a)$ is also true since $x$ was arbitrary, and so $\a \in J_0$.
    Since $\a \in J$ was arbitrary, this shows that $J_0$ is inductive.
    It then follows from Exercise~10.7 that $J_0 = J$.
    So consider any $\b \in J$ so that also $\b \in J_0$ and hence $P(\b)$ is true.
    Since $\b$ was arbitrary, this shows the desired result.
  }

  \mainprob

  (a)
  \qproof{
    First suppose that the domains of $h$ and $k$ are sets $H$ and $K$ where each is either a section of $J$ or $J$ itself.
    Since this is the case, we can assume without loss of generality that $H \ss K$  and so $H$ is exactly the domain common to both $h$ and $k$.
    Now suppose that the hypothesis we are trying to prove is \emph{not} true so that there is an $x$ in both domains (i.e. $x \in H$) where $h(x) \neq k(x)$.
    We can also assume that $x$ is the smallest such element since $H \ss J$ and $J$ is well-ordered.
    It then clearly follows that $S_x \ss H$ is a section of $J$ and that $h(y) = k(y)$ for all $y \in S_x$.
    From this we clearly have that $h(S_x) = k(S_x)$.
    But then we have
    \gath{
      h(x) = \text{smallest[$C - h(S_x)$]} = \text{smallest[$C - k(S_x)$]} = k(x)
    }
    since both $h$ and $k$ satisfy $(*)$ and $x$ is in the domain of both.
    This contradicts the supposition that $h(x) \neq k(x)$ so that it must be that no such $x$ exists and hence $h$ and $k$ are the same in their common domain as desired.
  }

  (b)
  \qproof{
    Suppose that $h : S_\a \to C$ is such a function satisfying $(*)$.
    Now let $\bar{S}_\a = S_\a \cup \braces{\a}$ and we define $k : \bar{S}_\a \to C$ as follows.
    For any $x \in \bar{S}_\a$ set
    \gath{
      k(x) = \begin{cases}
        h(x) & x \in S_\a \\
        \text{smallest[$C - h(S_\a)$]} & x = \a \,.
      \end{cases}
    }
    We note that clearly $S_\a$ and $\braces{\a}$ are disjoint so that this is unambiguous.
    We also note that $h$ is not surjective onto $C$ since $S_\a$ is a section of $J$, and hence $C - h(S_\a) \neq \es$ and so has a smallest element since $C$ is well-ordered.

    Now we show that $k$ satisfies $(*)$.
    First, clearly $h(S_x) = k(S_x)$ for any $x \leq \a$ since $k(y) = h(y)$ by definition for any $y \in S_x \ss S_\a$.
    Now consider any $x \in \bar{S}_\a$.
    If $x = \a$ then by definition we have
    \gath{
      k(x) = \text{smallest[$C - h(S_\a)$]} = \text{smallest[$C - k(S_\a)$]} = \text{smallest[$C - k(S_x)$]} \,.
    }
    On the other hand, if $x \in S_\a$ then $x < \a$ so that
    \gath{
      k(x) = h(x) = \text{smallest[$C - h(S_x)$]} = \text{smallest[$C - k(S_x)$]}
    }
    since $h$ satisfies $(*)$.
    Therefore, since $x$ was arbitrary, this shows that $k$ also satisfies $(*)$.
  }

  (c)
  \qproof{
    Let
    \gath{
      k = \bigcup_{\a \in K} h_\a \,,
    }
    which we claim is the function we seek.

    First we show that $k$ is actually a function from $\bigcup_{\a \in K} S_\a$ to $C$.
    So consider any $x$ in the domain of $k$.
    Suppose that $(x, a)$ and $(x, b)$ are both in $k$ so that there are $\a$ and $\b$ in $K$ where $(x, a) \in h_\a$ and $(x, b) \in h_\b$.
    Since $h_\a$ and $h_\b$ both satisfy $(*)$, it follows from part~(a) that $a = h_\a(x) = h_\b(x) = b$ since clearly $x$ is in the domain of both.
    This shows that $k$ is indeed a function since $(x,a)$ and $(x,b)$ were arbitrary.
    Also clearly the domain of $k$ is $\bigcup_{\a \in K} S_\a$ since, for any $x \in \bigcup_{\a \in K} S_\a$, we have that there is an $\a \in K$ where $x \in S_\a$.
    Hence $x$ is in the domain of $h_\a$ and so in the domain of $k$.
    In the other direction, clearly if $x$ is in the domain of $k$ then it is in the domain of $h_\a$ for some $\a \in K$.
    Since this domain is $S_\a$, clearly $x \in \bigcup_{\a \in K} S_\a$.
    Lastly, obviously the range of $k$ can be $C$ since this is the range of every $h_\a$.

    Now we show that $k$ satisfies $(*)$.
    So consider any $x \in \bigcup_{\a \in K} S_\a$ so that $x \in S_\a$ for some $\a \in K$.
    Clearly we have that $k(y) = h_\a(y)$ for every $y \in S_\a$ since $h_\a \ss k$.
    It then immediately follows that $k(x) = h(x)$ and $k(S_x) = h_\a(S_x)$ since $S_x \ss S_\a$.
    Then, since $h_\a$ satisfies $(*)$, we have
    \gath{
      k(x) = h_\a(x) = \text{smallest[$C - h_\a(S_x)$]} = \text{smallest[$C - k(S_x)$]} \,.
    }
    Since $x$ was arbitrary, this shows that $k$ satisfies $(*)$ as desired.
  }

  (d)
  \qproof{
    Consider any $\b \in J$ and suppose that, for every $x \in S_\b$, there is a function $h_x : S_x \to C$ satisfying $(*)$.
    Now, if $\b$ has an immediate predecessor $\a$ then we claim that $S_\b = S_\a \cup \braces{\a}$.
    First if $x \in S_\b$ then $x < \b$ so that $x \leq \a$ since $\a$ is the immediate predecessor of $\b$.
    If $x < \a$ then $x \in S_\a$ and if $x = \a$ then $x \in \braces{\a}$.
    Hence in either case we have that $x \in S_\a \cup \braces{\a}$.
    Now suppose that $x \in S_\a \cup \braces{\a}$.
    If $x \in S_\a$ then $x < \a < \b$ so that $s \in S_\b$.
    On the other hand if $x \in \braces{\a}$ then $x = \a < \b$ so that again $x \in S_\b$.
    Thus we have shown that $S_\b \ss S_\a \cup \braces{\a}$ and $S_\a \cup \braces{\a} \ss S_\b$ so that $S_\b = S_\a \cup \braces{\a}$.
    Since $\a \in S_\b$ it follows that there is an $h_\a : S_\a \to C$ that satisfies $(*)$.
    Then, by part~(b), we have that there is an $h_\b : S_\b = S_\a \cup \braces{\a} \to C$ that also satisfies $(*)$.

    If $\b$ does not have an immediate predecessor then we claim that $S_\b = \bigcup_{\g < \b} S_\g$.
    So consider any $x \in S_\b$ so that $x < \b$.
    Since $x$ cannot be the immediate predecessor of $\b$, there must be an $\a$ where $x < \a < \b$.
    Then $x \in S_\a$ so that, since $\a < \b$, clearly $x \in \bigcup_{\g < \b} S_\g$.
    Now suppose that $x \in \bigcup_{\g < \b} S_\g$ so that there is an $\a < \b$ where $x \in S_\a$.
    Then clearly $x < \a < \b$ so that also $x \in S_\b$.
    Thus we have shown that $S_\b \ss \bigcup_{\g < \b} S_\g$ and $\bigcup_{\g < \b} S_\g \ss S_\b$ so that $S_\b = \bigcup_{\g < \b} S_\g$.
    Now, clearly $S_\b$ is a subset of $J$ where there is an $h_x : S_x \to C$ satisfying $(*)$ for every $x \in S_\b$.
    Then it follows from what was shown in part~(c) that there is a function $h_\b$ from $\bigcup_{\g < S_\b} S_\g = \bigcup_{\g < \b} S_\g = S_\b$ to $C$ that satisfies $(*)$.

    Therefore, in either case, we have shown that there is an $h_\b : S_\b \to C$ that satisfies $(*)$.
    The desired result then follows by transfinite induction.
  }

  (e)
  \qproof{
    First suppose that $J$ has no largest element.
    Then we claim that $J = \bigcup_{\a \in J} S_\a$.
    For any $x \in J$ there must be a $y \in J$ where $x < y$ since $x$ cannot be the greatest element of $J$.
    Hence $x \in S_y$ so that also clearly $\bigcup_{\a \in J} S_\a$.
    Then, for any $x \in \bigcup_{\a \in J} S_\a$, there is an $\a \in J$ where $x \in S_\a$.
    Clearly $S_\a \ss J$ so that $x \in J$ also.
    Hence $J \ss \bigcup_{\a \in J} S_\a$ and $\bigcup_{\a \in J} S_\a \ss J$ so that $J = \bigcup_{\a \in J} S_\a$.
    Since we know from part~(d) that there is an $h_\a : S_\a \to C$ that satisfies $(*)$ for every $\a \in J$, it follows from part~(c) that there is a function $h$ from $\bigcup_{\a \in J} S_\a = J$ to $C$ that satisfies $(*)$.

    If $J$ does have a largest element $\b$ then clearly $J = S_\b \cup \braces{\b}$.
    Since we know that there is an $h_\b : S_\b \to C$ that satisfies $(*)$ by part~(d), it follows from part~(b) that there is a function $h$ from $S_\b \cup \braces{\b} = J$ to $C$ that satisfies $(*)$.
    Hence the desired function $h$ exists in both cases.
    part~(a) also clearly shows that this function is unique.
  }
}

\exercise{11}{
  Let $A$ and $B$ be two sets.
  Using the well-ordering theorem, prove that either they have the same cardinality, or one has cardinality greater than the other.
  [Hint: If there is no surjection $f: A \to B$, apply the preceding exercise.]
}
\sol{
  \begin{lem}\label{lem:wellord:injsur}
    For well-ordered sets $A \neq \es$ and $B$ there is an injection from $A$ to $B$ if and only if there is a surjection from $B$ to $A$.
  \end{lem}
  \qproof{
    $(\imp)$ Suppose that there is an injection $f: A \to B$ and $A \neq \es$.
    Then there is an $a \in A$.
    We then construct a surjection $g : B \to A$ as follows.
    For any $y \in B$ if $b \in f(A)$ then there is a unique $x \in A$ where $y = f(x)$.
    It is unique since, if $x$ and $x'$ are in $A$ where $f(x) = y = f(x')$, then $x = x'$ since $f$ is injective.
    So in this case set $g(y) = x$.
    If $b \notin f(A)$, then set $g(y) = a$.
    Clearly $g$ is a function from $B$ to $A$.
    To show that $g$ is surjective, consider any $x \in A$ and let $y = f(x)$, which is clearly an element of $B$.
    Then, since obviously $y \in f(A)$ and $x$ is the unique $x \in A$ such that $y = f(x)$, we have that $g(y) = x$ by definition.
    This shows that $g$ is surjective since $x$ was arbitrary.

    $(\pmi)$ Now suppose that $g: B \to A$ is surjective.
    We then construct an injection $f: A \to B$ as follows.
    For any $x \in A$ we have that the set $B_x = \braces{y \in B \where g(y) = x}$ is nonempty since $g$ is surjective.
    Hence $B_x$ has a unique smallest element $y$ since it is a nonempty subset of $B$ and $B$ is well-ordered.
    So simply set $f(x) = y$.
    Clearly $f$ is a function from $A$ to $B$.
    To show that $f$ is injective, consider $x,x' \in A$ where $x \neq x'$.
    Then clearly the sets $B_x$ and $B_{x'}$ have to be disjoint for otherwise there would be a $y \in B$ where $g(y) = x$ and $g(y) = x'$, which is impossible if $x \neq x'$ since $g$ is a function.
    Hence, since $f(x)$ and $f(x')$ are defined to be the smallest elements of $B_x$ and $B_{x'}$, respectively, we have $f(x) \neq f(x')$.
    This shows that $f$ is injective since $x$ and $x'$ were arbitrary.
  }

  \mainprob
  \qproof{
    First suppose that $A$ and $B$ are each well-ordered, which follows from the well-ordering theorem.
    Also suppose that $A$ and $B$ do \emph{not} have the same cardinality so that it suffices to show that either $B$ has greater cardinality than $A$ or vice versa.
    If $A = \es$ then it cannot be that $B = \es$ as well since then they would have the same cardinality ($\es$ would be a trivial bijection between them).
    Hence $B \neq \es$ so that clearly $B$ has greater cardinality than $A$.
    Thus in what follows assume that $A \neq \es$.

    Suppose that there is an injection from $A$ to $B$.
    Then there cannot be an injection from $B$ to $A$ since, if there were, then $A$ and $B$ would have the same cardinality by the Cantor-Schroeder-Bernstein Theorem (shown in Exercise~7.6 part~(b)).
    Thus $B$ has greater cardinality than $A$ by definition.

    On the other hand, if there is no injection from $A$ to $B$ then there is no surjection from $B$ to $A$ by Lemma~\ref{lem:wellord:injsur} since they are both well-ordered and $A \neq \es$.
    It then clearly follows that no section of $B$ can be a surjection onto $A$ since then any extension of such a function to all of $B$ would also be a surjection onto $A$.
    From this we have by Exercise~10.10 that there is a unique function $h : B \to A$ with the property that
    \gath{
      h(x) = \text{smallest[$A - h(S_x)$]} \,,
    }
    where of course $S_x$ is the section of $B$ by $x$.

    We claim that $h$ is injective.
    So consider any $y$ and $y'$ in $B$ where $y \neq y'$.
    Without loss of generality we can assume that $y < y'$ (by the well-ordering on $B$).
    It then follows that $y \in S_{y'}$ so that clearly $h(y) \in h(S_{y'})$.
    However, we have that $h(y')$ is the smallest element of $A - h(S_{y'})$ so that obviously $h(y') \notin h(S_{y'})$.
    Hence it must be that $h(y) \neq h(y')$, which shows that $h$ is injective since $y$ and $y'$ were arbitrary.

    Therefore there is an injection from $B$ to $A$ but none from $A$ to $B$ so that $A$ has greater cardinality than $B$ by definition.
    This shows the desired result since these cases are exhaustive.
  }
}
