% Set section label to text
\renewcommand\thesubsection{TG}
\def\secl{\thesubsection}

\subsection{Supplementary Exercises: Topological Groups}

For this section, recall that a group is a set $G$ together with some operation $\cdot$ that satisfies the following properties, called the group axioms:
\lparts{
\item \emph{(Closure)} For all $a,b \in G$, the result $a \cdot b$ is also in $G$.
\item \emph{(Associativity)} $(a \cdot b) \cdot c = a \cdot (b \cdot c)$ for all $a,b,c \in G$.
\item \emph{(Identity Element)} There is an element $e \in G$ such that $a \cdot e = e \cdot a = a$ for all $a \in G$, which is called an identity element.
\item \emph{(Inverse Element)} For every $a \in G$, there is an element $b \in G$ such that $a \cdot b = b \cdot a = e$, where $e$ is the identity element.
  This $b$ is called an inverse element of $a$.
}
It is easy to show directly from these axioms that the identity element of a group is unique, and so we refer to \emph{the} identity element.
Similarly, if $a \in G$, then its inverse element is also unique, and is usually denoted by $\inv{a}$.

\exercise{1}{
  Let $H$ denote a group that is also a topological space satisfying the $T_1$ axiom.
  Show that $H$ is a topological group if and only if the map of $H \times H$ into $H$ sending $x \times y$ to $x \cdot \inv{y}$ is continuous.
}
\sol{
  \begin{lem}\label{lem:setg:idinv}
    In any group $G$ the inverse element of an inverse element is the element itself, i.e. $\inv{\parens{\inv{x}}} = x$ for any $x \in G$.
  \end{lem}
  \qproof{
    Consider any $x$ in the group $G$ with operation $\cdot$ and identity element $e$, and let $y = \inv{\parens{\inv{x}}}$.
    Then we have of course that $y \cdot \inv{x} = e$ since $y$ is the inverse of $\inv{x}$.
    Since we of course also have $x \cdot \inv{x} = e$ since $\inv{x}$ is the inverse of $x$, it has to be that $y = x$ because the inverse of $\inv{x}$ must be unique.
    Therefore of course $\inv{\parens{\inv{x}}} = y = x$ as desired.
  }

  \mainprob
  
  \qproof{
    $(\imp)$ First suppose that $H$ is a topological group.
    Then $f: H \to H$ defined by $f(x) = \inv{x}$ and $g: H \times H \to H$ defined by $g(\pptt{}) = x \cdot y$ are both continuous.
    Define $h: H \times H \to H$ by 
    \gath{
      h = g \circ (i_H \times f) \,,
    }
    where of course $i_H$ is the identity function on $H$, and we have defined the function $i_H \times f : H \times H \to H \times H$ as in Exercise~18.10.

    Now, we know that both $i_H$ and $f$ are continuous so that $i_H \times f$ is continuous by Exercise~18.10.
    It then follows that $g \circ (i_H \times f) = h$ is continuous by Theorem~18.2 part~(c) since $g$ is also continuous.
    Now, for any $\pptt{} \in H \times H$, we have
    \ali{
      h(\pptt{}) &= \parens{g \circ (i_H \times f)}(\pptt{}) \\
      &= g((i_H \times f)(\pptt{})) \\
      &= g(i_H(x) \times f(y)) \\
      &= g(x \times \inv{y}) \\
      &= x \cdot \inv{y} \,.
    }
    Since we have shown that $h$ is continuous, this shows the desired result.

    $(\pmi)$ Now again define $h: H \times H \to H$ by $h(\pptt{}) = x \cdot \inv{y}$, and suppose that $h$ is continuous.
    Then $h$ is continuous in each variable separately by Exercise~18.11.
    So, if we let $e$ be the unique identity element of $H$, then we have that
    \gath{
      h(e \times x) = e \cdot \inv{x} = \inv{x}
    }
    is continuous for any $x$.
    Similarly, for any $x,y \in H$, we have that $\inv{\parens{\inv{y}}} = y$ by Lemma~\ref{lem:setg:idinv} so that
    \gath{
      h(x \times \inv{y}) = x \cdot \inv{\parens{\inv{y}}} = x \cdot y
    }
    must also be continuous.
  }
}

\def\gl{\mathrm{GL}}
\def\adj{\mathrm{adj}}
\exercise{2}{
  Show that the following are topological groups:
  \eparts{
  \item $(\ints, +)$
  \item $(\reals, +)$
  \item $(\realsp, \cdot)$
  \item $(S^1, \cdot)$, where we take $S^1$ to be the space of all complex numbers $z$ for which $\abs{z} = 1$.
  \item The \emph{general linear group} $\gl(n)$, under the operation of matrix multiplication.
    ($\gl(n)$ is the set of all nonsingular $n$ by $n$ matrices, topologized by considering it as a subset of euclidean space of dimension $n^2$ in the obvious way.)
  }
}
\sol{
  \begin{lem} \label{lem:setg:t1disc}
    Any discrete topology satisfies the $T_1$ axiom.
  \end{lem}
  \qproof{
    Suppose that $X$ is a set with the discrete topology and $C$ is a finite point set.
    Then $X - C$ is clearly still a subset of $X$ and so is open since $X$ is discrete.
    This shows by definition that $C$ is closed.
    In fact by this same argument \emph{any} subset of $X$ is both open and closed.
  }
    
  \begin{lem} \label{lem:setg:pdisc}
    If $Y$ and $Y$ are sets both with discrete topologies, then $X \times Y$ is also the discrete topology.
  \end{lem}
  \qproof{
    It suffices to show that the subset of $X \times Y$ containing a single arbitrary element is open, since clearly any other subset is the union of such single-element open subsets and is therefore also open by the definition of a topology.
    So consider any $(x,y) \in X \times Y$ and the subset $\braces{(x,y)} \ss X \times Y$.
    Then clearly $\braces{(x, y)} = \braces{x} \times \braces{y}$, which is a basis element of $X \times Y$ and therefore open by the definition of a product topology since both $\braces{x}$ and $\braces{y}$ are open in $X$ and $Y$, respectively, since they are discrete.
  }

  \begin{lem} \label{lem:setg:cdisc}
    If $X$ and $Y$ are topological spaces and $X$ has the discrete topology then any function $f: X \to Y$ is continuous.
  \end{lem}
  \qproof{
    This is fairly obvious since, for any open subset $V$ of $Y$, of course $\ivf(V)$ is a subset of $X$ and so is open since $X$ is discrete.
  }

  \mainprob
  
  (a)
  \qproof{
    First we must show that $(\ints, +)$ is even a group.
    Clearly $a+b$ is an integer when $a$ and $b$ are so that the closure axiom is satisfied.
    Also, we know that integer addition is associative.
    We clearly have that $0 \in \ints$ and that $a + 0 = a$ for any $a \in \ints$ so that $0$ is the identity element of $(\ints, +)$.
    Lastly, for any $a \in \ints$, we have that $-a \in \ints$ and that $a + (-a) = a - a = 0$ so that clearly $-a$ is the inverse of $a$.
    This shows that $(\ints, +)$ is in fact a group.

    To show that it is a topological group, we first note that $\ints$ clearly has the discrete topology when considered both an order topology or as a subspace of $\reals$, for similar reason as discussed in Example~3 of \S 14.
    Thus $\ints$ satisfies the $T_1$ axiom by Lemma~\ref{lem:setg:t1disc} since it is discrete.
    Also $X \times X$ is the discrete topology by Lemma~\ref{lem:setg:pdisc}.
    Thus the function $f$ defined by $f(x \times y) = x + \inv{y} = x + (-y) = x - y$ is a function from $X \times X$ to $X$, so that it follows that $f$ is continuous by Lemma~\ref{lem:setg:cdisc} since $X \times X$ is discrete.
    Hence $(\ints, +)$ is a topological group by Exercise~\secl.1.
  }

  (b)
  \qproof{
    Similarly to part~(a), clearly $(\reals,+)$ is a group with identity element $0$ and inverse element $-x$ for any $x \in \reals$.
    However, this time the topology is no longer discrete.
    Of course we know that $\reals$ satisfies the $T_1$ axiom.
    Now consider the function $f(\pptt{}) = x + \inv{y} = x + (-y) = x - y$ for any $x,y \in \reals$.
    Consider also any basis element $B = (a,b) \in \reals$, where here we are of course using the order topology basis.
    Then we clearly have
    \ali{
      \ivf(B) &= \braces{\pptt{} \where f(\pptt{}) \in (a,b)} = \braces{\pptt{} \where a < f(\pptt{}) < b} \\
      &= \braces{\pptt{} \where a < x-y < b} = \braces{\pptt{} \where a-x < -y < b-x} \\
      &= \braces{\pptt{} \where x-a > y > x-b} \,.
    }
    Clearly this is the region in $\reals^2$ between the lines $y=x-b$ and $y=x-a$, which is obviously an open set in $\reals^2$.
    This shows that $f$ is continuous since $B$ was an arbitrary basis element, so that $(\reals,+)$ is a topological group by Exercise~\secl.1.
  }

  (c)
  \qproof{
    First, clearly $\realsp$ satisfies the $T_1$ axiom since $\reals$ does.
    Next we note that for any $x,y \in \realsp$ we have that $x \cdot y$ is also positive so that $x \cdot y \in \realsp$ as well, which shows the closure property of a group.
    Also, clearly $1 \in \realsp$ is the identity element of multiplication, and the inverse element is $\inv{x} = 1/x$ for any $x \in \realsp$, noting that this is defined since $x > 0$, and that $1/x > 0$ so that $\inv{x} = 1/x \in \realsp$.
    Lastly, we know that multiplication is associative on the reals (and therefore also on $\realsp$), which completes the check that $(\realsp,\cdot)$ is in fact a group.

    As before, define the function $f: \realsp \times \realsp \to \realsp$ by $f(\pptt{}) = x \cdot \inv{y} = x \cdot 1/y = x/y$.
    Consider the order topology basis of $\reals$ and consider any basis element $B$ of the subspace $\realsp$ so that $B = \realsp \cap (a,b)$ for some $a,b \in \reals$ where $a<b$ by Lemma~16.1.
    Now, if $a \leq 0$ then clearly $B = (0, b)$, and we have that
    \ali{
      \ivf(B) &= \braces{\pptt{} \where f(\pptt{}) \in B} = \braces{\pptt{} \where 0 < f(\pptt{}) < b} \\
      &= \braces{\pptt{} \where 0 < x/y < b} = \braces{\pptt{} \where 0 < x < by} \\
      &= \braces{\pptt{} \where 0 < x/b < y} \,,
    }
    noting that $0 < b$ so that $x/b$ is defined.
    Obviously this is the region in $\realsp \times \realsp$ ($\realsp \times \realsp$ being the upper right quadrant of $\reals^2$ that does not include either axis) above the line $y = x/b$, which is easy to show is open in $\realsp \times \realsp$.

    On the other hand, if $a > 0$ than $B = (a,b)$ so that
    \ali{
      \ivf(B) &= \braces{\pptt{} \where f(\pptt{}) \in B} = \braces{\pptt{} \where a < f(\pptt{}) < b} \\
      &= \braces{\pptt{} \where a < x/y < b} = \braces{\pptt{} \where ay < x < by} \\
      &= \braces{\pptt{} \where ay < x \land x < by} = \braces{\pptt{} \where y < x/a \land x/b < y} \\
      &= \braces{\pptt{} \where x/b < y < x/a} \,,
    }
    which is clearly the region of $\realsp \times \realsp$ between the lines $y = x/b$ and $y = x/a$.
    It is easy to see that again this is an open subset of $\realsp \times \realsp$, which shows that $f$ is continuous either way.
    This in turn proves that $(\realsp,\cdot)$ is a topological space, again by Exercise~\secl.1.
  }

  (d)
  \qproof{
    Topologies on the complex plane $\cpx$ have not really been discussed, but $\cpx$ is usually defined as $\reals \times \reals$ having the usual product topology.
    Then of course $S^1$ is the unit circle in $\cpx$.
    We know that $\reals$ is Hausdorff so that $\cpx = \reals \times \reals$ is as well by Theorem~17.11.
    Then, again by Theorem~17.11, $S^1$ is Hausdorff since it is a subspace of $\cpx$, and so it also satisfies the $T_1$ axiom.
    
    While perhaps not immediately obvious, it is easy to show that $S^1$ is closed under multiplication.
    If $z,w \in S^1$ then $\abs{z} = \abs{w} = 1$ so that $\abs{z \cdot w} = \abs{z} \cdot \abs{w} = 1 \cdot 1 = 1$ by familiar rules of complex analysis so that $z \cdot w \in S^1$ as well.
    Clearly $1 \in S^1$ is the identity element where the inverse element of $z \in S^1$ is $1/z$, noting that $\abs{1/z} = 1/\abs{z} = 1/1 = 1$ since $z \in S^1$, and so $1/z \in S^1$.
    We also note that $\abs{0} = 0$, and hence $0 \notin S^1$ so that the inverse $1/z$ is always defined.
    Lastly, we know that multiplication is associative within $\cpx$ and therefore also within $S^1$.
    This shows that $(S^1,\cdot)$ satisfies all of the group axioms.

    To rigorously show that $S^1$ is a topological group is actually quite tedious so we shall omit some details.
    Suppose that $U$ is open in $S^1$ and that $z \times w \in \ivf(U)$ so that $f(z \times w) \in U$.
    Now, clearly the unit circle in $\cpx = \reals \times \reals$ is the set $S^1 = \braces{e^{i\th} \where \th \in \reals}$ so that we can express $z = e^{i\th}$ and $w = e^{i\phi}$ for some $\th,\phi \in \reals$.
    We then have that
    \gath{
      f(z \times w) = z/w = e^{i\th}/e^{i\phi} = e^{i\th}e^{-i\phi} = e^{i(\th-\phi)} \in U \,.
    }
    While tedious to show rigorously, it follows from the fact that $U$ is open in $S^1$ that there is an $\e > 0$ where $f(z \times w) \in A_{\th-\phi,\e} \ss U$, where we define
    \gath{
      A_{\a,\e} = \braces{e^{i\g} \where \a-\e < \g < \a+\e} \,,
    }
    noting that of course $A_{\a,\e} \ss S^1$.
    Now consider $A_{\th,\e/2}$ and $A_{\phi,\e/2}$, which are both clearly open in $S^1$ and noting that clearly $z \in A_{\th,\e/2}$ and $w \in A_{\phi,\e/2}$.
    For any $z' = e^{i\th'} \in A_{\th,\e/2}$ and $w' = e^{i\phi'} \in A_{\phi,\e/2}$ we then have that
    \ali{
      \th - \e/2 &< \th' < \th + \e/2 &
      \phi - \e/2 &< \phi' < \phi + \e/2 \,.
    }
    Hence
    \gath{
      -\phi + \e/2 > -\phi' > -\phi - \e/2 \\
      \th' - \phi + \e/2 > \th' - \phi' > \th' -\phi - \e/2 \\
      \th + \e/2 - \phi + \e/2 > \th' - \phi + \e/2 > \th' - \phi' >  \th' -\phi - \e/2 > \th - \e/2 -\phi - \e/2 \\
      (\th - \phi) + \e > \th' - \phi' > (\th - \phi) - \e
    }
    so that $f(z' \times w') = e^{i(\th' - \phi')} \in A_{\th-\phi,\e} \ss U$.
    Thus $z' \times w' \in \ivf(U)$ so that $z \times w \in A_{\th,\e/2} \times  A_{\phi,\e/2} \ss \ivf(U)$ since $z'$ and $w'$ were arbitrary.
    We also have that $A_{\th,\e/2} \times  A_{\phi,\e/2}$ is open in $S^1 \times S^1$ since both $A_{\th,\e/2}$ and $A_{\phi,\e/2}$ are open in $S^1$.
    Since $z \times w$ was an arbitrary element of $\ivf(U)$, this shows that $\ivf(U)$ is open in $S^1 \times S^1$, which in turn shows that $f$ is continuous by definition.
    Thus by Exercise~\secl.1 we have that $S^1$ is a topological group.
  }

  (e)
  \qproof{
    First, from linear algebra we know that the matrix product of two nonsingular $n$ by $n$ matrices is another nonsingular $n$ by $n$ matrix, so that $\gl(n)$ is closed under matrix multiplication.
    Clearly the identity matrix is the identity element of $\gl(n)$, while the inverse matrix $\inv{A}$ is the inverse element of the matrix $A \in \gl(n)$, noting that this inverse matrix exists since $A$ is nonsingular.
    Lastly, we know that matrix multiplication is associative, which suffices to show that $(\gl(n), \cdot)$ is a group.

    To show that it is a topological group takes more work.
    To begin, we note that of course $\reals^{n^2}$ is Hausdorff and so satisfies the $T_1$ axiom.
    Thus so does $\gl(n)$ since $\reals^{n^2}$ gives it its topology.
    Next, we denote a vector in $\reals^n$ by $\vx_n = \cpfin{x}{n}$, using the subscript on the vector itself to indicate its dimension.
    We show that the function $s_n: \reals^n \to \reals$ defined by
    \gath{
      s_n(\vx_n) = \sum_{i=1}^n x_i
    }
    is continuous for all $n \in \pints$, which we show by induction.
    First, for $n = 1$, we clearly have that $s_n$ is simply the identity function from $\reals$ to $\reals$, which is clearly continuous.
    Now suppose that $s_n$ is continuous.
    Define $g : \reals^{n+1} \to \reals^n$ by
    \gath{
      g(\vx_{n+1}) = \pi_1(\vx_{n+1}) \times \cdots \times \pi_n(\vx_{n+1}) = \cpfin{x}{n} = \vx_n \,,
    }
    which is continuous by Theorem~19.6 since we know that each $\pi_i$ is continuous.

    Then also $s_n \circ g$ is continuous by Theorem~18.2 part~(c).
    It then follows that the function $h : \reals^{n+1} \to \reals^2$ defined by $h(\vx_{n+1}) = (s_n \circ g)(\vx_{n+1}) \times \pi_{n+1}(\vx_{n+1})$ is continuous by Theorem~18.4 since both $s_n \circ g$ and $\pi_{n+1}$ are continuous.
    Lastly we then have that $k : \reals^{n+1} \to \reals$ defined by $+ \circ h$ is continuous by Theorem~18.2 part~(c), where of course $+$ is the usual addition operation from $\reals^2$ to $\reals$, which we showed is continuous in Exercise~21.12.
    Now we claim that $k = s_{n+1}$.
    For any $\vx_{n+1} \in \reals^{n+1}$ we have
    \ali{
      k(\vx_{n+1}) &= (+ \circ h)(\vx_{n+1}) = +(h(\vx_{n+1})) = +((s_n \circ g)(\vx_{n+1}) \times \pi_{n+1}(\vx_{n+1})) \\
      &= +(s_n(g(\vx_{n+1})), x_{n+1}) = +(s_n(\vx_n), x_{n+1}) = s_n(\vx_n) + x_{n+1} \\
      &= \sum_{i=1}^n x_i + x_{n+1} = \sum_{i=1}^{n+1} x_i \\
      &= s_{n+1}(\vx_{n+1}) \,.
    }
    This completes the induction since we have shown that $k = s_{n+1}$ is continuous.

    Next we show that the function $p_{ij} : \reals^n \times \reals^n \to \reals$ defined by $p_{ij}(\vx_n \times \vy_n) = x_i y_j$ is continuous, where of course $ij \in \intsfin{n}$.
    Define the function $g_{ij}: \reals^n \times \reals^n \to \reals^2$ by $g_{ij} = \pi_i \times \pi_j$ as in Exercise~18.10, which we know is continuous by that exercise since the coordinate functions are continuous.
    Then the function $\cdot \circ g_{ij}$ from $\reals^n \times \reals^n$ to $\reals$ is also continuous by Theorem~18.2 part~(c), where of course $\cdot$ is the normal multiplication operation from $\reals^2$ to $\reals$, which we know is continuous from Exercise~12.12.
    However, for any $\vx_n,\vy_n \in \reals^n$, we have
    \gath{
      (\cdot \circ g_{ij})(\vx_n \times \vy_n) = \cdot(g_{ij}(\vx_n \times \vy_n)) = \cdot(\pi_i(\vx_n) \times \pi_j(\vy_n))
      = \cdot(x_i \times y_j) = x_i \cdot y_j = p_{ij}(\vx_n \times \vy_n)
    }
    so that $\cdot \circ g_{ij} = p_{ij}$ is continuous, which shows the desired result.

    Now, by definition, each matrix component of the resultant matrix in matrix multiplication on $\gl(n)$ is a sum of products, where each product involves a term from each of the matrices, and the sum has $n$ terms going across a row of the first matrix and a column of the second.
    Thus each component is a composition of the sum function $s_n : \reals^n \to \reals$ with a mapping $f$ from $\reals^{n^2} \times \reals^{n^2} \to \reals^n$, where each element in $\reals^n$ of this mapping is a product function $p_{ij}$.
    Since have shown above that each $p_{ij}$ is continuous, it follows from Theorem~19.6 that the mapping $f$ is also continuous.
    Hence the composition $s_n \circ f$, i.e. the matrix component function, is also continuous by Theorem~18.2 part~(c) since we have also shown above that $s_n$ is continuous.
    Since each component function is continuous, it again follows from Theorem~19.6 that the overall matrix multiplication mapping from $\reals^{n^2} \times \reals^{n^2} \to \reals^{n^2}$ is continuous.

    Regarding the inverse element function, we recall from linear algebra that in the inverse of a matrix $A \in \gl(n)$ is
    \gath{
      \inv{A} = \frac{1}{\abs{A}} \adj(A) \,,
    }
    where $\adj(A)$ is the adjugate matrix of $A$ and $\abs{A}$ is the determinant of $A$, noting that this is nonzero since $A$ is nonsingular.
    Now, the determinant is a sum of products so that the function $g: \gl(N) \to \reals$ defined by $g(A) = \abs{A}$ is continuous by the same arguments as above for matrix multiplication.
    Likewise each element of the adjugate matrix is a sum of products as well so that the function $f_{ij} : \gl(A) \to \reals$ defined by $f_{ij}(A) = \adj(A)_{ij}$, i.e. the $i$th row and $j$th column component of the adjugate matrix, is also continuous.

    Then clearly the corresponding component of the inverse matrix is the function $h_{ij} : \gl(A) \to \reals$ defined by $h_{ij}(A) = f_{ij}(A)/g(A)$.
    Since both $f_{ij}$ and $g$ are continuous (and again noting that $g$ is always nonzero), then their quotient $h_{ij}$ is also continuous by Exercise~21.12.
    Hence, since each component $h_{ij}$ of the inverse matrix is continuous, it follows that the inversion operation as a whole is continuous by Theorem~19.6 as above, considering the matrices as elements of $\reals^{n^2}$.
    Since both multiplication and inversion are continuous, this shows that $\gl(n)$ is a topological group by definition.
  }
}

\def\clH{\closure{H}}
\exercise{3}{
  Let $H$ be a subspace of $G$.
  Show that if $H$ is also a subgroup of $G$, then both $H$ and $\clH$ are topological groups.
}
\sol{
  \begin{lem} \label{lem:setg:t1sub}
    Any subspace of a space satisfying the $T_1$ axiom also satisfies the $T_1$ axiom.
  \end{lem}
  \qproof{
    Suppose that $X$ is a subspace of $Y$ and consider two distinct points $x,y \in X$.
    Then there is a neighborhood $U$ of $x$ in $Y$ that does not contain $y$ by Exercise~17.15.
    It then follows that $U \cap X$ is a neighborhood of $x$ in $X$ that does not contain $y$ since $y \notin U$.
    A similar argument shows that there is a neighborhood of $y$ in $X$ that does not contain $x$.
    Hence $X$ also satisfies the $T_1$ axiom, again by Exercise~17.15.
  }

  \mainprob
  \qproof{
    Presumably here $G$ is a topological group.
    Hence $G$ satisfies the $T_1$ axiom so that $H$ and $\clH$ do as well by Lemma~\ref{lem:setg:t1sub} since they are subspaces of $G$.

    We first show that $\clH$ is a subgroup of $G$, noting that of course $\clH$ is nonempty since $H$ must be (as it is a subgroup) and $H \ss \clH$.
    Let $f : G \times G \to G$ be the operation defined by $f(\pptt{}) = x \cdot \inv{y}$ for $x,y \in G$.
    It is a well known theorem of group theory that $\clH$ is a subgroup of $G$ if and only if $x \cdot \inv{y} \in \clH$ for any $x,y \in \clH$, which is to say that $f(\clH \times \clH) \ss \clH$.
    This is exactly what we intend to show.
    We have the following deductions:
    \begin{itemize}
    \item As subsets of $G \times G$, $\clH \times \clH = \closure{H \times H}$ by Exercise~17.9 since $H \ss G$.
    \item Since $f$ is continuous on $G \times G$, it follows that $f(\closure{H \times H}) \ss \closure{f(H \times H)}$ by Theorem~18.1.
    \item Since $H$ is a subgroup we have that $x \cdot \inv{y} \in H$ for every $x,y \in H$, which is to say that $f(H \times H) \ss H$.
      It then follows from Exercise~17.6 part~(a) that $\closure{f(H \times H)} \ss \clH$.
    \end{itemize}
    Putting these all together, we can conclude that
    \gath{
      f(\clH \times \clH) = f(\closure{H \times H}) \ss \closure{f(H \times H)} \ss \clH \,,
    }
    which shows the desired result that $\clH$ is a subgroup of $G$.
    
    Next, obviously $H$ and $\clH$ are both groups by the definition of a subgroup.
    The operation of $H$ (or $\clH$) is of course the operation of $G$ with its domain restricted to $H \times H$  (or $\clH \times \clH$).
    The continuity of this operation on $H$  (or $\clH$) follows from Theorem~18.2 part~(d) since it is continuous on $G$ since $G$ is a topological group.
    Likewise the inversion function on $H$  (or $\clH$) is a restriction of the inversion function on $G$, and so is also continuous for the same reason.
    Hence $H$ and $\clH$ are topological groups by definition.
  }
}

\def\fa{f_\a}
\def\ga{g_\a}
\def\fia{f_\a'}
\def\gia{g_\a'}
\exercise{4}{
  Let $\a$ be an element of $G$.
  Show that the maps $\fa, \ga : G \to G$ defined by
  \gath{
    \fa(x) = \a \cdot x \condgap \text{and} \condgap \ga(x) = x \cdot \a
  }
  are homeomorphisms of $G$.
  Conclude that $G$ is a \emph{homogeneous space}.
  (This means that for every pair $x$, $y$ of points of $G$, there exists a homeomorphism of $G$ onto itself that carries $x$ to $y$.)
}
\sol{
  \qproof{
    Clearly $G$ is meant to be a topological group.
    Let $e$ be the identity element of $G$.
    Define the function $\fia : G \to G$ by $\fia(x) = \inv{\a} \cdot x$.
    For any any $x \in G$ we then have
    \gath{
      (\fia \circ \fa)(x) = \fia(\fa(x)) = \fia(\a \cdot x) = \inv{\a} \cdot (\a \cdot x) = (\inv{\a} \cdot \a) \cdot x
      = e \cdot x = x
    }
    and
    \gath{
      (\fa \circ \fia)(x) = \fa(\fia(x)) = \fa(\inv{\a} \cdot x) = \a \cdot (\inv{\a} \cdot x)
      = (\a \cdot \inv{\a}) \cdot x = e \cdot x = x \,.
    }
    This shows that both $\fia \circ \fa = i_G$ and $\fa \circ \fia = i_G$ so that $\fia$ is both a left inverse and a right inverse for $\fa$ (see Exercise~2.5).
    Hence $\fa$ is bijective and $\inv{\fa} = \fia$ by Exercise~2.5 part~(e).
    An analogous argument shows $\ga$ is bijective and that $\gia(x) = x \cdot \inv{\a}$ is its inverse function.
    
    To show that they are homeomorphisms, we note that the group operation is a continuous function since $G$ is a topological group.
    We then have that the operation is continuous in each variable separately by Exercise~18.11.
    From this it clearly follows that both $\fa$ and $\ga$ are continuous since $\fa(x) = \cdot(\a \times x)$ and $\ga(x) = \cdot (x \times \a)$.
    Similarly $\inv{\fa}$ and $\inv{\ga}$ are continuous since we have $\inv{\fa}(x) = \fia(x) = \cdot(\inv{\a} \times x)$ and $\inv{\ga}(x) = \gia(x) = \cdot(x \times \inv{\a})$.
    This shows that both $\fa$ and $\ga$ are homeomorphisms by definition.

    Now, to show that $G$ is a homogeneous space, consider any points $x_0,y_0 \in G$.
    Set $\a = y_0 \cdot \inv{x_0}$, noting that of course $\a \in G$.
    We then claim that $\fa$ as defined above is a homeomorphism that carries $x_0$ to $y_0$.
    Of course we already showed above that $\fa$ is a homeomorphism, so all that remains is to show that $\fa(x_0) = y_0$.
    To this end we have
    \gath{
      \fa(x_0) = \a \cdot x_0 = (y_0 \cdot \inv{x_0}) \cdot x_0 = y_0 \cdot (\inv{x_0} \cdot x_0) = y_0 \cdot e = y_0 \,,
    }
    which shows the desired result.
  }
}

\exercise{5}{
  Let $H$ be a subgroup of $G$.
  If $x \in G$, define $xH = \braces{x \cdot h \where h \in H}$; this set is called a \boldit{left coset} of $H$ in $G$.
  Let $G/H$ denote the collection of left cosets of $H$ in $G$; it is a partition of $G$.
  Give $G/H$ the quotient topology.
  \eparts{
  \item Show that if $\a \in G$, the map $f_\a$ of the preceding exercise induces a homeomorphism of $G/H$ carrying $xH$ to $(\a \cdot x)H$.
    Conclude that $G/H$ is a homogeneous space.
  \item Show that if $H$ is a closed set in the topology of $G$ then one-point sets are closed in $G/H$.
  \item Show that the quotient map $p: G \to G/H$ is open.
  \item Show that if $H$ is closed in the topology of $G$ and is a normal subgroup of $G$, then $G/H$ is a topological group.
  }
}
\sol{
  First we note that it is a well-known theorem of group theory that any subgroup of a group contains the identity element of the group, which is also the identity element of the subgroup.
  So, for what follows, let $e$ be the identity element of $G$ and $H$ above, from which it follows that $x \in xH$ for any $x \in G$ since we have $x = x \cdot e$ and $e \in H$.
  We also have that $H \in G/H$ since clearly $H = eH$.
  Also let $p : G \to G/H$ denote the quotient map corresponding to the quotient space.
      
  (a)
  \qproof{
    Now, for $\a \in G$, define the function $h_\a : G/H \to G/H$ by mapping the left coset $xH \in G/H$ to $f_\a(x) H = (\a \cdot x)H$.
    We note that if $xH = yH$ for $x,y \in G$ then of course $y \in yH = xH$ so that $y = x \cdot h$ for some $h \in H$.
    Then
    \gath{
      f_\a(y) = \a \cdot y = \a \cdot (x \cdot h) = (\a \cdot x) \cdot h = f_\a(x) \cdot h
    }
    so that $f_\a(y) \in f_\a(x) H$, which suffices to show that $f_\a(x)H = f_\a(y)H$ since $G/H$ is a partition.
    Hence $h_\a(yH) = f_\a(y)H = f_\a(x)H = h_\a(xH)$ so that the mapping $h_\a$ is a well-defined function.

    To show that $h_\a$ is a homeomorphism, we first show that it is a bijection.
    Suppose that $xH$ and $yH$ are left cosets where $h_\a(xH) = h_\a(yH)$.
    Then we have $h_\a(xH) = f_\a(x)H = f_\a(y)H = h_\a(yH)$, and hence $f_\a(x) \in f_\a(y)H$.
    From this it follows that
    \gath{
      f_\a(x) = f_\a(y) \cdot h = (\a \cdot y) \cdot h = \a \cdot (y \cdot h) = f_\a(y \cdot h)
    }
    for some $h \in H$.
    Since $f_\a$ is injective (since it was shown in Exercise~\secl.4 to be bijective), it has to be that $x = y \cdot h$ so that $x \in yH$.
    Since also $x \in xH$ and $G/H$ is a partition, it must be that $xH = yH$, which shows that $h_\a$ is injective.
    Now consider any coset $yH \in G/H$.
    Since $f_\a$ is a surjection we have that there is an $x \in G$ where $y = f_\a(x)$.
    Thus it immediately follows that $h_\a(xH) = f_\a(x)H = yH$, which shows that $h_\a$ is surjective since $yH$ was arbitrary.
    This completes the proof that $h_\a$ is a bijection.

    Next we digress a bit and show that $\bigcup h_\a(\cH) = f_\a\parens{\bigcup \cH}$ for any subset $\cH \ss G/H$, where we use the notation $\bigcup \cA = \bigcup_{A \in \cA} A$ for a collection of sets $\cA$.
    So first consider any $x_0 \in \bigcup h_\a(\cH)$ so that there is a coset $yH \in h_\a(\cH)$ where $x_0 \in yH$.
    Then, since $yH \in h_\a(\cH)$, there is another coset $xH \in \cH$ where $x_0 \in yH = h_\a(xH) = f_\a(x)H$.
    We then have that
    \gath{
      x_0 = f_\a(x) \cdot h = (\a \cdot x) \cdot h = \a \cdot (x \cdot h) = f_\a(x \cdot h)
    }
    for some $h \in H$.
    Since clearly $x \cdot h \in xH$ and $xH \in \cH$, we have that $x \cdot h \in \bigcup \cH$.
    Then of course $x_0 \in f_\a(\bigcup \cH)$ since $x_0 = f_\a(x \cdot h)$, which shows that $\bigcup h_\a(\cH) \ss f_\a\parens{\bigcup \cH}$ since $x_0$ was arbitrary.
    Now consider $x_0 \in f_\a(\bigcup \cH)$ so that there is a $y_0 \in \bigcup \cH$ where $x_0 = f_\a(y_0)$.
    Then also there is a coset $xH \in \cH$ where $y_0 \in xH$ since $y_0 \in \bigcup \cH$.
    Hence $y_0 = x \cdot h$ for some $h \in H$.
    Define the coset $yH = h_\a(xH) = f_\a(x)H$ and we have
    \gath{
      x_0 = f_\a(y_0) = \a \cdot y_0 = \a \cdot (x \cdot h) = (\a \cdot x) \cdot h = f_\a(x) \cdot h
    }
    so that $x_0 \in f_\a(x)H = yH$.
    Then, since $yH = h_\a(xH)$ and $xH \in \cH$, we have that $yH \in h_\a(\cH)$.
    As we also have $x_0 \in yH$, it follows that $x_0 \in \bigcup h_\a(\cH)$.
    This shows that $\bigcup h_\a(\cH) \sps f_\a\parens{\bigcup \cH}$ since $x_0$ was arbitrary, which completes the proof that $\bigcup h_\a(\cH) = f_\a\parens{\bigcup \cH}$.

    To return to the main goal, we therefore have that
    \ali{
      \text{$\cU$ is open in $G/H$} &\bic \text{$\bigcup \cU$ is open in $G$} & \text{(by the definition of the quotient space)} \\
      &\bic \text{$f_\a\parens{\bigcup \cU} = \bigcup h_\a(\cU)$ is open in $G$} & \text{(since $f_\a$ is a homeomorphism)} \\
      &\bic \text{$h_\a(\cU)$ is open in $G/H$} \,, & \text{(by the definition of the quotient space)}
    }
    noting that $f_\a$ was shown to be a homeomorphism in Exercise~\secl.4.
    This shows that $h_\a$ is a homeomorphism as desired.

    Lastly, to show that $G/H$ is homogeneous, consider two cosets $xH,yH \in G/H$.
    We know from what was shown in Exercise~\secl.4 that there is an $f_\a$ such that $y = f_\a(x)$.
    If $h_\a$ is the homeomorphism on $G/H$ induced by $f_\a$ as defined above then we have $h_\a(xH) = f_\a(x)H = yH$.
    This suffices to show that $G/H$ is homogeneous since $xH$ and $yH$ were arbitrary.
  }

  (b)
  \qproof{
    Define the single-point subset $\cH_0 = \braces{H}$ of $G/H$, noting that we showed above why $H \in G/H$.
    We have that $\cH_0$ is closed in $G/H$ since we know that $\ivp(\cH_0) = \bigcup \cH_0 = H$ is closed in $G$, which follows from the alternative definition of a quotient map.
    Now consider any arbitrary one-point subset $\cH = \braces{xH} \ss G/H$.
    Since it was shown in part~(a) that $G/H$ is a homogeneous space, there is a homeomorphism $h_\a : G/H \to G/H$ that maps $H$ to $xH$.
    Then we clearly have that $h_\a(\cH_0) = h_\a(\braces{H}) = \braces{xH} = \cH$.
    Since $\cH_0$ is closed in $G/H$ and $h_\a$ is a homeomorphism, it follows that $h_\a(\cH_0) = \cH$ is also closed in $G/H$.
    This shows the desired result since $\cH$ was an arbitrary single-point subset.
  }

  (c)
  \qproof{
    Let $g_\a : G \to G$ be the function defined by $g_\a(x) = x \cdot \a$ for $\a,x \in G$, which we know is a homeomorphism for any $\a \in G$ by Exercise~\secl.4.
    Consider any open set $U$ of $G$.

    We first show that $\ivp(p(U)) = \bigcup_{h \in H} g_h(U)$.

    $(\ss)$
    Consider arbitrary $x \in \ivp(p(U))$ so that $p(x) \in p(U)$.
    Of course $p(x) = xH$ so that $xH \in p(U)$ so that there is a $y \in U$ where $xH = p(y) = yH$.
    From this it follows that $x \in yH$ so that $x = y \cdot h_0$ for some $h_0 \in H$, and so $x = g_{h_0}(y)$.
    Since $y \in U$ we have that $x \in g_{h_0}(U)$, and thus of course $x \in \bigcup_{h \in H} g_h(U)$ since $h_0 \in H$.
    This shows that $\ivp(p(U)) \ss \bigcup_{h \in H} g_h(U)$ since $x$ was arbitrary.

    $(\sps)$
    Now consider $x \in \bigcup_{h \in H} g_h(U)$ so that there is an $h_0 \in H$ where $x \in g_{h_0}(U)$.
    Hence $x = g_{h_0}(y)$ for some $y \in U$, and so $x = y \cdot h_0$.
    This shows that $x \in yH$ since $h_0 \in H$, and thus it must be that $xH = yH$.
    However, we have that $xH = yH = p(y)$ and $y \in U$ so that $xH \in p(U)$.
    Moreover $xH = p(x)$ so that $p(x) \in p(U)$ and hence $x \in \ivp(p(U))$.
    This shows that $\ivp(p(U)) \sps \bigcup_{h \in H} g_h(U)$, which shows the desired result.

    Now, since each $g_h$ is a homeomorphism for $h \in H$ and $U$ is open in $G$, it follows that each $g_h(U)$ also open in $G$.
    Then of course their union $\bigcup_{h \in H} g_h(U) = \ivp(p(U))$ is open in $G$ by the definition of a topology.
    Since $\ivp(p(U))$ is open in $G$, it follows that $p(U)$ is open in $G/H$ since $p$ is a quotient map.
    Then, since $U$ was an arbitrary open set of $G$, this proves that $p$ is an open map.
  }

  (d)
  Recall from algebra that $H$ being a normal subgroup of $G$ means that $g h \inv{g} \in H$ for any $h \in H$ and $g \in G$.
  It is also an equivalent definition of that $xy \in H$ if and only if $yx \in H$ for $x,y \in G$.
  \qproof{
    First we need to show that we can define an operation on $G/H$ that makes it into a group.
    This is done in the expected way: for $xH,yH \in G/H$ define $xH \cdot yH = (x \cdot y) H$.
    To show that this operation is well-defined, suppose that $x_0H = x_1H$ and $y_0H = y_1H$ are elements of $G/H$.
    Then of course $x_1 \in x_0H$ so that $x_1 = x_0 \cdot h_x$ for some $h_x \in H$.
    Similarly $y_1 \in y_0H$ so that $y_1 = y_0 \cdot h_y$ for some $h_y \in H$.
    Freely utilizing the associativity of the operation of $G$ and suppressing the $\cdot$ by using multiplication notation, we then have that
    \ali{
      x_1 y_1 &= (x_0 h_x) (y_0 h_y) \\
      (x_1 y_1) \inv{y_0} &= (x_0 h_x) (y_0 h_y) \inv{y_0}\\
      x_1 y_1 \inv{y_0} &= x_0 h_x (y_0 h_y \inv{y_0}) \\
      x_1 y_1 \inv{y_0} &= x_0 h_x h_1 & \text{(where $h_1 = y_0 h_y \inv{y_0} \in H$ since $H$ is normal)} \\
      x_1 y_1 \inv{y_0} &= x_0 h_2 & \text{(where $h_2 = h_x h_1 \in H$ since $H$ is a group)} \\
      x_1 y_1 \inv{y_0} \inv{x_0} &= x_0 h_2 \inv{x_0} \\
      x_1 y_1 \inv{y_0} \inv{x_0} &= h_3 & \text{(where $h_3 = x_0 h_2 \inv{x_0} \in H$ since $H$ is normal)}
    }
    Hence $(x_1 y_1) (\inv{y_0} \inv{x_0}) = x_1 y_1 \inv{y_0} \inv{x_0} \in H$ so that also $(\inv{y_0} \inv{x_0})(x_1 y_1) \in H$ by the equivalent definition of a normal subgroup.
    Therefore, for some $h \in H$, we have
    \gath{
      \inv{y_0} \inv{x_0} x_1 y_1 = (\inv{x_0} \inv{y_0})(x_1 y_1) = h \\
      \inv{x_0} x_1 y_1 = y_0 h \\
      x_1 y_1 = x_0 y_0 h \\
      x_1 y_1 = (x_0 y_0) h
    }
    so that $x_1 \cdot y_1 \in (x_0 \cdot y_0) H$, which of course shows that $(x_1 \cdot y_1)H = (x_0 \cdot y_0)H$, and hence the operation on $G/H$ is well-defined.

    Now we show that $G/H$ with this operation satisfies the group axioms.
    We first note that, for $x,y \in G$, we have $xH,yH \in G/H$ and $x \cdot y \in G$ since $G$ is a group so that $xH \cdot yH = (x \cdot y)H \in G/H$.
    Hence the operation is closed in $G/H$.
    Next, clearly $eH = H$ itself is the identity element for $G/H$ since we have $eH \cdot xH = (e \cdot x)H = xH$ and $xH \cdot eH = (x \cdot e)H = xH$ for any $xH \in G/H$.
    We also have that the inverse element of $xH$ is $\inv{x} H$ since $xH \cdot \inv{x}H = (x \cdot \inv{x})H = eH$ and $\inv{x}H \cdot xH = (\inv{x} \cdot x)H = eH$.
    Lastly, we have that
    \ali{
      (xH \cdot yH) \cdot zH &= (x \cdot y)H \cdot zH = ((x \cdot y) \cdot z)H = (x \cdot (y \cdot z))H \\
      &= xH \cdot (y \cdot z)H = xH \cdot (yH \cdot zH)
    }
    since of course the operation on $G$ is associative.
    This shows that the operation on $G/H$ is associative as well, which completes the proof that $G/H$ is a group.

    To show that $G/H$ is in fact a topological group, first it was shown in part~(b) that one-point subsets of $G/H$ are closed in $G/H$ since $H$ is closed in $G$.
    From this is follows that $G/H$ satisfies the $T_1$ axiom since any finite subset of $G/H$ is a finite union of one-point sets and so is also closed in $G/H$.

    At this point we take a short digression and show that if $f: G \times G \to G$ is continuous, then the function $h : G/H \times G/H \to G/H$ defined by $h(xH, yH) = f(x,y)H$ is also continuous.
    To see this, we first claim that $h \circ (p \times p) = p \circ f$, where the function $p \times p : G \times G \to G/H \times G/H$ is defined as $(p \times p)(x,y) = (p(x),p(y))$ as in Exercise~18.10.
    This is easy to show as we have
    \ali{
      (h \circ (p \times p))(x,y) &= h((p \times p)(x,y)) = h(p(x), p(y)) = h(xH, yH) \\
      &= f(x,y)H = p(f(x,y)) = (p \circ f)(x,y)
    }
    for any $x,y \in G$.
    Since both $p$ and $f$ are continuous, it follows that $p \circ f = h \circ (p \times p)$ is also continuous by Theorem~18.2 part~(c).
    We also have that $p \times p$ is an open quotient map by the remarks in the text since $p$ is an open quotient map by part~(c).

    At this point, we use Theorem~22.2 to show that $h$ is continuous.
    As this can be confusing, we include the following table, which shows how the sets and functions in the statement of Theorem~22.2 map to the sets and functions we are working with:
    \begin{center}
      \begin{tabular}{c|c|c}
        Type & Theorem~22.2 & Ours \\
        \hline
        Set & $X$ & $G \times G$ \\
        Set & $Y$ & $G/H \times G/H$ \\
        Set & $Z$ & $G/H$ \\
        Function & $p: X \to Y$ & $p \times p$ \\
        Function & $g: X \to Z$ & $p \circ f = h \circ (p \times p)$ \\
        Function & $f: Y \to Z$ & $h$
      \end{tabular}
    \end{center}
    Now we show that the conditions of the theorem are met.
    We have already shown that $p \times p$ is a quotient map.
    Now let $P = \braces{(xH, yH)}$ be a one-point subset of $G/H \times G/H$.
    Since $p \times p$ is a quotient map, it is surjective so that $(p \times p)(\inv{(p \times p)}(P)) = P$ by Exercise~2.1.
    Then clearly
    \ali{
      (p \circ f)(\inv{(p \times p)}(P)) &= (h \circ (p \times p))(\inv{(p \times p)}(P)) \\
      &= h((p \times p)(\inv{(p \times p)}(P))) = h(P) \\
      &= \braces{h(xH, yH)} \,,
    }
    which shows that $p \circ f$ is constant on the set $\inv{(p \times p)}(P)$.
    Thus $p \circ f$ induces the function $h$ per Theorem~22.2 since $h \circ (p \times p) = p \circ f$ as shown above.
    It then follows by the theorem that $h$ is continuous since we have shown above that $p \circ f$ is.

    Returning to the main problem, since $G$ is a topological group the function $f: G \times G \to G$ defined by $f(x,y) = x \cdot \inv{y}$ is continuous by Exercise~\secl.1.
    It then follows by what was just shown that $h : G/H \times G/H \to G/H$ defined by
    \gath{
      h(xH, yH) = xH \cdot \inv{(yH)} = xH \cdot \inv{y}H = (x \cdot \inv{y})H = f(x,y)H
    }
    is also continuous.
    This suffices to show that $G/H$ is also a topological group as desired, again by Exercise~\secl.1.
  }
}

\exercise{6}{
  The integers $\ints$ are a normal subgroup of $(\reals,+)$.
  The quotient $\reals/\ints$ is a familiar topological group; what is it?
}
\sol{
  The quotient $\reals/\ints$ is the reals with addition modulo one.
  This is to say that two reals $x$ and $y$ are considered equivalent (i.e. they are in the same equivalence class) if $x-y \in \ints$.
  This is also homeomorphic and isomorphic (with respect to the group) to the topological group $(S^1,\cdot)$ from Exercise~\secl.2 part~(d) by the map $f: \reals/\ints \to S^1$ defined by $f(x\ints) = e^{i 2\pi x}$.
  We shall not show this rigorously as doing so would be tedious, but it is not difficult to see intuitively.
}

\exercise{7}{
  If $A$ and $B$ are subsets of $G$, let $A \cdot B$ denote the set of all points $a \cdot b$ for $a \in A$ and $b \in B$.
  Let $\inv{A}$ denote the set of all points $\inv{a}$, for $a \in A$.
  \eparts{
  \item A neighborhood $V$ of the identity element $e$ is said to be \boldit{symmetric} if $V = \inv{V}$.
    If $U$ is a neighborhood of $e$, show that there is a symmetric neighborhood $V$ of $e$ such that $V \cdot V \ss U$.
    [Hint: If $W$ is a neighborhood of $e$, then $W \cdot \inv{W}$ is symmetric.]
  \item Show that $G$ is Hausdorff.
    In fact, show that if $x \neq y$, there is a neighborhood $V$ of $e$ such that $V \cdot x$ and $V \cdot y$ are disjoint.
  \item Show that $G$ satisfies the following separation axiom, which is called the \boldit{regularity axiom}: Given a closed set $A$ and a point $x$ not in $A$, there exist disjoint open sets containing $A$ and $x$, respectively.
    [Hint: There is a neighborhood $V$ of $e$ such that $V \cdot x$ and $V \cdot A$ are disjoint.]
  \item Let $H$ be a subgroup of $G$ that is closed in the topology of $G$; let $p: G \to G/H$ be the quotient map.
    Show that $G/H$ satisfies the regularity axiom.
    [Hint: Examine the proof of (c) when $A$ is saturated.]
  }
}
\sol{
  \begin{lem} \label{lem:setg:contf}
    Suppose that $X$ and $Y$ are topological spaces and $f: X \times X \to Y$ is continuous.
    Also suppose that $f(x, x) = y$ for some $x \in X$ and $y \in Y$.
    Then, for any neighborhood $V$ of $y$, there is a neighborhood $U$ of $x$ such that $f(U \times U) \ss V$.
  \end{lem}
  \qproof{
    Let $V$ be any neighborhood of $y = f(x,x)$ in $Y$.
    Then there is a neighborhood $U'$ of $(x,x)$ in $X \times X$ such that $f(U') \ss V$ by Theorem~18.1 part~(4).
    Now, since $U'$ is an open set of $X \times X$ containing $(x,x)$, there is a basis element $B = U_1 \times U_2$ of $X \times X$ such that $(x,x) \in U_1 \times U_2 = B \ss U'$, where of course $U_1$ and $U_2$ are open in $X$.
    Then, being a finite intersection of open sets, $U = U_1 \cap U_2$ is also open in $X$ and we have $x \in U$ since $x \in U_1$ and $x \in U_2$.
    Hence $U$ is a neighborhood of $x$ in $X$, and of course both $U \ss U_1$ and $U \ss U_2$.

    Now consider any $z \in f(U \times U)$ so that there is an $(x_1,x_2) \in U \times U$ where $f(x_1,x_2) = z$.
    Then $x_1 \in U \ss U_1$ and $x_2 \in U \ss U_2$ so that $(x_1,x_2) \in U_1 \times U_2 = B \ss U'$.
    Hence $z = f(x_1,x_2) \in f(U')$ so that also $z \in V$ since $f(U') \ss V$.
    This shows the desired result that $f(U \times U) \ss V$ since $z$ was arbitrary.
  }

  \begin{lem} \label{lem:setg:mulinv}
    $\inv{(x \cdot y)} = \inv{y} \cdot \inv{x}$ for any $x$ and $y$ in a group.
  \end{lem}
  \qproof{
    Let $e$ be the identity element of the group.
    Then we have
    \ali{
      \inv{(x \cdot y)} \cdot (x \cdot y) &= e & \text{(definition of the inverse)} \\
      \parens{\inv{(x \cdot y)} \cdot x} \cdot y &= e & \text{(associativity of the operation)} \\
      \inv{(x \cdot y)} \cdot x &= e \cdot \inv{y} \\
      \inv{(x \cdot y)} \cdot x &= \inv{y} & \text{(definition of the identity element)} \\
      \inv{(x \cdot y)} &= \inv{y} \cdot \inv{x} \,,
    }
    which shows the desired result.
  }

  \begin{lem} \label{lem:setg:topen}
    If $G$ is a topological group and $U$ is an open set of $G$ then the sets $\a \cdot U$ and $U \cdot \a$ are also open in $G$ for any $\a \in G$.
    Similarly, if $C$ is a closed set of $G$ then $\a \cdot C$ and $C \cdot \a$ are also closed.
  \end{lem}
  \qproof{
    For $\a \in G$, of course $\a \cdot U$ denotes the set $\braces{\a \cdot x \where x \in U}$, and analogously $U \cdot \a = \braces{x \cdot \a \where x \in U}$.
    The openness of $\a \cdot U$ and $U \cdot \a$ follow almost immediately from what was shown in Exercise~\secl.4.
    It is trivial to show that $\a \cdot U = f_\a(U)$, where $f_\a : G \to G$ is defined by $f_\a(x) = \a \cdot x$ as in Exercise~\secl.4.
    We know from that exercise that $f_\a$ is a homeomorphism so that $\a \cdot U = f_\a(U)$ is open since $U$ is.
    Analogously $U \cdot \a = g_\a(U)$ is also open for the same reason, where $g_\a(x) = x \cdot \a$ as in Exercise~\secl.4.
    If $C$ is a closed set of $G$ then $\a \cdot C = f_\a(C)$ and $C \cdot \a = g_\a(C)$ are also closed since the homeomorphisms $f_\a$ and $g_\a$ also of course preserve closed sets.
  }
  
  \begin{lem} \label{lem:setg:sym}
    If $G$ is a topological group with identity element $e$ and $U$ is a neighborhood of $e$, then $U \cdot \inv{U}$ is a symmetric neighborhood of $e$.
  \end{lem}
  \qproof{
    First we show that $U \cdot \inv{U}$ is indeed a neighborhood of $e$.
    We claim that
    \gath{
      U \cdot \inv{U} = \bigcup_{\a \in \inv{U}} (U \cdot \a) \,,
    }
    We have
    \ali{
      x \in U \cdot \inv{U} &\bic \exists y \in U \exists \a \in \inv{U} \parens{x = y \cdot \a} \\
      &\bic \exists \a \in \inv{U} \exists y \in U \parens{x = y \cdot \a} \\
      &\bic \exists \a \in \inv{U} \parens{x \in U \cdot \a} \\
      &\bic x \in \bigcup_{\a \in \inv{U}} (U \cdot \a) \,
    }
    which of course shows the desired result.
    Now, we know from Lemma~\ref{lem:setg:topen} that each $U \cdot \a$ is open since $U$ is open (being a neighborhood of $e$).
    Hence their union is open by the definition of a topology, which shows that $U \cdot \inv{U}$ is in fact open.
    Also, since $e \in U$ and $\inv{e} = e$ (a well-known property of the identity element in any group), we clearly have that $e = e \cdot e = e \cdot \inv{e} \in U \cdot \inv{U}$ and therefore $U \cdot \inv{U}$ is a neighborhood of $e$.

    To show that $U \cdot \inv{U}$ is symmetric, we have
    \ali{
      z \in \inv{(U \cdot \inv{U})} &\bic \exists x \in U \exists y \in U \parens{z = \inv{(x \cdot \inv{y})}} \\
      &\bic \exists x \in U \exists y \in U \parens{z = \inv{(\inv{y})} \cdot \inv{x}} & \text{(by Lemma~\ref{lem:setg:mulinv})} \\
      &\bic \exists x \in U \exists y \in U \parens{z = y \cdot \inv{x}} \\
      &\bic \exists y \in U \exists x \in U \parens{z = y \cdot \inv{x}} \\
      &\bic z \in U \cdot \inv{U} \,,
    }
    which shows that $\inv{(U \cdot \inv{U})} = U \cdot \inv{U}$ so that $U \cdot \inv{U}$ is symmetric by definition.
  }

  \mainprob

  (a)
  \qproof{
    Suppose that $U$ is any neighborhood of $e$.
    Since $G$ is a topological group, we know that the function $f: G \times G \to G$ defined by $f(x,y) = x \cdot y$ is continuous.
    Then, since $f(e,e) = e \cdot e = e$, it follows from Lemma~\ref{lem:setg:contf} that there is a neighborhood $V'$ of $e$ such that $f(V' \times V') \ss U$.
    Now we claim that $f(V' \times V') = V' \cdot V'$.
    For any $z \in f(V' \times V')$ we have that there is an $(x,y) \in V' \times V$ where $f(x,y) = x \cdot y = z$.
    Since $x,y \in V'$ and $z = x \cdot y$, this shows that $z \in V' \cdot V'$ so that $f(V' \times V') \ss V' \cdot V'$.
    To show the other direction, for any $z \in V' \cdot V'$ we have that $z = x \cdot y$ for some $x,y \in V'$.
    Then $(x,y) \in V' \times V'$ and $z = x \cdot y = f(x,y)$ so that $z \in f(V' \times V')$, which shows that $f(V' \times V') \sps V' \cdot V'$.
    This shows the desired result that $V' \cdot V' = f(V' \times V') \ss U$.
    
    Similarly the function $g: G \times G \to G$ defined by $g(x,y) = x \cdot \inv{y}$ is also continuous by Exercise~\secl.1 since $G$ is a topological group.
    Then, since $V'$ is a neighborhood of $e$ and $g(e,e) = e \cdot \inv{e} = e \cdot e = e$, it follows again from Lemma~\ref{lem:setg:contf} that there is a neighborhood $W$ of $e$ such that $g(W \times W) \ss V'$.
    We have that $W \cdot \inv{W} = g(W \times W)$ by an argument analogous to that for $f$ above so that $W \cdot \inv{W} = g(W \times W) \ss V'$.
    Let $V = W \cdot \inv{W} \ss V'$, which we know is a symmetric neighborhood of $e$ by Lemma~\ref{lem:setg:sym} and is the neighborhood we seek.

    So consider any $z \in V \cdot V$ so that $z = x \cdot y$ for some $x,y \in V$.
    Then also $x,y \in V'$ since $V \ss V'$.
    From this it follows that $z = x \cdot y \in V' \cdot V'$ so that also $z \in U$ since $V' \cdot V' \ss U$.
    This shows the desired result that $V \cdot V \ss U$ since $z$ was arbitrary, which completes the overall proof.
  }

  (b)
  \qproof{
    Suppose that $x,y \in G$ and that $x \neq y$.
    Then there are neighborhoods $U_x'$ of $x$ and $U_y'$ of $y$ such that $y \notin U_x'$ and $x \notin U_y'$.
    This follows from Exercise~17.15 since $G$ satisfies the $T_1$ axiom on account of it being a topological group.
    Then $U_x = U_x' \cdot \inv{x}$ and $U_y = U_y' \cdot \inv{y}$ are both neighborhoods of $e$ since they are open by Lemma~\ref{lem:setg:topen} and we have that $e = x \cdot \inv{x} \in U_x' \cdot \inv{x} = U_x$ since $x \in U_x'$, and analogously $e = y \cdot \inv{y} \in U_y' \cdot \inv{y} = U_y$ since $y \in U_y'$.
    Note that we also have that
    \ali{
      z \in U_y \cdot y &\bic z \in (U_y' \cdot \inv{y}) \cdot y \\
      &\bic \exists y' \in U_y' (z = (y' \cdot \inv{y}) \cdot y) \\
      &\bic \exists y' \in U_y' (z = y' \cdot (\inv{y} \cdot y)) \\
      &\bic  \exists y' \in U_y' (z = y' \cdot e) \\
      &\bic  \exists y' \in U_y' (z = y') \\
      &\bic z \in U_y'
    }
    so that clearly $U_y \cdot y = U_y'$.

    Now, let $U = U_x \cap U_y$, which is also obviously a neighborhood of $e$.
    It then follows from part~(a) that there is a symmetric neighborhood $V$ of $e$ such that $V \cdot V \ss U$.
    Suppose that $V \cdot x$ and $V \cdot y$ are not disjoint so that there is a $z \in V \cdot x$ where also $z \in V \cdot y$.
    Then we have that $z = v_x \cdot x = v_y \cdot y$ for some $v_x,v_y \in V$.
    It then follows that $x = (\inv{v_x} \cdot v_y) \cdot y$ and, since $\inv{v_x} \in \inv{V} = V$ as $V$ is symmetric, we have $\inv{v_x} \cdot v_y \in V \cdot V \ss U \ss U_y$.
    Thus $x = (\inv{v_x} \cdot v_y) \cdot y \in U_y \cdot y = U_y'$, but we know that $x$ cannot be in $U_y'$ by its definition per the $T_1$ axiom!
    This contradiction means that it must be that $V \cdot x$ and $V \cdot y$ are disjoint, which shows the desired result.

    From this the fact that $G$ is Hausdorff readily follows.
    Clearly $V \cdot x$ is a neighborhood of $x$ since it is open by Lemma~\ref{lem:setg:topen} and we have $x = e \cdot x \in V \cdot x$.
    Similarly $V \cdot y$ is a neighborhood of $y$.
    As we have shown that these are disjoint, this suffices to show that $G$ is a Hausdorff space.
  }

  (c)
  \qproof{
    A proof of this is similar to the proof of part~(b).
    Since $A$ is closed, we know that $A \cdot \inv{x}$ is also closed by Lemma~\ref{lem:setg:topen}.
    Then $G - A \cdot \inv{x}$ is of course open.
    Moreover if $e$ were in $A \cdot \inv{x}$ then we would have $e = a \cdot \inv{x}$ for some $a \in A$ so that $a = e \cdot x = x$, which is not possible since we know that $x \notin A$.
    So it must be that $e \notin A \cdot \inv{x}$ so that $e \in G - A \cdot \inv{x}$ since of course $e \in G$.
    Hence $G - A \cdot \inv{x}$ is a neighborhood of $e$, and thus there is a symmetric neighborhood $V$ of $e$ such that $V \cdot V \ss G - A \cdot \inv{x}$ by what was shown in part~(a).

    We claim that $V \cdot A$ and $V \cdot x$ are disjoint.
    To see this, suppose to the contrary that there is a $y \in V \cdot A$ where also $y \in V \cdot x$.
    Then $y = v_a \cdot a = v_x \cdot x$ for some $v_a,v_x \in V$ and $a \in A$.
    Then we have
    \gath{
      v_x \cdot x = v_a \cdot a \\
      \inv{v_a} \cdot v_x \cdot x = a \\
      \inv{v_a} \cdot v_x = a \cdot \inv{x} \,.
    }
    Since $V$ is symmetric we have that $\inv{v_a} \in \inv{V} = V$ so that $\inv{v_a} \cdot v_x \in V \cdot V \ss G - A \cdot \inv{x}$ and hence $\inv{v_a} \cdot v_x \notin A \cdot \inv{x}$.
    However, clearly $\inv{v_a} \cdot v_x = a \cdot \inv{x}$ so that $\inv{v_a} \cdot v_x \in A \cdot \inv{x}$ since $a \in A$.
    This contradiction can only mean that in fact $V \cdot A$ and $V \cdot x$ are disjoint.

    Now, for any $a \in A$, we have that $a = e \cdot a \in V \cdot A$ since $e \in V$, which shows that $A$ is contained in $V \cdot A$.
    Similar to what was done in the beginning of the proof of Lemma~\ref{lem:setg:sym}, it is easy to show that
    \gath{
      V \cdot A = \bigcup_{a \in A} V \cdot a \,,
    }
    which is open since each $V \cdot a$ is open by Lemma~\ref{lem:setg:topen}.
    Thus $V \cdot A$ is an open set containing $A$.
    We also have that $V \cdot x$ is a neighborhood of $x$ since it is open by Lemma~\ref{lem:setg:topen} and $x = e \cdot x \in V \cdot x$ since $e \in V$.
    Since we have already shown that $V \cdot A$ and $V \cdot x$ are disjoint, this shows the desired result that $G$ satisfies the regularity axiom.
  }

  (d)
  \qproof{
    Suppose that $A$ is a closed subset of $G/H$ and $xH$ is an element of $G/H$ not contained in $A$.
    Then $\ivp(A)$ is a closed subset of $G$ since $p$ is a quotient map.
    Moreover $\ivp(A)$ cannot contain $x$ for, if it did, then we would have $xH = p(x) \in A$.
    So let $V$ be a symmetric neighborhood of $e$ such that $V \cdot \ivp(A)$ and $V \cdot x$ are disjoint, as shown to exist in part~(c).

    It was also shown in part~(c) that $V \cdot \ivp(A)$ is an open set in $G$ containing $\ivp(A)$ and $V \cdot x$ is a neighborhood of $x$ in $G$.
    So, for any $yH \in A$ we have that $p(y) = yH \in A$ so that $y \in \ivp(A)$ and hence also $y \in V \cdot \ivp(A)$.
    Then of course $yH = p(y) \in p(V \cdot \ivp(A))$, which shows that $A \ss p(V \cdot \ivp(A))$ since $yH$ was arbitrary.
    We also have that $p(V \cdot \ivp(A))$ is open in $G/H$ since $V \cdot \ivp(A)$ is open in $G$ and $p$ is an open map by Exercise~\secl.5 part~(c).
    Similarly, since $x \in V \cdot x$, we have that $xH = p(x) \in p(V \cdot x)$ and $p(V \cdot x)$ is open in $G/H$ since $p$ is an open map.
    Thus $p(V \cdot x)$ is a neighborhood of $xH$ in $G/H$.

    Now we show that $p(V \cdot \ivp(A))$ and $p(V \cdot x)$ are disjoint subsets of $G/H$.
    Suppose to the contrary that they are not so that there is a $zH \in p(V \cdot \ivp(A))$ where also $zH \in p(V \cdot x)$.
    Then we have that there is a $y_a \in V \cdot \ivp(A)$ where $z = p(y_a)$ and likewise a $y_x \in V \cdot x$ where $z = p(y_x)$.
    Thus $y_a H = p(y_a) = z = p(y_x) = y_x H$ so that $y_x \in y_a H$, and hence there is an $h \in H$ where $y_x = y_a \cdot h$.
    Also since $y_a \in V \cdot \ivp(A)$ we have $y_a = v_a \cdot a'$ for some $v_a \in V$ and $a' \in \ivp(A)$.
    Putting this together we have
    \gath{
      y_x = y_a \cdot h = v_a \cdot a' \cdot h \,.
    }
    Since $a' \in \ivp(A)$ we have that $a'H = p(a') \in A$.
    Also clearly $a' \cdot h \in a'H$ since $h \in H$.
    Therefore $a' \cdot h \in \bigcup A = \ivp(A)$.
    Hence we have that $y_x = v_a \cdot (a' \cdot h) \in V \cdot \ivp(A)$.
    Since also $y_x \in V \cdot x$, this violates the fact that $V \cdot \ivp(A)$ and $V \cdot x$ are disjoint.
    This contradiction means that it must be that in fact $p(V \cdot \ivp(A))$ and $p(V \cdot x)$ are disjoint, which completes the proof that $G/H$ satisfies the regularity axiom.
  }
}

% Restore section label to number
\renewcommand\thesubsection{\arabic{subsection}}
