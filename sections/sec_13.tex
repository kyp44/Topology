\setcounter{subsection}{13-1}
\subsection{Basis for a Topology}

\def\cA{\col{A}}
\def\cB{\col{B}}
\def\cC{\col{C}}
\def\cT{\col{T}}
\def\cTa{\col{T}_\a}
\def\cTb{\col{T}_\b}

\exercise{1}{
  Let $X$ be a topological space; let $A$ be a subset of $X$.
  Suppose that for each $x \in A$ there is an open set $U$ containing $x$ such that $U \ss A$.
  Show that $A$ is open in $X$.
}
\sol{
  \dwhitman

  \qproof{
    For each $x \in A$ we can choose an open set $U_x$ containing $x$ such that $U_x \ss A$.
    We then claim that $\bigcup_{x \in A} U_x = A$.
    So first consider any $y \in \bigcup_{x \in A} U_x$ so that there is an $x \in A$ such that $y \in U_x$.
    Then clearly also $y \in A$ since $U_x \ss A$.
    Hence $\bigcup_{x \in A} U_x \ss A$ since $y$ was arbitrary.
    Now consider $y \in A$ so that clearly $y \in U_y$.
    Then obviously $y \in \bigcup_{x \in A} U_x$ so that $A \ss \bigcup_{x \in A} U_x$ since $y$ was arbitrary.
    Thus we have shown that $\bigcup_{x \in A} U_x = A$, and since each $U_x$ is open, it follows from the definition of a topology that the union $\bigcup_{x \in A} U_x = A$ is open as well.
  }
}

\exercise{2}{
  Consider the nine topologies on the set $X = \braces{a,b,c}$ indicated in Example~1 of \S 12.
  Compare them; that is, for each pair of topologies, determine whether they are comparable, and if so, which is finer.
}
\sol{
  \dwhitman

  We label each of the topologies in Figure~12.1 with an ordered pair $(i,j)$ where $1 \leq i,j \leq 3$, $i$ is the row, $j$ is the column, and $(1,1)$ is the upper left corner.
  The following matrix lists which of each pair is \emph{finer}, or ``Inc'' if they are incomparable.
  \begin{center}
    \begin{tabular}{c|c|c|c|c|c|c|c|c|c|}
      & $(1,1)$ & $(1,2)$ & $(1,3)$ & $(2,1)$ & $(2,2)$ & $(2,3)$ & $(3,1)$ & $(3,2)$ & $(3,3)$ \\
      \hline
      $(1,1)$ & $=$ & $(1,2)$ & $(1,3)$ & $(2,1)$ & $(2,2)$ & $(2,3)$ & $(3,1)$ & $(3,2)$ & $(3,3)$ \\
      \hline
      $(1,2)$ &  & $=$ & Inc & Inc & Inc & Inc & $(1,2)$ & $(3,2)$ & $(3,3)$ \\
      \hline
      $(1,3)$ &  &  & $=$ & $(1,3)$ & Inc & $(2,3)$ & $(1,3)$ & Inc & $(3,3)$ \\
      \hline
      $(2,1)$ &  &  &  & $=$ & Inc & $(2,3)$ & Inc & $(3,2)$ & $(3,3)$ \\
      \hline
      $(2,2)$ &  &  &  &  & $=$ & Inc & Inc & Inc & $(3,3)$ \\
      \hline
      $(2,3)$ &  &  &  &  &  & $=$ & $(2,3)$ & Inc & $(3,3)$ \\
      \hline
      $(3,1)$ &  &  &  &  &  &  & $=$ & $(3,2)$ & $(3,3)$ \\
      \hline
      $(3,2)$ &  &  &  &  &  &  &  & $=$ & $(3,3)$ \\
      \hline
      $(3,3)$ &  &  &  &  &  &  &  &  & $=$ \\
      \hline
    \end{tabular}
  \end{center}
  We know that $\pss$ forms a strict partial order on these topologies.
  So we can also list all the maximal simply ordered subsets, each in order:
  \gath{
    (1,1) \pss (2,2) \pss (3,3) \\
    (1,1) \pss (3,1) \pss (1,2) \pss (3,2) \pss (3,3) \\
    (1,1) \pss (3,1) \pss (1,3) \pss (2,3) \pss (3,3) \\
    (1,1) \pss (2,1) \pss (1,3) \pss (2,3) \pss (3,3) \\
    (1,1) \pss (2,1) \pss (3,2) \pss (3,3)
  }
}

\def\cTi{\col{T}_\infty}
\exercise{3}{
  Show that the collection $\cTa$ given in Example~4 of \S 12 is a topology on $X$.
  Is the collection
  \gath{
    \cTi = \braces{U \where X - U \text{ is infinite or empty or all of } X}
  }
  a topology on $X$?
}
\sol{
  \dwhitman

  Recall that $\cTa$ from Example~12.4 is the set of all subsets $U$ of $X$ such that $X - U$ either is countable or is all of $X$.
  First we show that $\cTa$ is a topology on $X$.
  \qproof{
    First, clearly $\es \in \cTa$ since $X - \es = X$ is all of $X$.
    Also $X \in \cTa$ since $X - X = \es$ is countable.
    Now suppose that $\cA$ is a subcollection of $\cTa$ so that $X - U$ is countable (or all of $X$) for every $U \in \cA$.
    Then we have that
    \gath{
      X - \bigcup \cA = X - \bigcup_{A \in \cA} A = \bigcap_{A \in \cA} (X - A)
    }
    is countable (or all of $X$) since every $X - A$ is countable (or all of $X$).
    Therefore $\bigcup \cA \in \cTa$ by definition.

    Now suppose that $U_1, \ldots U_n$ are nonempty elements of $\cTa$ so that $X - U_i$ is a countable subset of $X$ or $X$ itself for each $i \in \intsfin{n}$.
    Then we have
    \gath{
      X - \bigcap_{i=1}^n U_i = \bigcup_{i=1}^n (X - U_i)
    }
    is a finite union of sets that are either countable subsets of $X$, or $X$ itself.
    It then follows that the union is countable or $X$ itself so that $\bigcap_{i=1}^n U_i \in \cTa$ by definition.
    This completes the proof that $\cTa$ is a topology on $X$.
  }

  Now we claim that the collection $\cTi$ as defined above is not always a topology on $X$.
  \qproof{
    As a counterexample, let $X = \pints$ and suppose that $\cTi$ is a topology on $X$.
    Clearly if $U$ is a finite subset of $X$, then $X - U$ is infinite since $X$ is infinite so that $U$ is open.
    Now consider the subcollection
    \gath{
      \cA = \braces{\braces{i} \where i \in \pints \text{ and } i > 1} = \braces{\braces{2}, \braces{3}, \ldots} \,.
    }
    Then clearly we have that $\bigcup \cA = \braces{2,3, \ldots}$ so that $X - \bigcup \cA = \braces{1}$ is neither infinite, empty, nor all of $X$.
    Therefore $\bigcup \cA$ cannot be open, which violates property (2) of a topology.
    So it must be that $\cTi$ is not a topology, which of course contradicts our supposition that it is!
  }
}

\def\cTo{\col{T}_1}
\def\cTt{\col{T}_2}
\def\cTs{\col{T}_s}
\def\cTl{\col{T}_l}
\exercise{4}{
  \eparts{
  \item If $\braces{\cTa}$ is a family of topologies on $X$, show that $\bigcap \cTa$ is a topology on $X$.
    Is $\bigcup \cTa$ a topology on $X$?
  \item Let $\braces{\cTa}$ be family of topologies on $X$.
    Show that there is a unique smallest topology on $X$ containing all the collections $\cTa$, and a unique largest topology contained in all $\cTa$.
  \item If $X = \braces{a,b,c}$, let
    \gath{
      \cTo = \braces{\es, X, \braces{a}, \braces{a,b}} \condgap \text{and} \condgap \cTt = \braces{\es, X, \braces{a}, \braces{b,c}} \,.
    }
    Find the smallest topology containing $\cTo$ and $\cTt$, and the largest topology contained in $\cTo$ and $\cTt$.      
  }
}
\sol{
  \dwhitman

  (a) First we show that $\bigcap \cTa$ is a topology on $X$.
  \qproof{
    First, clearly since $\es$ and $X$ are in every $\cTa$ since they are topologies, they are both in $\bigcap \cTa$ so that property (1).
    Now suppose that $\cA$ is a subcollection of $\bigcap \cTa$.
    Consider any $\cTb$ and any $A \in \cA$.
    Then $A$ is also in $\bigcap \cTa$ since $\cA \ss \bigcap \cTa$.
    It then follows that $A$ is in our specific $\cTb$.
    Since $A$ was arbitrary it follows that $\cA$ is a subcollection of $\cTb$ so that $\bigcup \cA \in \cTb$ also  since $\cTb$ is a topology.
    Since $\cTb$ was also arbitrary it follows that $\bigcup \cA \in \bigcap \cTa$.
    Lastly, since the subcollection $\cA$ was arbitrary, this shows property (2) for $\bigcap \cTa$.

    Finally, suppose that $U_1, \ldots, U_n$ are sets in $\bigcap \cTa$.
    Consider any $\cTb$ so that clearly then $U_i \in \cTb$ for every $i \in \intsfin{n}$.
    It then follows that $\bigcap_{i=1}^n U_i \in \cTb$ since $\cTb$ is a topology.
    Since $\cTb$ was arbitrary, this shows that $\bigcap_{i=1}^n U_i \in \bigcap \cTa$, which shows property (3) for $\bigcap \cTa$.
    This completes the proof that $\bigcap \cTa$ is a topology on $X$ since all three properties have been shown.
  }

  Now we claim that $\bigcup \cTa$ is \emph{not} generally a topology.
  \qproof{
    As a counterexample consider the set $X = \braces{a,b,c}$, the topologies $\cTo = \braces{\es, X, \braces{a}}$ and $\cTt = \braces{\es, X, \braces{b}}$, and the collection of topologies $\col{C} = \braces{\cTo, \cTt}$.
    Then we clearly have that $\bigcup \col{C} = \cTo \cup \cTt = \braces{\es, X, \braces{a}, \braces{b}}$, which is not a topology since $\cA = \braces{\braces{a}, \braces{b}}$ is a subcollection of $\bigcup \col{C}$ but $\bigcup \cA = \braces{a,b}$ is not in $\bigcup \col{C}$.
  }

  (b) First we show that there is a unique smallest topology that contains each $\cTa$.
  \qproof{
    It was proven in part (a) that $\bigcup \cTa$ is not necessarily a topology.
    However, it is clearly always a subbasis for a topology since clearly $X \in \bigcup \cTa$ since it is in each $\cTa$ since they are topologies.
    Hence obviously then $\bigcup \parens{\bigcup \cTa} = X$ so that $\bigcup \cTa$ is a subbasis by definition.
    Then let $\cTs$ be the topology generated by the subbasis $\bigcup \cTa$.
    We claim that $\cTs$ is then the smallest topology that contains all the $\cTa$ as subsets.

    First, from the proof following the definition of a subbasis, we know that the set $\cB$ of finite intersections of elements of $\bigcup \cTa$ is a basis for the topology $\cTs$, and that $\cTs$ is the set of all unions of subcollections of $\cB$.

    We first show that every $\cTa$ is indeed contained as a subset of $\cTs$.
    So consider any specific $\cTb$ and any $U \in \cTb$.
    Then clearly $U \in \bigcup \cTa$ so that $U \in \cB$ since $U = \bigcap \braces{U}$ is a finite intersection of elements of $\bigcup \cTa$.
    It then follows that $U \in \cTs$ since $U = \bigcup \braces{U}$ is the union of a subcollection of $\cB$.
    Since $U$ was arbitrary, this shows that $\cTb \ss \cTs$, which shows the result since $\cTb$ was arbitrary.

    Now we show that $\cTs$ is the smallest such topology as ordered by $\pss$.
    So suppose that $\cT$ is a topology that contains every $\cTa$ as a subset.
    Consider any $U \in \cTs$ so that $U = \bigcup \cC$ for some subcollection $\cC \ss \cB$.
    Now consider any $Y \in \cC$ so that also $Y \in \cB$.
    Then $Y = \bigcap_{i=1}^n Y_i$ where each $Y_i \in \bigcup \cTa$.
    Then each $Y_i$ is in some $\cTb \ss \cT$ so that also $Y_i \in \cT$.
    Since $\cT$ is a topology, it follows that the finite intersection $\bigcap_{i=1}^n Y_i = Y$ is also in $\cT$.
    Since $Y$ was arbitrary, this shows that $\cC \ss \cT$ so that $\cC$ is a subcollection of $\cT$.
    It then follows that $\bigcup \cC = U$ is also in $\cT$ since $\cT$ is a topology.
    Since $U$ was arbitrary, we have that $\cTs \ss \cT$, which shows that $\cTs$ is the smallest topology since $\cT$ was arbitrary.

    It is easy to see that $\cTs$ is unique since, if both $\cTo$ and $\cTs$ are the smallest topologies that contain each $\cTa$ as subsets, then we would have that both $\cTo \ss \cTt$ and $\cTt \ss \cTo$ so that $\cTo = \cTt$.
    Really this follows from the more general fact that smallest elements in any order are always unique.
  }

  Next we show that there is a unique largest topology that is contained in each $\cTa$.
  \qproof{
    It was shown in part (a) that $\cTl = \bigcap \cTa$ is a topology on $X$.
    We claim that in fact this is the unique largest topology contained in all $\cTa$.
    First, clearly $\cTl = \bigcap \cTa$ is contained in each $\cTa$ since the intersection of a collection of sets is always a subset of every set in the collection.
    Now suppose that $\cT$ is a topology that is contained in every $\cTa$, i.e. $\cT \ss \cTa$ for every $\cTa$.
    Then clearly for any $U \in \cT$ we have that $U \in \cTa$ for every $\cTa$ so that $U \in \bigcap \cTa = \cTl$.
    Thus $\cT \ss \cTl$ since $U$ was arbitrary.
    This shows that $\cTl$ is the largest such topology since $\cT$ was arbitrary.

    Clearly also $\cTl$ is unique since, if $\cTo$ and $\cTt$ are two such largest topologies that are contained in every $\cTa$.
    Then we would have $\cTo \ss \cTt$ and $\cTt \ss \cTo$ so that $\cTo = \cTt$.
    This also follows from the fact that the largest element in any ordered set (or collection of sets in this case) is unique.
  }

  (c)
  Note that the proofs in part (b) are constructive so that we can construct these topologies as done in the proof.
  For the smallest topology containing $\cTo$ and $\cTt$ we have that
  \gath{
    \bigcup \braces{\cTo, \cTt} = \cTo \cup \cTt = \braces{\es, X, \braces{a}, \braces{a,b}, \braces{b,c}}
  }
  is a subbasis for the smallest topology $\cTs$.
  Then the collection of all finite intersections of elements of this set is a basis for $\cTs$:
  \gath{
    \cB = \braces{\es, X, \braces{a}, \braces{b}, \braces{a,b}, \braces{b,c}} \,.
  }
  Then the topology $\cTs$ is the set of all unions of subcollections of $\cB$:
  \gath{
    \cTs = \braces{\es, X, \braces{a}, \braces{b}, \braces{a,b}, \braces{b,c}} = \cB
  }
  so that evidently the basis and the topology are the same set here!

  For the largest topology contained in $\cTo$ and $\cTt$ we have simply
  \gath{
    \cTl = \bigcap \braces{\cTo, \cTt} = \cTo \cap \cTt = \braces{\es, X, \braces{a}} \,.
  }
}
