% This ensures that the unnumberred section is in the PDF index
\def\stitle{Supplementary Exercises: Well-Ordering}
\currentpdfbookmark{\stitle}{sec:wo}
\subsection*{\stitle} 

% Set section label to text
\renewcommand\thesubsection{WO}

\def\cF{\col{F}}
\exercise{1}{
  \emph{Theorem (General principle of recursive definition).}
  Let $J$ be a well-ordered set; let $C$ be a set.
  Let $\cF$ be the set of all functions mapping sections of $J$ into $C$.
  Given a function $\r : \cF \to C$, there is a unique $h : J \to C$ such that $h(\a) = \r(h \rest S_\a)$ for each $\a \in J$.
  [Hint: Follow the pattern outlined in Exercise~10 of \S 10.]
}\label{tester}
\sol{
  \dwhitman

  Following the hint, we follow the pattern of Exercise~10.10.
  In what follows denote by $(*)$ the property
  \gath{
    h(\a) = \r(h \rest S_\a)
  }
  for a function $h$ from $J$ or a section of $J$ to $C$.
  
  \begin{lem}\label{lem:sewo:hunique}
    If $h$ and $k$ map sections of $J$, or all of $J$, into $C$ and satisfy $(*)$ for all $x$ in their respective domains, then $h(x) = k(x)$ for all $x$ in both domains.
  \end{lem}
  \qproof{
    First suppose that the domains of $h$ and $k$ are sets $H$ and $K$ where each is either a section of $J$ or $J$ itself.
    Since this is the case, we can assume without loss of generality that $H \ss K$  and so $H$ is exactly the domain common to both $h$ and $k$.
    Now suppose that the hypothesis we are trying to prove is \emph{not} true so that there is an $x$ in both domains (i.e. $x \in H$) where $h(x) \neq k(x)$.
    We can also assume that $x$ is the smallest such element since $H \ss J$ and $J$ is well-ordered.
    It then clearly follows that $S_x \ss H$ is a section of $J$ and that $h(y) = k(y)$ for all $y \in S_x$.
    From this we clearly have that $h \rest S_x = k \rest S_x$ so that
    \gath{
      h(x) = \r(h \rest S_x) = \r(k \rest S_x) = k(x)
    }
    since both $h$ and $k$ satisfy $(*)$ and $x$ is in the domain of both.
    This contradicts the supposition that $h(x) \neq k(x)$ so that it must be that no such $x$ exists and hence $h$ and $k$ are the same in their common domain as desired.
  }

  \begin{lem} \label{lem:sewo:hsucc}
    If there exists a function $h: S_\a \to C$ satisfying $(*)$, then there exists a function $k : S_\a \cup \braces{\a} \to C$ satisfying $(*)$.
  \end{lem}
  \qproof{
    Suppose that $h : S_\a \to C$ is such a function satisfying $(*)$.
    Now let $\bar{S}_\a = S_\a \cup \braces{\a}$ and we define $k : \bar{S}_\a \to C$ as follows.
    For any $x \in \bar{S}_\a$ set
    \gath{
      k(x) = \begin{cases}
        h(x) & x \in S_\a \\
        \r(h) & x = \a \,.
      \end{cases}
    }
    We note that clearly $S_\a$ and $\braces{\a}$ are disjoint so that this is unambiguous.
    We also note that $h$ is a function from a section of $J$ to $C$ so that $h \in \cF$ and $\r(h) \in C$ is therefore defined.

    Now we show that $k$ satisfies $(*)$.
    First, clearly $h \rest S_x = k \rest S_x$ for any $x \leq \a$ since $k(y) = h(y)$ by definition for any $y \in S_x \ss S_\a$.
    Now consider any $x \in \bar{S}_\a$.
    If $x = \a$ then by definition we have
    \gath{
      k(x) = \r(h) = \r(h \rest S_\a) = \r(k \rest S_\a) = \r(k \rest S_x)
    }
    since clearly $h = h \rest S_\a$ since $S_\a$ is the domain of $h$.
    On the other hand, if $x \in S_\a$ then $x < \a$ so that
    \gath{
      k(x) = h(x) = \r(h \rest S_x) = \r(k \rest S_x)
    }
    since $h$ satisfies $(*)$.
    Therefore, since $x$ was arbitrary, this shows that $k$ also satisfies $(*)$.
  }

  \begin{lem}\label{lem:sewo:hunion}
    If $K \ss J$ and for all $\a \in K$ there exists a function $h_\a : S_\a \to C$ satisfying $(*)$, then there exists a function
    \gath{
      k : \bigcup_{\a \in K} S_\a \to C
    }
    satisfying $(*)$.
  \end{lem}
  \qproof{
        Let
    \gath{
      k = \bigcup_{\a \in K} h_\a \,,
    }
    which we claim is the function we seek.

    First we show that $k$ is actually a function from $\bigcup_{\a \in K} S_\a$ to $C$.
    So consider any $x$ in the domain of $k$.
    Suppose that $(x, a)$ and $(x, b)$ are both in $k$ so that there are $\a$ and $\b$ in $K$ where $(x, a) \in h_\a$ and $(x, b) \in h_\b$.
    Since $h_\a$ and $h_\b$ both satisfy $(*)$, it follows from Lemma~\ref{lem:sewo:hunique} that $a = h_\a(x) = h_\b(x) = b$ since clearly $x$ is in the domain of both.
    This shows that $k$ is indeed a function since $(x,a)$ and $(x,b)$ were arbitrary.
    Also clearly the domain of $k$ is $\bigcup_{\a \in K} S_\a$ since, for any $x \in \bigcup_{\a \in K} S_\a$, we have that there is an $\a \in K$ where $x \in S_\a$.
    Hence $x$ is in the domain of $h_\a$ and so in the domain of $k$.
    In the other direction, clearly if $x$ is in the domain of $k$ then it is in the domain of $h_\a$ for some $\a \in K$.
    Since this domain is $S_\a$, clearly $x \in \bigcup_{\a \in K} S_\a$.
    Lastly, obviously the range of $k$ can be $C$ since this is the range of every $h_\a$.

    Now we show that $k$ satisfies $(*)$.
    So consider any $x \in \bigcup_{\a \in K} S_\a$ so that $x \in S_\a$ for some $\a \in K$.
    Clearly we have that $k(y) = h_\a(y)$ for every $y \in S_\a$ since $h_\a \ss k$.
    It then immediately follows that $k(x) = h(x)$ and $k \rest S_x = h_\a \rest S_x$ since $S_x \ss S_\a$.
    Then, since $h_\a$ satisfies $(*)$, we have
    \gath{
      k(x) = h_\a(x) = \r(h_\a \rest S_x) = \r(k \rest S_x) \,.
    }
    Since $x$ was arbitrary, this shows that $k$ satisfies $(*)$ as desired.
  }

  \begin{lem} \label{lem:sewo:hind}
    For every $\b \in J$, there exists a function $h_\b : S_\b \to C$ satisfying $(*)$.
  \end{lem}
  \qproof{
    We show this by transfinite induction.
    So consider any $\b \in J$ and suppose that, for every $x \in S_\b$, there is a function $h_x : S_x \to C$ satisfying $(*)$.
    Now, if $\b$ has an immediate predecessor $\a$ then we claim that $S_\b = S_\a \cup \braces{\a}$.
    First if $x \in S_\b$ then $x < \b$ so that $x \leq \a$ since $\a$ is the immediate predecessor of $\b$.
    If $x < \a$ then $x \in S_\a$ and if $x = \a$ then $x \in \braces{\a}$.
    Hence in either case we have that $x \in S_\a \cup \braces{\a}$.
    Now suppose that $x \in S_\a \cup \braces{\a}$.
    If $x \in S_\a$ then $x < \a < \b$ so that $s \in S_\b$.
    On the other hand if $x \in \braces{\a}$ then $x = \a < \b$ so that again $x \in S_\b$.
    Thus we have shown that $S_\b \ss S_\a \cup \braces{\a}$ and $S_\a \cup \braces{\a} \ss S_\b$ so that $S_\b = S_\a \cup \braces{\a}$.
    Since $\a \in S_\b$ it follows that there is an $h_\a : S_\a \to C$ that satisfies $(*)$.
    Then, by Lemma~\ref{lem:sewo:hsucc}, we have that there is an $h_\b : S_\b = S_\a \cup \braces{\a} \to C$ that also satisfies $(*)$.

    If $\b$ does not have an immediate predecessor then we claim that $S_\b = \bigcup_{\g < \b} S_\g$.
    So consider any $x \in S_\b$ so that $x < \b$.
    Since $x$ cannot be the immediate predecessor of $\b$, there must be an $\a$ where $x < \a < \b$.
    Then $x \in S_\a$ so that, since $\a < \b$, clearly $x \in \bigcup_{\g < \b} S_\g$.
    Now suppose that $x \in \bigcup_{\g < \b} S_\g$ so that there is an $\a < \b$ where $x \in S_\a$.
    Then clearly $x < \a < \b$ so that also $x \in S_\b$.
    Thus we have shown that $S_\b \ss \bigcup_{\g < \b} S_\g$ and $\bigcup_{\g < \b} S_\g \ss S_\b$ so that $S_\b = \bigcup_{\g < \b} S_\g$.
    Now, clearly $S_\b$ is a subset of $J$ where there is an $h_x : S_x \to C$ satisfying $(*)$ for every $x \in S_\b$.
    Then it follows from Lemma~\ref{lem:sewo:hunion} that there is a function $h_\b$ from $\bigcup_{\g < S_\b} S_\g = \bigcup_{\g < \b} S_\g = S_\b$ to $C$ that satisfies $(*)$.

    Therefore, in either case, we have shown that there is an $h_\b : S_\b \to C$ that satisfies $(*)$.
    The desired result then follows by transfinite induction.
    }
    
  \mainprob
  \qproof{
    First suppose that $J$ has no largest element.
    Then we claim that $J = \bigcup_{\a \in J} S_\a$.
    For any $x \in J$ there must be a $y \in J$ where $x < y$ since $x$ cannot be the greatest element of $J$.
    Hence $x \in S_y$ so that also clearly $\bigcup_{\a \in J} S_\a$.
    Then, for any $x \in \bigcup_{\a \in J} S_\a$, there is an $\a \in J$ where $x \in S_\a$.
    Clearly $S_\a \ss J$ so that $x \in J$ also.
    Hence $J \ss \bigcup_{\a \in J} S_\a$ and $\bigcup_{\a \in J} S_\a \ss J$ so that $J = \bigcup_{\a \in J} S_\a$.
    Since we know from Lemma~\ref{lem:sewo:hind} that there is an $h_\a : S_\a \to C$ that satisfies $(*)$ for every $\a \in J$, it follows from Lemma~\ref{lem:sewo:hunion} that there is a function $h$ from $\bigcup_{\a \in J} S_\a = J$ to $C$ that satisfies $(*)$.

    If $J$ does have a largest element $\b$ then clearly $J = S_\b \cup \braces{\b}$.
    Since we know that there is an $h_\b : S_\b \to C$ that satisfies $(*)$ by Lemma~\ref{lem:sewo:hind}, it follows from Lemma~\ref{lem:sewo:hsucc} that there is a function $h$ from $S_\b \cup \braces{\b} = J$ to $C$ that satisfies $(*)$.
    Hence the desired function $h$ exists in both cases.
    Lemma~\ref{lem:sewo:hunique} also clearly shows that this function is unique.
  }
}

\exercise{2}{
  \eparts{
  \item Let $J$ and $E$ be well-ordered sets; let $h: J \to E$.
    Show that the following statements are equivalent:
    \begin{enumerate}[label=(\roman*)]
    \item $h$ is order preserving and its image is $E$ or a section of $E$.
    \item $h(\a) = \text{smallest } [E - h(S_\a)]$ for all $\a$.
    \end{enumerate}
        [Hint: Show that each of these conditions implies that $h(S_\a)$ is a section of $E$; conclude that it must be the section by $h(\a)$.]
      \item If $E$ is a well-ordered set, show that no section of $E$ has the order type of $E$, nor do two different sections of $E$ have the same order type.
        [Hint: Given $J$, there is a most one order preserving map of $J$ into $E$ whose image is $E$ or a section of $E$.]
  }
}
\sol{
  \dwhitman

  (a)
  \qproof{
    First, for any $\a \in J$ and $\b \in E$, let $S_\a$ denote the section of $J$ by $\a$, and $T_\b$ denote the section of $E$ by $\b$.
    To avoid ambiguity, also suppose that $<$ is the well-order on $J$ and $\prec$ is the well-order on $E$.
    We show that each of these conditions are equivalent to the condition that $h(S_\a) = T_{h(\a)}$ for every $\a \in J$.
    Call this condition (iii).
    This of course also shows that the conditions are equivalent to each other.

    First we show that (i) implies (iii).
    So suppose that $h$ is order preserving and its image is $E$ or a section of $E$.
    Consider any $\a \in J$ and any $y \in h(S_\a)$ so that there is an $x \in S_\a$ where $y = h(x)$.
    Then $x < \a$ and $y = h(x) \prec h(\a)$ since $h$ preserves order.
    Therefore $y \in T_{h(\a)}$ so that $h(S_\a) \ss T_{h(\a)}$ since $y$ was arbitrary.
    Now consider $y \in T_{h(\a)}$ so that $y \prec h(\a)$.
    Since also clearly $y \in E$ (since $T_{h(\a)} \ss E$), $y$ is in the image of $h$ if its image is all of $E$.
    If the image of $h$ is some section of $E$, say $T_\b$, then clearly $h(\a) \in T_\b$ since $h(\a)$ is obviously in the image of $h$.
    Hence we have $y \prec h(\a) \prec \b$ so that $y \in T_\b$ and hence in the image of $h$.
    Since $y$ is in the image of $h$ in either case, there is an $x \in J$ such that $y = h(x)$.
    Then $h(x) = y \prec h(\a)$ so that $x < \a$ since $h$ preserves order.
    Hence $x \in S_\a$ so that $y \in h(S_\a)$ since $y = h(x)$.
    This shows that $T_{h(\a)} \ss h(S_\a)$ since $y$ was arbitrary.
    Therefore $h(S_\a) = T_{h(\a)}$ so that condition (iii) is true since $\a$ was arbitrary.

    Next we show that (iii) implies (i).
    So suppose that $h(S_\a) = T_{h(\a)}$ for all $\a \in J$.
    First, it is easy to see that $h$ preserves order since, if $x,y \in J$ where $x < y$, then we have that $x \in S_y$ so that clearly $h(x) \in h(S_y) = T_{h(y)}$, and hence $h(x) < h(y)$.
    To show that the image of $h$, i.e. $h(J)$, is either $E$ or a section of $E$, consider the set $E - h(J)$.

    Case: $E - h(J) = \es$.
    Then clearly for any $y \in E$ we must have that $y \in h(J)$ since otherwise it would be that $y \in E - h(J)$.
    Thus $E \ss h(J)$ since $y$ was arbitrary.
    Also clearly $h(J) \ss E$ since $E$ is the range of $h$.
    This shows that $h(J) = E$.

    Case: $E - h(J) \neq \es$.
    Then clearly $E - h(J)$ is a nonempty subset of $E$ so that it has a smallest element $\b$ since $E$ is well-ordered, noting that clearly $\b \notin h(J)$.
    We claim that $h(J) = T_\b$.
    So consider any $y \in h(J)$ so that there is an $x \in J$ where $y = h(x)$.
    Suppose for a moment that $\b \eprec y$.
    Now it cannot be that $\b = y$ since $y \in h(J)$ but $\b \notin h(J)$, and so $\b \prec y$.
    But then $\b \in T_y = T_{h(x)} = h(S_x)$ since $x \in J$.
    Then $\b$ is in the image of $h$ since clearly $h(S_x) \ss h(J)$.
    As this contradicts the fact that $\b \notin h(J)$, it must be that $y \prec \b$ so that $y \in T_\b$.
    This shows that $h(J) \ss T_\b$ since $y$ was arbitrary.
    Suppose now that $y \in T_\b$ so that $y \prec \b$.
    Since $\b$ is the smallest element of $E - h(J)$, it follows that $y \notin E - h(J)$.
    Since clearly $y \in E$ (since $T_\b \ss E$), it must be that $y \in h(J)$.
    This shows that $T_\b \ss h(J)$ since $y$ was arbitrary.
    Hence we have shown that $h(J) = T_\b$.

    Therefore in every case either the image of $h$ is $E$ or a section of $E$ as desired.
    This completes the proof of (i).

    Now we show that (ii) implies (iii).
    So suppose that $h(\a)$ is the smallest element of $E - h(S_\a)$ for every $\a \in J$.
    First we show that $h$ is injective.
    So consider any $x,y \in J$ where $x \neq y$.
    We can assume without loss of generality that $x < y$ so that $x \in S_y$ and hence $h(x) \in h(S_y)$.
    However since we have that $h(y)$ is the smallest element of $E - h(S_y)$, clearly $h(y) \notin h(S_y)$.
    Therefore we have that $h(x) \neq h(y)$ so that $h$ is injective.
    
    Now consider any $\a \in J$ so that clearly $h(\a)$ is the smallest element of $E - h(S_\a)$.
    Suppose that $y \in h(S_\a)$ so that there is an $x \in S_\a$ where $y = h(x)$, and therefore $x < \a$.
    Consider the possibility that $h(\a) \prece h(x) = y$.
    It cannot be that $h(\a) = h(x) = y$ since $x \neq \a$ and $h$ is injective, so it must be that $h(\a) \prec h(x)$.
    It then follows that $h(\a) \notin E - h(S_x)$ since $h(x)$ is the smallest element of $E - h(S_x)$.
    Thus $h(\a) \in h(S_x)$ since clearly $h(\a) \in E$.
    It then follows from the fact that $h$ is injective that $\a \in S_x$ so that we have $\a < x < \a$, which is clearly a contradiction.
    So it must be that $y = h(x) \prec h(\a)$ so that $y \in T_{h(\a)}$.
    This shows that $h(S_\a) \ss T_{h(\a)}$ since $y$ was arbitrary.
        
    Now suppose that $y \in T_{h(\a)}$ so that $y \prec h(\a)$.
    Since $h(\a)$ is the smallest element of $E - h(S_\a)$, it follows that $y \notin E - h(S_\a)$.
    Since clearly $y \in E$, it must be that $y \in h(S_\a)$.
    This shows that $T_{h(\a)} \ss h(S_\a)$ since $y$ was arbitrary, and hence $h(S_\a) = T_{h(\a)}$, which shows (iii) since $\a$ was arbitrary.

    Lastly, we show that (iii) implies (ii).
    So suppose that $h(S_\a) = T_{h(\a)}$ for every $\a \in J$ and consider any such $\a$.
    Clearly we have that $h(\a) \in E$ but $h(\a) \notin T_{h(\a)} = h(S_\a)$ so that $h(\a) \in E - h(S_\a)$.
    Suppose for the moment that $h(\a)$ is not the smallest element of $E - h(S_\a)$ so that there is a $\b \in E - h(S_\a)$ where $\b \prec h(\a)$.
    Then $\b \in T_{h(\a)}$ so that it must be that $\b \notin E - T_{h(\a)} = E - h(S_\a)$ since $h(S_\a) = T_{h(\a)}$.
    Clearly this is a contradiction so that it must be that $h(\a)$ really is the smallest element of $E - h(S_\a)$, which shows (ii) since $\a$ was arbitrary.
  }
}

\exercise{3}{
  Let $J$ and $E$ be well-ordered sets; suppose there is an order preserving map $k: J \to E$.
  Using Exercises~1 and 2, show that $J$ has the order type of $E$ or a section of $E$.
  [Hint: Choose $e_0 \in E$.
    Define $h: J \to E$ by the recursion formula
    \gath{
      h(\a) = \text{smallest } [E - h(S_\a)] \condgap \text{if} \condgap h(S_\a) \neq E \,,
    }
    and $h(\a) = e_0$ otherwise.
    Show that $h(\a) \leq k(\a)$ for all $\a$; conclude that $h(S_\a) \neq E$ for all $\a$.]
}
\sol{
  \dwhitman

  \qproof{
    First, if $E = \es$ then it must be that $J = \es$ as well so that they vacuously have the same order type.
    Otherwise, following the hint, choose $e_0 \in E$ and define $h: J \to E$ by
    \gath{
      h(\a) = \text{smallest } [E - h(S_\a)] \condgap \text{if} \condgap h(S_\a) \neq E \,,
    }
    and $h(\a) = e_0$ otherwise, noting that this function is uniquely defined by the general principle of recursive definition (Exercise~\thesubsection.1).
    We show that $h(\a) \leq k(\a)$ for all $\a \in J$ using transfinite induction (see Lemma~\ref{lem:wellord:tind}).
    So consider $\a \in J$ and assume that $h(x) \leq k(x)$ for all $x \in S_\a$.
    Since $k$ preserves order we have that $h(x) \leq k(x) < k(\a)$ when $x < \a$.
    In particular, this means that $h(x) \neq k(\a)$ for all $x \in S_\a$ so that $k(\a) \in E - h(S_\a)$.
    Hence $E - h(S_\a)$ is not empty so that $h(S_\a) \neq E$.
    Thus $h(\a)$ is the smallest element of $E - h(S_\a)$ and so $h(\a) \leq k(\a)$ since $k(\a) \in E - h(S_\a)$.
    This completes the induction.

    Therefore, for any $\a \in J$ and any $x < \a$ we have $h(x) \leq k(x) < k(\a)$ since $k$ preserves order so that $h(x) \neq k(\a)$.
    As in the induction step above, it follows that $h(S_\a) \neq E$.
    Hence, since $\a$ was arbitrary,
    \gath{
      h(\a) = \text{smallest } [E - h(S_\a)]
    }
    for all $\a \in J$.
    It then follows from Exercise~\thesubsection.2 part (a) that $h$ is order preserving and maps $J$ onto $E$ or a section of $E$.
    This clearly shows that $J$ has the order type of $E$ or a section of $E$ as desired.
  }
}

\exercise{4}{
  Use Exercises 1-3 to prove the following:
  \eparts{
  \item If $A$ and $B$ are well-ordered sets, then exactly one of the following three conditions holds: $A$ and $B$ have the same order type, or $A$ has the order type of a section of $B$, or $B$ has the order type of a section of $A$.
    [Hint: Form a well-ordered set containing both $A$ and $B$, as in Exercise~8 of \S 10; then apply the preceding exercise.]
  \item Suppose that $A$ and $B$ are well-ordered sets that are uncountable, such that every section of $A$ and $B$ is countable.
    Show that $A$ and $B$ have the same order type.
  }
}
\sol{
  \dwhitman

  (a)
  \qproof{
    First, we can assume that $A$ and $B$ are disjoint since, if not, we can form $A' = \braces{(x,1) \where x \in A}$ and $B' = \braces{(x,2) \where x \in B}$, which clearly \emph{are} disjoint and have the same order types as $A$ and $B$ if ordered in the same way.
    So let $\prec$ be the order on $A \cup B$ as in Exercise~10.8 with all the elements of $A$ before the elements of $B$.
    From the exercise, we know that $A \cup B$ is well-ordered by $\prec$.
    Now, clearly the identity function $i_B$ with $A \cup B$ as the range is an order-preserving function from $B$ to $A \cup B$ so that $B$ is the same order type as $A \cup B$ or a section of $A \cup B$ by Exercise~\thesubsection.3.

    If $B$ has the same order type as $A \cup B$, then there is a an order preserving bijection $g: A \cup B \to B$.
    Let $b$ be the smallest element of $B$ so that $y = g(b) \in B$.
    Since $b$ is the smallest element of $B$, clearly the section $S_b = \braces{x \in A \cup B \where x \prec b} = A$.
    Also clearly $g(A) = g(S_b) = S_y = \braces{x \in B \where x < y}$ so that $A$ has the same order type as a section of $B$ since $g$ preserves order.

    If $B$ has the same order type as a section of $A \cup B$ then there is an order preserving bijection $f: B \to S_\a$ for some $\a \in A \cup B$.
    If $\a \in A$ then clearly $S_\a$ lies entirely in $A$ and is a section of $A$ so that $B$ has the same order type as a section of $A$.
    So now suppose that $\a \in B$.
    If $\a$ is the smallest element of $B$ then again it has to be that $S_\a$ lies in $A$ and is in fact the entirety of $A$ so that $B$ and $A$ have the same order type.
    If $\a$ is not the smallest element of $B$ then $S_\a$ contains elements of both $A$ and $B$.
    So let $b$ be the smallest element of $B$ so that $b \in S_\a$, and let $y \in B$ be such that $f(y) = b$, which exists since $f$ is surjective.
    We also have that $S_b = A$ since $b$ is the smallest element of $B$.
    It then follows that $f(S_y) = S_b = A$ since $f(y) = b$ so that $A$ has the same order type as the section $S_y$ of $B$ since $f$ preserves order.

    Hence in all cases one of the desired results always follows.
    To show that exactly one of these is the case, note that if $A$ and $B$ have the same order type then clearly it cannot be that $A$ has the same order type as a section of $B$ since then $B$ would also have the same order type is its own section, which would violate Exercise~\thesubsection.2b.
    Similarly $B$ cannot have the same order type as a section of $A$ since then $A$ would have the same order type as its own section.
    Now suppose that $A$ has the same order type as a section $S_b$ of $B$.
    Then $A$ and $B$ cannot have the same order type since then $B$ would have the same order type as its section $S_b$.
    Also $B$ cannot have the same order type as a section $S_a$ of $A$ since then the section $S_b$, and therefore $A$, would have the same order type as a smaller section of $A$.
    An analogous argument shows the result when $B$ has the same order type as a section of $A$.
  }

  (b)
  \qproof{
    Suppose that $A$ has the same order type as a section of $B$.
    Then there would be a bijection from $A$, an uncountable set, to a section of $B$, which is countable.
    A similar contradiction arises if $B$ were to have the same order type as a section of $A$.
    By part (a), the only remaining possibility is that $A$ and $B$ have the same order type as desired.
  }
}

\exercise{5}{
  Let $X$ be a set; let $\cA$ be the collection of all pairs $(A, <)$, where $A$ is a subset of $X$ and $<$ is a well-ordering of $A$.
  Define
  \gath{
    (A, <) \prec (A', <')
  }
  if $(A, <)$ \emph{equals} a section of $(A', <')$.
  \eparts{
  \item Show that $\prec$ is a strict partial order on $\cA$.
  \item Let $\cB$ be a subcollection of $\cA$ that is simply ordered by $\prec$.
    Define $B'$ to be the union of the sets $B$, for all $(B, <) \in \cB$; and define $<'$ to be the union of the relations $<$, for all $(B, <) \in \cB$.
    Show that $(B', <')$ is a well-ordered set.
  }
}
\sol{
  \dwhitman

  (a)
  \qproof{
    For any $(A,<) \in \cA$ we have that it is not equal to a section of itself since then it would then clearly have the same order type as its own section, which would violate Exercise~\thesubsection.2b.
    Hence it is not true that $(A, <) \prec (A, <)$ by definition, which shows that $\prec$ is nonreflexive.

    Now consider $(A,<)$, $(A',<')$, and $(A'',<'')$ in $\cA$ where $(A,<) \prec (A',<')$ and $(A',<') \prec (A'',<'')$.
    Then $(A,<)$ is a section of $(A',<')$.
    Also $(A',<')$ is a section of $(A'',<'')$ so that clearly any section of $(A',<')$ is also a section of $(A'',<'')$.
    Since $(A,<)$ is such a section we have that $(A,<)$ is a section of $(A'',<'')$ so that $(A,<) \prec (A'',<'')$.
    This shows that $\prec$ is transitive.

    This completes the proof that $\prec$ is a strict partial order.
  }

  (b)
  \qproof{
    First we must show that $B'$ is simply ordered by $<'$.

    First consider any $(x,y) \in <'$ so that there is a $(B,<) \in \cB$ where $(x,y) \in <$ and $x,y \in B$.
    Clearly then $x$ and $y$ are in the union $B'$ so that $(x,y) \in B' \times B'$.
    This shows that $<' \ss B' \times B'$ so that $<'$ is a relation on $B'$.

    Next consider any $x$ and $y$ in $B'$ where $x \neq y$.
    Then there are well-ordered sets $(B_1, <_1)$ and $(B_2, <_2)$ in $\cB$ where $x \in B_1$ and $y \in B_2$.
    Since $\cB$ is simply ordered by $\prec$ we have that $(B_1, <_1) \prec (B_2, <_2)$ or $(B_2, <_2) \prec (B_1, <_1)$.
    Without loss of generality we can assume the former case (since otherwise we can just swap the roles of $x$ and $y$).
    Then $(B_1 <_1)$ is a section of $(B_2, <_2)$ and is thus also a subset so that $x,y \in B_2$,
    It then follows that $x$ and $y$ are comparable by $<_2$ since $x \neq y$ and $<_2$ is a well-order and therefore a simple order.
    Thus $(x,y)$ or $(y,x)$ are in $<_2$.
    Since $<'$ is the union of all relations $<$ where $(B,<) \in \cB$, clearly we have that $(x,y)$ or $(y,x)$ are in $<'$ since $<_2$ is such a relation.
    Thus shows that $<'$ has the comparability property.

    Now consider any $x \in B'$ so that there is a $(B,<) \in \cB$ where $x \in B$.
    Consider also any $(B'', <'') \in \cB$.
    Then, since $\cB$ is simply ordered, it follows that $(B,<)$ and $(B'', <'')$ are comparable in $\prec$.
    If $(B,<) \prec (B'',<'')$ then $(B,<)$ is a section of $(B'',<'')$ so that $x \in B''$ as well.
    Then it cannot be that $x <'' x$ since $<''$ is a simple order.
    If $(B'',<'') \prec (B,<)$ then $(B'',<'')$ is a section of $(B,<)$.
    If $x \in B''$ then again it cannot be that $x <'' x$ since $<''$ is a simple order.
    If $x \notin B''$ then $(x, x) \notin <''$ since it is a relation on $B''$.
    Thus in all cases and sub-cases it is not true that $x <'' x$ so that $x <' x$ does not hold since $<''$ was arbitrary and $<'$ is their union.
    This shows that $<'$ is nonreflexive.

    Lastly, suppose that $x <' y$ and $y <' z$.
    Then it has to be that there is a $(B_1, <_1)$ and $(B_2, <_2)$ in $\cB$ where $z <_1 y$ and  $y <_2 z$.
    Then $(B_1, <_1)$ and $(B_2, <_2)$ are comparable in $\prec$ since $\cB$ is simply ordered.
    Hence one is a section of the other so that, in either case, it follows that $x < y$ and $y < z$ where either $< = <_1$ or $< = <_2$.
    Then clearly $x < z$ since both $<_1$ and $<_2$ are transitive since they are simple orders.
    Thus $x <' z$ since $<'$ is the union of all the orders in $\cB$ and $<$ is such an order.
    This shows that $<'$ is transitive.

    This completes the proof that $<'$ is a simple order on $B'$.
    To show that it is a well-order, consider any nonempty subset $A \ss B'$.
    Then there is an $x \in A$ so that $x \in B'$ as well.
    It then follows that there is a $(B, <) \in \cB$ where $x \in B$.
    Then clearly $B \cap A$ is a nonempty subset of $B$ since $x \in B$ and $x \in A$.
    Let $b$ be the $<$-smallest element in $B \cap A$, and we claim that this is the smallest element of $A$ by $<'$.
    First, obviously $b \in A$ since $b \in B \cap A$.
    Next consider any $y \in A$ so that $y \in B'$ as well.
    Then there is a $(B'', <'') \in \cB$ where $y \in B''$.
    Since $\cB$ is simply ordered by $\prec$ we have that $(B,<)$ and $(B'',<'')$ are comparable.
    Hence $(B,<)$ is a section of $(B'',<'')$ or vice-versa.
    
    In the first case we have that both $b$ and $y$ are in $B''$.
    If $y \in B$ then also $y \in B \cap A$ so that $b \leq y$ since it is the smallest element of $B \cap A$ by $<$.
    If $y \notin B$ then $b <'' y$ since $B$ is a section of $B''$, and therefore $b \leq'' y$ is true.
    In the second case in which $(B'',<'')$ is a section of $(B,<)$ we have that both $b$ and $y$ are in $B$ and hence in $B \cap A$.
    Then, again $b \leq y$ since $b$ is the smallest element of $B \cap A$ by $<$.
    Hence in all cases either $b \leq y$ or $b \leq'' y$.
    Either way it follows that $b \leq' y$ as well since $<'$ is the union.
    This shows that $b$ is the least element of $A$ by $<'$ as desired.
    Since $A$ was an arbitrary nonempty subset, this shows that $B'$ is well-ordered by $<'$.
  }
}

\exercise{6}{
  Use Exercises~1 and 5 to prove the following:

  \emph{Theorem.}
  The maximum principle is equivalent to the well-ordering theorem.
}
\sol{
  \dwhitman

  \qproof{
    First suppose that the maximum principle is true and let $X$ be any set.
    Then let $\cA$ be the collection of all pairs $(A,<)$, where $A \ss X$ and $<$ is a well-ordering of $A$ as in Exercise~\thesubsection.5.
    Define the relation $\prec$ on $\cA$ also as in Exercise~\thesubsection.5, i.e $(A,<) \prec (A', <')$ if $(A,<)$ is a section of $(A',<')$.
    It was then shown in that exercise that $\prec$ is a strict partial order on $\cA$ so that, by the maximum principle, there is a maximal simply ordered subset $\cB \ss \cA$.
    Now let $(B', <')$ be the unions of the corresponding elements of $\cB$ so that we know that $<'$ well-orders $B'$ by part (b) of Exercise~\thesubsection.5.

    We claim that $B' = X$.
    Suppose that this is not the case so that there is a $x \in X$ where $x \notin B'$ (since we know that $B' \ss X$).
    Then define $B'' = B' \cup \braces{x}$ and the relation $<'' = <' \cup \braces{(b', x) \where b' \in B'}$.
    It is then easy to see (and trivial but tedious to show) that $B''$ is well-ordered by $<''$.
    Also, clearly $B'$ is the section of $B'' $ by $x$ so that, for any $B \in \cB$, we have $B \eprec B' \prec B''$.
    Since $B$ was arbitrary, this shows that the set $\cB \cup \braces{B''}$ is simply ordered by $\prec$ and is a subset of $\cA$.
    Since $x \notin B'$ we have that $x \notin B$ for any $B \in \cB$ (since $B'$ is their union) so that $B'' \neq B$ since $x \in B$.
    It follows that $\cB \pss \cB \cup \braces{B''}$, but this contradicts the maximality of $\cB$!
    So it has to be that in fact $B' = X$ itself so that $<'$ is a well-ordering of $X$.
    Since $X$ was an arbitrary set, this shows the well-ordering theorem.

    Now suppose the well-ordering theorem and that $A$ is a set with strict partial ordering $\prec$.
    Then we know that $A$ has a well-ordering, say $<$.
    Now, for any function $f$ from a section $S_x$ (by $<$) to $\pset{A}$, define
    \gath{
      \r(f) = \begin{cases}
        \bigcup f(S_x) \cup \braces{x} & \text{if $\prec$ is a simple order on $\bigcup f(S_x) \cup \braces{x}$} \\
        \bigcup f(S_x) & \text{otherwise} \,.
      \end{cases}
    }
    Then by the general principle of recursive defamation (Exercise~\thesubsection.1) there is a unique function $h : A \to \pset{A}$ such that $h(\a) = \r(h \rest S_\a)$ for all $\a \in A$.

    First we show that, for $\a,\b \in A$ where $\a < \b$, we have $h(\a) \ss h(\b)$.
    So consider any $x \in h(\a) = \r(h \rest S_\a)$, and hence either $x \in \bigcup h(S_\a) \cup \braces{\a}$ or $x \in \bigcup h(S_\a)$.
    Either way obviously $x \in \bigcup h(S_\a)$ so that there is a set $X \in h(S_\a)$ where $x \in X$.
    Then there is a $\g \in S_\a$ where $X = h(\g)$.
    Since we have $\a < \b$, clearly also $\g \in S_\b$ and hence $X \in h(S_\b)$.
    Then also clearly both $x \in \bigcup h(S_\b) \cup \braces{\b}$ and $x \in \bigcup h(S_\b)$ so that for sure $x \in \r(h \rest S_\b) = h(\b)$.
    Since $x$ was arbitrary this shows that $h(\a) \ss h(\b)$ as desired.

    Next we show by transfinite induction that $h(\a)$ is simply ordered by $\prec$ for every $\a \in A$.
    So consider $\a \in A$ and suppose that $h(\b)$ is simply ordered by $\prec$ for every $\b < \a$.
    If $\bigcup h(S_\a) \cup \braces{\a}$ is simply ordered by $\prec$ then clearly $h(\a)$ is since then $h(\a) = \r(h \rest S_\a) = \bigcup h(S_\a) \cup \braces{\a}$.
    So suppose that this is not the case so that $h(\a) = \r(h \rest S_\a) = \bigcup h(S_\a)$.
    Consider then any $x,y \in h(\a) = \bigcup h(S_\a)$ where $x \neq y$ so that there are $X$ and $Y$ in $h(S_\a)$ where $x \in X$ and $y \in Y$.
    Then there is a $\b$ and $\g$ in $S_\a$ where $X = h(\b)$ and $Y = h(\g)$.
    If $\b = \g$ then $x$ and $y$ are both in $X = h(\b) = h(\g) = Y$, which is simply ordered by the induction hypothesis so that $x$ and $y$ are comparable in $\prec$.
    If $\b < \g$ then by what was shown above we have that $x \in X = h(\b) \ss h(\g)$ so that $x$ and $y$ are both in $h(\g)$, which is simply ordered by the induction hypothesis so that again $x$ and $y$ are comparable.
    A similar argument shows that $x$ and $y$ are both in $h(\b)$ and thus are comparable when $\b > \g$.
    This completes the induction since $x$ and $y$ are comparable in all cases so that $h(\a)$ is always simply ordered.

    We then claim that the set $B = \bigcup_{\a \in A} h(\a)$ is a maximal simply ordered (by $\prec$) subset of $A$, which of course shows the maximum principle.
    First, it is obviously a subset of $A$ since each $h(\a) \in \pset{A}$ so and so is a subset of $A$.
    To show that that $B$ is simply ordered by $\prec$, consider $x$ and $y$ in $B$ where $x \neq y$ so that there is an $\a$ and $\b$ in $A$ where $x \in h(\a)$ and $y \in h(\b)$.
    Without loss of generality we can assume that $\a < \b$ so that $h(\a) \ss h(\b)$ by what was shown below.
    Then both $x$ and $y$ are in $h(\b)$, which is simply ordered by what was shown above.
    Hence $x$ and $y$ are comparable in $\prec$ so that $B$ is simply ordered.

    To show that $B$ is maximal, suppose that $B \pss Z$ and $Z \ss A$ is simply ordered by $\prec$.
    Then there is a $z \in Z$ where $z \notin B$.
    Now let $x \in \bigcup h(S_z)$ so that there is an $X \in h(S_z)$ where $x \in X$.
    Then there is an $\a \in S_z$ where $x \in X = h(\a)$.
    Hence clearly $x \in B$ so that also $x \in Z$ and so $x$ and $z$ are comparable in $\prec$ since $Z$ is simply ordered.
    Since $x$ was arbitrary this shows that the set $\bigcup h(S_z) \cup \braces{z}$ is simply ordered so that $h(z) = \r(h \rest S_z) = \bigcup h(S_z) \cup \braces{z}$.
    However, then we have that $z \in h(z)$ so that $z \in B$, which is a contradiction.
    So it must be that there is no such set $Z$ and hence $B$ is maximal.
  } 
}

% Restore section label to number
\renewcommand\thesubsection{\arabic{subsection}}
